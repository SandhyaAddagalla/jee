\renewcommand{\theequation}{\theenumi}
\begin{enumerate}[label=\arabic*.,ref=\thesubsection.\theenumi]
\numberwithin{equation}{enumi}

%\documentclass[journal,12pt,twocolumn]{IEEEtran}
%\usepackage{setspace}
%\usepackage{gensymb}
%\usepackage{caption}
%\usepackage{subfiles}
%%\usepackage{multirow}
%%\usepackage{multicolumn}
%%\usepackage{subcaption}
%%\doublespacing
%\singlespacing
%\usepackage{csvsimple}
%\usepackage{amsmath}
%\usepackage{multicol}
%%\usepackage{enumerate}
%\usepackage{amssymb}
%%\usepackage{graphicx}
%\usepackage{newfloat}
%%\usepackage{syntax}
%\usepackage{listings}
%%\usepackage{iithtlc}
%\usepackage{color}
%\usepackage{tikz}
%\usetikzlibrary{shapes,arrows}
%
%
%
%%\usepackage{graphicx}
%%\usepackage{amssymb}
%%\usepackage{relsize}
%%\usepackage[cmex10]{amsmath}
%%\usepackage{mathtools}
%%\usepackage{amsthm}
%%\interdisplaylinepenalty=2500
%%\savesymbol{iint}
%%\usepackage{txfonts}
%%\restoresymbol{TXF}{iint}
%%\usepackage{wasysym}
%\usepackage{amsthm}
%\usepackage{mathrsfs}
%\usepackage{txfonts}
%\usepackage{stfloats}
%\usepackage{cite}
%\usepackage{cases}
%\usepackage{mathtools}
%\usepackage{caption}
%\usepackage{enumerate}
%\usepackage{tfrupee}	
%\usepackage{enumitem}
%\usepackage{amsmath}
%%\usepackage{xtab}
%\usepackage{longtable}
%\usepackage{multirow}
%%\usepackage{algorithm}
%%\usepackage{algpseudocode}
%\usepackage{enumitem}
%\usepackage{mathtools}
%\usepackage{hyperref}
%%\usepackage[framemethod=tikz]{mdframed}
%\usepackage{listings}
%    %\usepackage[latin1]{inputenc}                                 %%
%    \usepackage{color}                                            %%
%    \usepackage{array}                                            %%
%    \usepackage{longtable}                                        %%
%    \usepackage{calc}                                             %%
%    \usepackage{multirow}                                         %%
%    \usepackage{hhline}                                           %%
%    \usepackage{ifthen}                                           %%
%  %optionally (for landscape tables embedded in another document): %%
%    \usepackage{lscape}     
%
%
%\usepackage{url}
%\def\UrlBreaks{\do\/\do-}
%
%
%%\usepackage{stmaryrd}
%
%
%%\usepackage{wasysym}
%%\newcounter{MYtempeqncnt}
%\DeclareMathOperator*{\Res}{Res}
%%\renewcommand{\baselinestretch}{2}
%\renewcommand\thesection{\arabic{section}}
%\renewcommand\thesubsection{\thesection.\arabic{subsection}}
%\renewcommand\thesubsubsection{\thesubsection.\arabic{subsubsection}}
%
%\renewcommand\thesectiondis{\arabic{section}}
%\renewcommand\thesubsectiondis{\thesectiondis.\arabic{subsection}}
%\renewcommand\thesubsubsectiondis{\thesubsectiondis.\arabic{subsubsection}}
%
%% correct bad hyphenation here
%\hyphenation{op-tical net-works semi-conduc-tor}
%
%%\lstset{
%%language=C,
%%frame=single, 
%%breaklines=true
%%}
%
%%\lstset{
%	%%basicstyle=\small\ttfamily\bfseries,
%	%%numberstyle=\small\ttfamily,
%	%language=Octave,
%	%backgroundcolor=\color{white},
%	%%frame=single,
%	%%keywordstyle=\bfseries,
%	%%breaklines=true,
%	%%showstringspaces=false,
%	%%xleftmargin=-10mm,
%	%%aboveskip=-1mm,
%	%%belowskip=0mm
%%}
%
%%\surroundwithmdframed[width=\columnwidth]{lstlisting}
%\def\inputGnumericTable{}                                 %%
%\lstset{
%%language=C,
%frame=single, 
%breaklines=true,
%columns=fullflexible
%}
% 
%
%\begin{document}
%%
%\tikzstyle{block} = [rectangle, draw,
%    text width=3em, text centered, minimum height=3em]
%\tikzstyle{sum} = [draw, circle, node distance=3cm]
%\tikzstyle{input} = [coordinate]
%\tikzstyle{output} = [coordinate]
%\tikzstyle{pinstyle} = [pin edge={to-,thin,black}]
%
%\theoremstyle{definition}
%\newtheorem{theorem}{Theorem}[section]
%\newtheorem{problem}{Problem}
%\newtheorem{proposition}{Proposition}[section]
%\newtheorem{lemma}{Lemma}[section]
%\newtheorem{corollary}[theorem]{Corollary}
%\newtheorem{example}{Example}[section]
%\newtheorem{definition}{Definition}[section]
%%\newtheorem{algorithm}{Algorithm}[section]
%%\newtheorem{cor}{Corollary}
%\newcommand{\BEQA}{\begin{eqnarray}}
%\newcommand{\EEQA}{\end{eqnarray}}
%\newcommand{\define}{\stackrel{\triangle}{=}}
%
%\bibliographystyle{IEEEtran}
%%\bibliographystyle{ieeetr}
%
%\providecommand{\nCr}[2]{\,^{#1}C_{#2}} % nCr
%\providecommand{\nPr}[2]{\,^{#1}P_{#2}} % nPr
%\providecommand{\mbf}{\mathbf}
%\providecommand{\pr}[1]{\ensuremath{\Pr\left(#1\right)}}
%\providecommand{\qfunc}[1]{\ensuremath{Q\left(#1\right)}}
%\providecommand{\sbrak}[1]{\ensuremath{{}\left[#1\right]}}
%\providecommand{\lsbrak}[1]{\ensuremath{{}\left[#1\right.}}
%\providecommand{\rsbrak}[1]{\ensuremath{{}\left.#1\right]}}
%\providecommand{\brak}[1]{\ensuremath{\left(#1\right)}}
%\providecommand{\lbrak}[1]{\ensuremath{\left(#1\right.}}
%\providecommand{\rbrak}[1]{\ensuremath{\left.#1\right)}}
%\providecommand{\cbrak}[1]{\ensuremath{\left\{#1\right\}}}
%\providecommand{\lcbrak}[1]{\ensuremath{\left\{#1\right.}}
%\providecommand{\rcbrak}[1]{\ensuremath{\left.#1\right\}}}
%\theoremstyle{remark}
%\newtheorem{rem}{Remark}
%\newcommand{\sgn}{\mathop{\mathrm{sgn}}}
%\providecommand{\abs}[1]{\left\vert#1\right\vert}
%\providecommand{\res}[1]{\Res\displaylimits_{#1}} 
%\providecommand{\norm}[1]{\left\Vert#1\right\Vert}
%\providecommand{\mtx}[1]{\mathbf{#1}}
%\providecommand{\mean}[1]{E\left[ #1 \right]}
%\providecommand{\fourier}{\overset{\mathcal{F}}{ \rightleftharpoons}}
%%\providecommand{\hilbert}{\overset{\mathcal{H}}{ \rightleftharpoons}}
%\providecommand{\system}{\overset{\mathcal{H}}{ \longleftrightarrow}}
%	%\newcommand{\solution}[2]{\textbf{Solution:}{#1}}
%\newcommand{\solution}{\noindent \textbf{Solution: }}
%\newcommand{\myvec}[1]{\ensuremath{\begin{pmatrix}#1\end{pmatrix}}}
%\providecommand{\dec}[2]{\ensuremath{\overset{#1}{\underset{#2}{\gtrless}}}}
%\DeclarePairedDelimiter{\ceil}{\lceil}{\rceil}
%%\numberwithin{equation}{section}
%%\numberwithin{problem}{subsection}
%%\numberwithin{definition}{subsection}
%\makeatletter
%\@addtoreset{figure}{section}
%\makeatother
%
%\let\StandardTheFigure\thefigure
%%\renewcommand{\thefigure}{\theproblem.\arabic{figure}}
%\renewcommand{\thefigure}{\thesection}
%
%
%%\numberwithin{figure}{subsection}
%
%%\numberwithin{equation}{subsection}
%%\numberwithin{equation}{section}
%%\numberwithin{equation}{problem}
%%\numberwithin{problem}{subsection}
%\numberwithin{problem}{section}
%%%\numberwithin{definition}{subsection}
%%\makeatletter
%%\@addtoreset{figure}{problem}
%%\makeatother
%\makeatletter
%\@addtoreset{table}{section}
%\makeatother
%
%\let\StandardTheFigure\thefigure
%\let\StandardTheTable\thetable
%\let\vec\mathbf
%%%\renewcommand{\thefigure}{\theproblem.\arabic{figure}}
%%\renewcommand{\thefigure}{\theproblem}
%
%%%\numberwithin{figure}{section}
%
%%%\numberwithin{figure}{subsection}
%
%
%
%\def\putbox#1#2#3{\makebox[0in][l]{\makebox[#1][l]{}\raisebox{\baselineskip}[0in][0in]{\raisebox{#2}[0in][0in]{#3}}}}
%     \def\rightbox#1{\makebox[0in][r]{#1}}
%     \def\centbox#1{\makebox[0in]{#1}}
%     \def\topbox#1{\raisebox{-\baselineskip}[0in][0in]{#1}}
%     \def\midbox#1{\raisebox{-0.5\baselineskip}[0in][0in]{#1}}
%
%\vspace{3cm}
%
%\title{ 
%%	\logo{
%Matrices and Determinants
%%	}
%}
%
%\author{ G V V Sharma$^{*}$% <-this % stops a space
%	\thanks{*The author is with the Department
%		of Electrical Engineering, Indian Institute of Technology, Hyderabad
%		502285 India e-mail:  gadepall@iith.ac.in. All content in this manual is released under GNU GPL.  Free and open source.}
%	
%}	
%
%\maketitle
%
%%\tableofcontents
%
%\bigskip
%
%\renewcommand{\thefigure}{\theenumi}
%\renewcommand{\thetable}{\theenumi}
%
%
%
%\begin{enumerate}[label=\arabic*]
%\numberwithin{equation}{enumi}
%
\item Let $p\lambda^4+q\lambda^3+r\lambda^2+s\lambda+t$ =
$\begin{vmatrix} \lambda^2+3\lambda & \lambda-1 & \lambda+3  \\ \lambda+1 & -2\lambda & \lambda-4 \\ \lambda-3 & \lambda+4 & 3\lambda \end{vmatrix}$
be an identity in $\lambda$ where p,q,r,s and t are constants. Then, the value of t is...............
\item The solution set of the equation $\begin{vmatrix} 1 & 4 & 20  \\ 1 & -2 & 5 \\ 1 & 2x & 5x^2 \end{vmatrix}= 0$ is.................
\item A determinant is chosen at random from the set of all determinants of order 2 with elements 0 or 1 only. The probability that the value of determinant chosen is positive is................
\item Given that x=-9 is a root of $\begin{vmatrix} x & 3 & 7  \\ 2 & x & 2 \\ 7 & 6 & x  \end{vmatrix}=0$ the other two roots are........... and ...............
\item The system of equations  
\begin{align} 
\lambda x+y+z=0
\end{align}    
\begin{align}
-x+\lambda y+z=0
\end{align}  
\begin{align}
-x-y+\lambda z=0
\end{align} will have a non-zero solution if real values of 
$\lambda$ are given by............
\item The value of the determinant of 
$\begin{vmatrix} 1 & a & a^2-bc  \\ 1 & b & b^2-ca \\ 1 & c & c^2-ab \end{vmatrix}$ is..............
\item For positive numbers x,y and z, the numerical value of the determinant $\begin{vmatrix} 1 & log_x y & log_x z  \\ log_y x & 1 & log_y z \\ log_z x & log_z y & 1 \end{vmatrix}$ is............
\item The determinants $\begin{vmatrix} 1 & a & bc  \\ 1 & b & ca \\ 1 & c & ab \end{vmatrix}$ and $\begin{vmatrix} 1 & a & a^2  \\ 1 & b & b^2 \\ 1 & c & c^2 \end{vmatrix}$ are not identically equal.
\item If $\begin{vmatrix} x_1 & y_1 & 1  \\ x_2 & y_2 & 1 \\ x_3 & y_3 & 1 \end{vmatrix}$ = $\begin{vmatrix} a_1 & b_1 & 1  \\ a_2 & b_2 & 1 \\ a_3 & b_3 & 1 \end{vmatrix}$ then the two triangles with vertices $(x_1, y_1)$, $(x_2, y_2)$, $(x_3, y_3)$ and $(a_1, b_1)$, $(a_2, b_2)$, $(a_3, b_3)$ must be congruent.
\item Consider the set of A of all determinants of order 3 with entries 0 and 1 only. Let B be the subset of A consisting of all determinants with value 1. Let C be the subset of A consisting of all determinants with value -1. Then 
\begin{enumerate}
 \item C is empty
 \item B has as many elements as C
 \item A = $B\cup C$ 
 \item B has twice as many elements as elements as C
 \end{enumerate}
 \item If $\omega(\neq1)$ is a cube root of unity, then $\begin{vmatrix} 1 & 1+i+\omega^2 & \omega^2  \\ 1-i & -1 & \omega^2-1 \\ -i & -i+\omega -1 & -1 \end{vmatrix}$=
 \begin{enumerate}
 \item 0
 \item 1
 \item i
 \item $\omega$
 \end{enumerate}
 \item Let a.b,c be the real numbers. Then the following system of equations in x,y and z
 \begin{align} \frac{x^2}{a^2} +\frac{y^2}{b^2}-\frac{z^2}{c^2} = 1\end{align} 
 \begin{align} \frac{x^2}{a^2}-\frac{y^2}{b^2}+\frac{z^2}{c^2} = 1\end{align}  
 \begin{align} \frac{-x^2}{a^2}+\frac{y^2}{b^2}+\frac{z^2}{c^2} = 1\end{align} has
 \begin{enumerate}
 \item no solution
 \item unique solution
 \item infinitely many solutions 
 \item finitely many solutions
 \end{enumerate}
\item If A and B are square matrices of equal degree, then which one is correct among the following?
\begin{enumerate}
 \item A + B= B + A
 \item A + B= A - B
 \item A - B= B - A
 \item AB= BA
 \end{enumerate}
 \item The parameter on which the value of determinant $\begin{vmatrix} 1 & a & a^2  \\ \cos(p-d)x & \cos(px) & \cos(p+d)x \\ \sin(p-d)x & -\sin(px) & \sin(p+d)x \end{vmatrix}$ does not depend upon  is
 \begin{enumerate}
 \item a
 \item p
 \item d
 \item x
 \end{enumerate}
 \item If f(x)= $\begin{vmatrix} 1 & x & x+1  \\ 2x & x(x-1) & (x+1)x \\ 3x(x-1) & x(x-1)(x-2) & x(x-1)(x+1) \end{vmatrix}$ then f(100) is equal to
 \begin{enumerate}
 \item 0
 \item 1
 \item 100
 \item -100
 \end{enumerate}
 \item If the equations 
 \begin{align} x-ky-z=0 
 \end{align}  
 \begin{align} kx-y-z=0 
 \end{align} 
 \begin{align} x+y-z=0 
 \end{align} has a non-zero solution, then the possible values of k are
 \begin{enumerate}
 \item -1,2
 \item 1,2
 \item 0,1
 \item -1,1
 \end{enumerate}
 \item Let $\omega=\frac{-1}{2}+i\frac{\sqrt3}{2}$. Then the value of the determinant $\begin{vmatrix} 1 & 1 & 1  \\ 1 & -1-\omega^2 & \omega^2 \\ 1 & \omega^2 & \omega^4 \end{vmatrix}$ is
 \begin{enumerate}
 \item $3\omega$
 \item $3\omega(\omega-1)$
 \item $3\omega^2$
 \item $3\omega(1-\omega)$
 \end{enumerate}
 \item The number of values of k for which the system of equations 
 \begin{align} (k+1)x+8y=4k 
 \end{align} 
 \begin{align} kx+(k+3)y=3k-1
 \end{align} has infinitely many solutions is
\begin{enumerate}
 \item 0
 \item 1
 \item 2
 \item infinite
 \end{enumerate}
 \item If A= $\begin{pmatrix} \alpha & 0  \\ 1 & 1 \end{pmatrix}$ and B=$\begin{pmatrix} 1 & 0  \\ 5 & 1 \end{pmatrix}$ then the value of $\alpha$ for which $A^2=B$, is
 \begin{enumerate}
 \item 1
 \item -1
 \item 4
 \item non real values
 \end{enumerate}
 \item If the system of equations 
 \begin{align} x+ay=0
 \end{align}  
 \begin{align} az+y=0
 \end{align} 
 \begin{align} ax+z=0
 \end{align} has infinite solutions, then the value of a is 
 \begin{enumerate}
 \item -1
 \item 1
 \item 0
 \item non real values
 \end{enumerate}
 \item Given 
 \begin{align} 2x-y+2z=2
 \end{align}  
 \begin{align} x-2y+z=-4
 \end{align}  
 \begin{align} x+y+\lambda z=4
 \end{align} then the value of $\lambda$ such that the given system of equation has no solution, is
 \begin{enumerate}
 \item 3
 \item 1
 \item 0
 \item -3
\end{enumerate}
\item If A= $\begin{pmatrix} \alpha & 2 \\ 2 & \alpha \end{pmatrix}$ and  $\begin{vmatrix} A^3\end{vmatrix}=125$ then the value of $\alpha$ is 
\begin{enumerate}
 \item $\pm1$
 \item $\pm2$
 \item $\pm3$
 \item $\pm5$
\end{enumerate}
\item A= $\begin{bmatrix} 1 & 0 & 0  \\ 0 & 1 & 1 \\ 0 & -2 & 4 \end{bmatrix}$  and I= $\begin{bmatrix} 1 & 0 & 0  \\ 0 & 1 & 0 \\ 0 & 0 & 1 \end{bmatrix}$ $A^{-1}$= $\begin{bmatrix} \frac{1}{6}(A^2+cA+dI) \end{bmatrix}$, then the value of c and d are
\begin{enumerate}
 \item (-6,-11)
 \item (6,11)
 \item (-6,11)
 \item (6,-11)
\end{enumerate}
\item If P= $\begin{bmatrix} \frac{\sqrt3}{2} & \frac{1}{2}  \\ \frac{-1}{2} & \frac{\sqrt3}{2} \end{bmatrix}$ and A= $\begin{bmatrix} 1 & 1  \\ 0 & 1 \end{bmatrix}$ and $Q= PAP^T$ and $x=P^TQ^{2005} P$ then x is equal to 
\begin{enumerate}
\item $\begin{bmatrix} 1 & 2005  \\ 0 & 1 \end{bmatrix}$
\item $\begin{bmatrix} 4+2005\sqrt3 & 6015  \\ 2005 &  4-2005\sqrt3 \end{bmatrix}$
\item $\frac{1}{4} \begin{bmatrix} 2+\sqrt3 & 1  \\ -1 &  2-\sqrt3 \end{bmatrix}$
\item $\frac{1}{4} \begin{bmatrix} 2005 & 2-\sqrt3  \\ 2+\sqrt3 &  2005 \end{bmatrix}$
\end{enumerate}
\item Consider three points  
P=$\begin{bmatrix}(-\sin(\beta-\alpha), -\cos\beta)\end{bmatrix}$, 
Q=$\begin{bmatrix} (\cos(\beta-\alpha), -\sin\beta)\end{bmatrix} $ 
R=$\begin{bmatrix} (\cos(\beta-\alpha+\theta), \sin\beta-\theta)\end{bmatrix},$ 
where $0 < \alpha,\beta,\theta < \frac{\pi}{4}$ Then
\begin{enumerate}
 \item P lies on the segment RQ
 \item Q lies on the segment PR
 \item R lies on the segment QP
 \item P,Q,R are non-colinear
\end{enumerate}
\item The number of $3\times3$ matrices A whose entries are either 0 or 1 and for which the system A$\begin{bmatrix} x \\ y \\  z \end{bmatrix}$ = $\begin{bmatrix} 1 \\ 0 \\  0 \end{bmatrix}$ has exactly two distinct solutions is
\begin{enumerate}
 \item 0
 \item $2^9-1$
 \item 168
 \item 2
\end{enumerate}
\item Let $\omega\neq1$ be a cube root of unity and S be the set of all non-singular matrices of the form  $\begin{bmatrix} 1 & a & b \\ \omega & 1 & c\\  \omega^2 & \omega & 1 \end{bmatrix}$ where each of a,b and c is either $\omega$ or $\omega^2$. Then the number of distinct matrices in the set S is
\begin{enumerate}
 \item 2
 \item 6
 \item 4
 \item 8
\end{enumerate}
\item Let P=$[a_{ij}]$ be a $3 \times 3$ matrix and let Q= $[b_{ij}]$, where $b_{ij}=2^{i+j}a_{ij}$  $1\leq i,j\leq3$. If the determinant of P is 2, then the determinant of the matrix Q is 
\begin{enumerate}
 \item $2^{10}$
 \item $2^{11}$
 \item $2^{12}$
 \item $2^{13}$
\end{enumerate}
\item If P is $3\times3$ matrix such that $P^T=2P+I$, where$P^T$ is the transpose of P and I is the $3\times3$ identity matrix, then there exists a column matrix X=$\begin{bmatrix} x \\ y \\ z \end{bmatrix}$ $\neq$ $\begin{bmatrix} 0 \\ 0 \\ 0 \end{bmatrix}$ such that 
\begin{enumerate}
 \item $PX=\begin{bmatrix} 0 \\ 0 \\ 0 \end{bmatrix}$
 \item $PX=X$
 \item $PX=2X$
 \item $PX=-X$
\end{enumerate}
\item P=$\begin{bmatrix} 1 & 0 & 0  \\ 4 & 1 & 0  \\ 16 & 4 & 1  \end{bmatrix}$ and I be a identity matrix of order 3. If $Q=[q_{ij}]$ is a matrix such that $P^{50}-Q = I$, then $\frac{q_{31}+q_{23}}{q_{21}}$ equals
\begin{enumerate}
 \item 52
 \item 103
 \item 201
 \item 205
\end{enumerate}
\item How many $3\times3$ matrices M with entries from {0,1,2} are there, for which the sum of diagonal entries of $M^TM$ is 5?
\begin{enumerate}
 \item 126
 \item 198
 \item 162
 \item 135
\end{enumerate}
\item Let M=$\begin{bmatrix} \sin^4\theta & -1-\sin^2\theta \\ 1+cos^2\theta & \cos^4\theta \end{bmatrix}$=$\alpha I$+$\beta M^{-1}$ Where $\alpha=\alpha(\theta)$ and $\beta=\beta(\theta)$ are real numbers and I is the $2\times2$ identity matrix. If $a^*$ is the minimum of the ${\alpha(\theta): \in [0,2,\pi)}$ and $\beta^*$ is the minimum of the set ${\beta(\theta): \in [0,2\pi)}$. Then the value of $a^*+b^*$ is
\begin{enumerate}
 \item $\frac{-31}{16}$
 \item $\frac{-17}{16}$
 \item $\frac{-37}{16}$
 \item $\frac{-29}{16}$
\end{enumerate}
\item The determinant $\begin{vmatrix} a & b & a\alpha+b  \\ b & c & b\alpha+c \\ a\alpha+b & b\alpha+c & 0 \end{vmatrix}$ is equal to zero, if
\begin{enumerate}
 \item a,b,c are in A.P.
 \item a,b,c are in G.P.
 \item a,b,c are in H.P.
 \item$\alpha$ is a root of the equation $ax^2+bx+c=0$
 \item $(x-\alpha)$ is a factor of $ax^2+2bx+c=0$
\end{enumerate}
\item If $\begin{vmatrix} 6i &-3i & 1  \\ 4 & 3i& -1 \\ 20 &3 & i \end{vmatrix}$=x+iy, then
\begin{enumerate}
 \item a=3, y=1
 \item a=1, y=3
 \item a=0, y=3
 \item a=0, y=0
\end{enumerate}
\item Let M and N be two $3\times3$ non singular skew-symmetric matrices such that MN=NM. if $P^T$ denotes transpose of P, then $M^2N^2(MN^{-1})^T$ is equal to
\begin{enumerate}
 \item $M^2$
 \item $-N^2$
 \item $-M^2$
 \item  MN
\end{enumerate}
\item If adjoint of a $3\times3$ matrix P is $\begin{bmatrix} 1 & 4 & 4  \\ 2 & 1 & 7 \\ 1 & 1 & 3 \end{bmatrix}$ then the possible values of the determinant of P is/are
\begin{enumerate}
 \item -2
 \item -1
 \item  1
 \item  2
\end{enumerate}
\item For $3\times3$ matrices M and N, which of the following statement is(are) NOT correct?
\begin{enumerate}
 \item $N^TMN$ is symmetric or skew symmetric according as M is symmetric or skew symmetric
 \item MN-NM is skew symmetric for all symmetric matrices M and N
 \item MN is a symmetric fro all symmetric matrices M and N
 \item (adjM)(adjN)=(adjMN) for all $invertible$ matrices M and N
\end{enumerate}
\item Let $\omega$ be complex cube root of unity with $\omega\neq 1$ and $P=[p_{ij}]$ be an $n\times n$ matrix with $p_{ij}=\omega^{i+j}$, Then $p^2\neq0$, when n =
\begin{enumerate}
 \item 57
 \item 55
 \item 58
 \item  56
\end{enumerate}
\item Let M be a $2\times2$ symmetric matrix with integer entries. Then M is invertible if
\begin{enumerate}
 \item The first column of M is the transpose of the second row of M
 \item The second row of M is the transpose of the first column of M
 \item M is a diagonal matrix with non-zero entries in the main diagonal
 \item The product of entries in the main diagonal of M is not the square of an integer. \end{enumerate}
\item Let M and N be two $3\times 3$ matrices such that MN=NM. Further,if $M\neq N^2$ and  $M^2=N^4$, then
\begin{enumerate}
 \item determinant of $(M^2+MN^2)$ is 0
 \item there is $3\times3$ non-zero matrix U such that  $(M^2+MN^2)U$ is the zero matrix.
 \item determinant of  $(M^2+MN^2)\geq1$
 \item for $3\times3$ matrix U, $(M^2+MN^2)U$ equals the zero matrix then U is the zero matrix. 
 \end{enumerate}
 \item Which of the following values of $\alpha$ satisfies the equation $\begin{vmatrix} (1+\alpha)^2 & (1+2\alpha)^2 &(1+3\alpha)^2  \\ (2+\alpha)^2 & (2+2\alpha)^2 & (2+3\alpha)^2 \\(3+\alpha)^2 & (3+2\alpha)^2 & (3+3\alpha)^2 \end{vmatrix}$ = -648$\alpha$?
 \begin{enumerate}
 \item -4
 \item  9
 \item -9
 \item  4
\end{enumerate}
\item Let X and Y be two arbitrary $3\times3$ non-zero, skew symmetric matrices and Z be an arbitrary $3\times3$ non-zero symmetric matrix. Then which of the following matrices is(are) skew symmetric?
\begin{enumerate}
 \item $Y^4Z^4-Z^4Y^3$
 \item $X^{44}+Y^{44}$
 \item $X^4Z^3-Z^3X^4$ 
 \item $X^{23}+Y^{23}$
 \end{enumerate}
\item Let P= $\begin{pmatrix} 3 & -1 & -2  \\ 2 & 0 & \alpha \\ 3 & -5 & 0\end{pmatrix}$, where $\alpha\in R$. Suppose $Q=[q_{ij}]$ is a matrix such that PQ=kI, where $k\in R$, $k\neq0$ and I is the identity matrix of order 3. If $q_{23}$ =-k/8 and det(Q) = $k^2/2$, then, 
\begin{enumerate}
 \item $a=0, k=8$
 \item $4a-k+8=0$
 \item $det(Padj(Q)) = 2^9$ 
 \item $det(Qadj(P))$=$2^{13}$
\end{enumerate}
\item Let a,$\lambda$,$\mu$,$\in R$. Consider the system of linear equations 
\begin{align}
ax+2y=\lambda 
\end{align}   
\begin{align} 
3x-2y=\mu 
\end{align}
which of the following statement is correct?
\begin{enumerate}
 \item If a =-3, then the system of has infinitely many solutions for all values of $\lambda$ and $\mu$
 \item If $a\neq-3$, then the system of has unique  solution for all values of $\lambda$ and $\mu$
 \item If $\lambda$ + $\mu$ = 0, then the system has infinitely many solutions for a=-3
 \item If $\lambda + \mu\neq0$, then the system has infinitely many solutions for a=-3
\end{enumerate}
\item Which of the following is(are) the not the square of $3\times3$ matrix with real numbers?
\begin{enumerate}
 \item $\begin{bmatrix} 1 & 0 & 0  \\ 0 & 1 & 0 \\0 & 0 & 1 \end{bmatrix}$
 \item $\begin{bmatrix} 1 & 0 & 0  \\ 0 & 1 & 0 \\0 & 0 & -1 \end{bmatrix}$
 \item $\begin{bmatrix} 1 & 0 & 0  \\ 0 & -1 & 0 \\0 & 0 & -1 \end{bmatrix} $
 \item $\begin{bmatrix} -1 & 0 & 0  \\ 0 & -1 & 0 \\0 & 0 & -1 \end{bmatrix}$
 \end{enumerate}
 \item Let S be the set of all column matrices $\begin{bmatrix} b_1  \\ b_2 \\b_3 \end{bmatrix}$ such that $b_1$, $b_2$, $b_3$,$\in$ R and the system of equations(in real variables) 
 \begin{align} 
 -x+2y+5z=b_1
 \end{align}   
 \begin{align} 
 2x-4y+3z= b_2
 \end{align} 
 \begin{align} 
 x-2y+2z= b_3
 \end{align}  has at least one solution. Then, which of the following system has at least one solution for each $\begin{bmatrix} b_1  \\ b_2 \\b_3 \end{bmatrix}$ $\in$ S?
 \begin{enumerate}
 \item  $x+2y+3z=b_1$  $4y+5z=b_2$  $x+2y+6z=b_3$ 
 \item  $x+y+3z=b_1$  $5x+2y+6z=b_2$  $-2x-y-3z=b_3$ 
 \item  $-x+2y-5z=b_1$ $2x-4y+10z=b_2$  $x-2y+5z=b_3$  
 \item  $x+2y+5z=b_1$ $2x+3z=b_2$  $x+4y-5z=b_3$ 
 \end{enumerate}
 \item Let M=$\begin{bmatrix} 0 & 1 & a  \\ 1 & 2 &3 \\3 & b & 1 \end{bmatrix}$, (adjM)=$\begin{bmatrix} -1 & 1 & -1  \\ 8 & -6 & 2 \\-5 & 3 & -1 \end{bmatrix}$ where a and b are real numbers. Which of the following options is(are) correct?
\begin{enumerate}
 \item a+b=3
 \item $det(adj(M^2)) = 81$
 \item $(adjM)^{-1} + adjM^{-1}$
 \item if M$\begin{bmatrix} \alpha  \\ \beta \\\gamma \end{bmatrix}$=$\begin{bmatrix} 1  \\ 2 \\3 \end{bmatrix}$, then $\alpha-\beta+\gamma=3$
\end{enumerate}
\item Let $x\in R$ and let P= $\begin{bmatrix} 1 & 1 & 1  \\ 0 & 2 &2 \\0 & 0 & 3 \end{bmatrix}$, Q=$\begin{bmatrix} 2 & x & x  \\ 0 & 4 & 0 \\x & x & 6 \end{bmatrix}$ and $R=PQP^{-1}$. Then which of the following option is(are) correct?
\begin{enumerate}
\item det R=det$\begin{bmatrix} 2 & x & x  \\ 0 & 4 & 0 \\x & x & 6 \end{bmatrix}$ and $R=PQP^{-1}$+8 
\item For x=1 there exists a unit vector $\alpha i\hat{}+\beta j\hat{}+\gamma k\hat{}$ for which R$\begin{bmatrix} \alpha \\ \beta \\\gamma \end{bmatrix}$=$\begin{bmatrix} 0  \\  0 \\ 0  \end{bmatrix}$
\item There exists a real number x such that PQ=QP
\item For x=0, if R=$\begin{bmatrix} 1  \\  a \\ b  \end{bmatrix}$=6$\begin{bmatrix} 1  \\  a \\ b  \end{bmatrix}$, then a+b=5.
\end{enumerate}
\item For what value of k do the following system of equations possess a non trivial(i,e.,not all zero) solution over the set of rationals Q ? 
\begin{align} 
x+ky+3z=0 
\end{align},
\begin{align}
3x+ky-2z=0
\end{align},
\begin{align}
2x+3y-4z=0
\end{align}.For that value of k, find all the solutions for the system.
\item Let a, b, c be positive and not all equal.Show that the value of the determinant $\begin{vmatrix}a & b & c \\ b & c & a \\ c & a & b\end{vmatrix}$ is negative.
\item Without expanding a determinant at any stage, show that 
$\begin{vmatrix}x^2+x & x+1 & x-2 \\ 2x^2+3x-1 & 3x & 3x-3 \\ x^2+2x+3 & 2x-1 & 2x-1\end{vmatrix}$ =xA+B, where A and B are determinants of order 3 not involving x.
\item Show that $\begin{vmatrix} x_{C_r} & x_{C_r+1} & x_{C_r+2} \\ y_{C_r} & y_{C_r+1} & y_{C_r+2} \\ z_{C_r} & z_{C_r+1} & z_{C_r+2}\end{vmatrix}$ =  $\begin{vmatrix}x_{C_r} & x+1_{C_r+1} & x+2_{C_r+2} \\ y_{C_r} & y+1_{C_r+1} & y+2_{C_r+2} \\ z_{C_r} & z+1_{C_r+1} & z+2_{C_r+2}\end{vmatrix}$
\item Consider the system of linear equations in x, y, z:
\begin{align} 
(\sin 3\theta) x-y+z=0,
\end{align}
\begin{align} 
(\cos 2\theta) x+4y+3z=0,
\end{align} 
\begin{align}
2x+7y+7z=0 .
\end{align} Fint the values of $\theta$ for which this system has nontrivial solutions.
\item Let $\Delta a$=$\begin{vmatrix} a-1 & n & 6 \\ (a-1)^2 & 2n^2 & 4n-2 \\ (a-1)^3 & 3n^3 & 3n^2 - 3n\end{vmatrix}$. Show that $\sum_{a=1}^{n} \Delta a = c$, a constant.
\item Let the three digit numbers A28, 3B9, and 62C, where A,B and C are integers between 0 and 9, be divisible by a fixed integer K. Show that the determinant $\begin{vmatrix} A & 3 & 6 \\ 8 & 9 & C \\ 2 & B & 2\end{vmatrix}$ is divisible by k.
\item If $a \neq p$, $b \neq q$, $c \neq r$ and $\begin{vmatrix} p & b & c \\ a & q & c \\ a & b & r\end{vmatrix}$=0. Then find the value of $\frac{p}{p-a}$ + $\frac{q}{q-b}$ +  $\frac{r}{r-c}$.
\item For a fixed positive integer n, if D=$\begin{vmatrix} n! & (n+1)! & (n+2)! \\ (n+1)! & (n+2)! & (n+3)! \\ (n+2)! & (n+3)! & (n+4)!\end{vmatrix}$ then show that $[\frac{D}{(n!)^3} -4]$ is divisible by n.
\item Let $\lambda$ and $\alpha$ be real. Find the set of all values of $\lambda$ for which the system of linear equations \\
$\lambda$x+(sin$\alpha$)y+(cos$\alpha$)z=0,\\
 x+(cos$\alpha$)y+(sin$\alpha$)z=0,\\
  -x+(sin$\alpha$)y-(cos$\alpha$)z=0\\
   has a non-trivial solution. For $\lambda$=1, find all values of $\alpha$.
\item For all values of A,B,C and P,Q,R show that $\begin{vmatrix} cos(A-P) & cos(A-Q) & cos(A-R) \\ cos(B-P) & cos(B-Q) & cos(B-R) \\ cos(C-P) & cos(C-Q) & cos(C-R)\end{vmatrix}$=0.
\item Let $a>0$, $d>0$. Find the value of the determinant\\
 $\begin{vmatrix} \frac{1}{a} & \frac{1}{a(a+d)} & \frac{1}{(a+d)(a+2d)} \\ \frac{1}{(a+d)} & \frac{1}{(a+d)(a+2d)} & \frac{1}{(a+2d)(a+3d)} \\ \frac{1}{a(a+2d)} & \frac{1}{(a+2d)(a+3d)} & \frac{1}{(a+3d)(a+4d)}\end{vmatrix}$\\
\item Prove that for all values of $\theta$\\.  
$\begin{vmatrix} sin\theta & cos\theta  & sin2\theta \\ sin(\theta+\frac{2\pi}{3}) & cos(\theta+\frac{2\pi}{3}) & sin(2\theta+\frac{4\pi}{3}) \\ sin(\theta-\frac{2\pi}{3}) & cos(\theta-\frac{2\pi}{3}) & sin(2\theta-\frac{4\pi}{3})\end{vmatrix}$=0 \\
\item If matrix A=$\begin{bmatrix} a & b & c \\ b & c & a \\ c & a & b\end{bmatrix}$ where a,b,c are real positive numbers, abc=1, and $A^TA=I$, then find the value of $a^3+b^3+c^3$.
\item If M is a 3x3 matrix, where det M=1 and  $MM^T=I$, where 'I' is an identity matrix, prove that det (M-I)=0.
\item If A=$\begin{bmatrix} a & 1 & 0 \\ 1 & b & d \\ 1 & b & c\end{bmatrix}$, B=$\begin{bmatrix} a & 1 & 1 \\ 0 & d & c \\ f & g & h\end{bmatrix}$, U=$\begin{bmatrix} f \\ g \\ h\end{bmatrix}$, V=$\begin{bmatrix} a^2 \\ 0 \\ 0\end{bmatrix}$, X=$\begin{bmatrix} x \\ y \\ z\end{bmatrix}$ and AX=U has infinitely many solutions, prove that BX=V has no unique solution. Also show that if $afd \neq 0$ then BX=V has no solution.
Let A =$\begin{bmatrix}
1& 0& 0 \\ 2&1&0 \\3&2&1 
\end{bmatrix}$ and $U_1,U_2$ and $U_3$ are columns of a 3$\times$3 matrix U. If column matrices $U_1,U_2$ and $U_3$ satisfiying $AU_1= \begin{bmatrix}
1 \\0 \\0
\end{bmatrix}$ $AU_2= \begin{bmatrix}
2 \\3 \\0
\end{bmatrix}$ $AU_3= \begin{bmatrix}
2 \\3 \\1
\end{bmatrix}$ evaluate as directed in the following questions.
\item The value $\begin{vmatrix} U \end{vmatrix}$ is 
\begin{enumerate}
\item 3
\item -3
\item $\frac{3}{2}$
\item 2
\end{enumerate}
\item The sum of the elements of the matrics $U^{-1}$ is 
\begin{enumerate}
\item -1
\item 0
\item 1
\item 3
\end{enumerate}
\item The value of $\begin{bmatrix}
3&2&0 
\end{bmatrix}$U $\begin{bmatrix}
3 \\ 2 \\0
\end{bmatrix}$ is 
\begin{enumerate}
\item 5
\item $\frac{5}{2}$
\item 4
\item $\frac{3}{2}$
\end{enumerate}

Let P be the set of all $3\times3$ symmetric matrices all of whose entries are either 0 or 1. Five of these entries are 1 and four of them are 0.
\item The number of matrices in P is
\begin{enumerate}
 \item 12
 \item 6
 \item 9
 \item 3
\end{enumerate}
\item The number of matrices A in P for which the system of linear equations A$\begin{bmatrix} x  \\  y \\ z  \end{bmatrix}$=$\begin{bmatrix} 1  \\  0 \\ 0   \end{bmatrix}$ has a unique solution is 
\begin{enumerate}
 \item 0
 \item more than 2
 \item 2
 \item 1
\end{enumerate}
Let p be an odd prime number and $T_{P}$ be the following set of $2\times2$ matrices:
$T_{P}$ =  A$\begin{bmatrix} a & b \\ c & a \end{bmatrix}$: a,b,c $\in$ $\begin{bmatrix}0,1,2,....,{p-1}\end{bmatrix}$
\item The number of A in $T_p$ such that A is either symmetric or skew-symmetric or both and det(A) divisible by p is 
\begin{enumerate}
 \item $(p-1)^2$
 \item 2(p-1)
 \item $(p-1)^2$+1
 \item 2(p-1)
\end{enumerate}
\item The number of A $T_p$ such that the trace of A is not divisible by p but det(A) is divisible by p is
\begin{enumerate}
 \item $(p-1)(p^2-p+1)$
 \item $p^3-(p-1)^2$
 \item $(p-1)^2$
 \item $(p-1)(p^2-2)$
\end{enumerate}
\item The number of A in $T_p$ such that det(A) is not divisible by p is
\begin{enumerate}
 \item $2p^2$
 \item $p^3-5p$
 \item $p^3-3p$
 \item $p^3-p^2$
\end{enumerate}
Let a,b and c be three real numbers satisfying [abc]$\begin{bmatrix} 1 & 9 & 7 \\ 8 & 2 & 7 \\ 7 & 3 & 7 \end{bmatrix}$=[000]
\item If the point P(a,b,c) with reference to (E),lies on the plane 
\begin{align} 
2x+y+z=1
\end{align} then the value of 
\begin{align} 
7a+b+c
\end{align} is
\begin{enumerate}
 \item 0
 \item 12
 \item 7
 \item 6
\end{enumerate}
\item Let $\omega$ be a solution of $x^3-1=0$ with $Im(\omega)>0$, if a=2 with b and c satisfying(E),then the value of $\frac{3}{\omega^a}+\frac{1}{\omega^b}+\frac{3}{\omega^c}$ is equal to 
\begin{enumerate}
 \item -2
 \item  2
 \item 3
 \item -3
\end{enumerate}
\item Let b=6, with a and c satisfying(E). If $\alpha$ and $\beta$ are the roots of the equation of  
\begin{align}
ax^2+bx+c=0
\end{align} then $\sum_{n = 0}^\infty (\frac{1}{\alpha}+\frac{1}{\beta})^n$ is 
\begin{enumerate}
 \item 6
 \item 7
 \item $\frac{6}{7}$
 \item $\infty$
\end{enumerate}
\item Consider the system of equations 
\begin{align}
 x-2y+3z=-1
 \end{align}  
 \begin{align}
  -x+y-2z=k
  \end{align}  
  \begin{align}
   x-3y+4z=1
   \end{align}
\textbf {STATEMENT - 1:} The system of equations no solution for $k\neq3$ and\\
\textbf {STATEMENT - 2:} The determinant \\
$\begin{vmatrix} 1 & 3 & -1 \\ -1 & -2 & k \\ 1 & 4 & 1 \end{vmatrix}$$\neq0$, for k$\neq3$
\begin{enumerate}
 \item STATEMENT-1 is True,STATEMENT-2 is True; \\
       STATEMENT-2 is a correct explanation for STATEMENT-1
 \item STATEMENT-1 is True,STATEMENT-2 is True; \\
       STATEMENT-2 is NOT a correct explanation for STATEMENT-1
 \item STATEMENT-1 is True,STATEMENT-2 is False
 \item STATEMENT-1 is False,STATEMENT-2 is True
\end{enumerate}
\item Let $\omega$ be the complex number $\cos\frac{2\pi}{3}+i\sin\frac{2\pi}{3}$. Then the number of distinct complex numbers z satisfying \\
$\begin{vmatrix} z+1 & \omega & \omega^2 \\ \omega &z+\omega^2  & 1 \\ \omega^2 &  1 & z+\omega \end{vmatrix}=0$ is equal to
\item Let k be a positive real number and let A=\\
$\begin{vmatrix} 2k-1 & 2\sqrt k & 2\sqrt k \\ 2\sqrt k & 1  & -2k \\ -2\sqrt k &  2k & -1\end{vmatrix}$
 B=$\begin{vmatrix} 0 & 2k-1 & \sqrt k \\ 1-2k & 0  & 2\sqrt k \\ -\sqrt k &  -2\sqrt k & 0\end{vmatrix}$\\
 If $det(adj A ) + det(adj B )=10^6$. then [k] is equal to 
\item Let M be $3\times3$ matrix satisfying M $\begin{bmatrix} 0 \\ 1\\ 0 \end{bmatrix}$=$\begin{bmatrix} -1 \\ 2 \\ 3 \end{bmatrix}$, M$\begin{bmatrix} 1 \\ -1\\ 0 \end{bmatrix}$=$\begin{bmatrix} 1 \\ 1 \\ -1 \end{bmatrix}$ and M$\begin{bmatrix} 1 \\ 1\\ 1 \end{bmatrix}$=$\begin{bmatrix} 0 \\ 0 \\ 12 \end{bmatrix}$.Then the sum of diagonal entries of M is
\item The total number of distinct $x\in R$ for which $\begin{bmatrix} x & x^2 & 1+x^3 \\ 2x & 4x^2 & 1+8x^3 \\ 3 & 9x^2 & 1+27x^3\end{bmatrix} = 10$ is 
\item Let $z=\frac{-1+\sqrt 3i}{2}$, where $i=\sqrt -1$ and r,s$\in$(1,2,3) Let P=$\begin{bmatrix} (-z)^r & z^{2s} \\ z^{2s} & z^r\end{bmatrix}$ and I be the identity matrix of order 2. Then the total number of ordered pairs (r,s) for which $P^2=-I$ is
\item For real number $\alpha$, if the system $\begin{bmatrix} 1 & \alpha & \alpha^2 \\ \alpha & 1 & \alpha \\ \alpha^2 &\alpha & 1\end{bmatrix}$ $\begin{bmatrix} x \\y\\z \end{bmatrix}$=
$\begin{bmatrix} 1 \\ -1 \\ 1 \end{bmatrix}$ 
of linear equations has infinitely many solutions, then 1+$\alpha+\alpha^2$=
\item Let P be a matrix of order $3\times 3$ such that all the entries in P are from the set (-1, 0, 1). Then the maximum possible value of the determinant of P is.......
\item If $a>0$ and discriminant of 
\begin{align} 
ax^2+2bx+c 
\end{align} is -ve then 
$\begin{vmatrix} a & b & ax+b \\ b & c  & bx+c \\ ax+b & bx+c & 0\end{vmatrix}$ is equal to
\begin{enumerate}
 \item +ve
 \item $(ac-b^2)(ax^2+2bx+c)$
 \item -ve
 \item 0
\end{enumerate}
\item If the system of linear equations \begin{align} x+2ay+az=0\end{align}  \begin{align} x+3by+bz=0\end{align}  \begin{align} x+4cy+cz=0\end{align} has a non zero solution then a, b, c.
\begin{enumerate}
 \item satisfy a+2b+3c=0
 \item are in A.P.
 \item are in G.P.
 \item are in H.P.
\end{enumerate}
\item If 1, $\omega$, $\omega^2$ are the cube roots of unity, then $\Delta$=$\begin{vmatrix}  1 & \omega^n &\omega^{2n} \\ \omega^n & \omega^{2n}  & 1 \\ \omega^{2n} & 1 & \omega^n \end{vmatrix}$ is equal to
\begin{enumerate}
 \item $\omega^2$
 \item 0
 \item 1
 \item $\infty$
\end{enumerate}
\item If A=$\begin{bmatrix} a & b \\ b & a \end{bmatrix}$ and $A^2$=$\begin{bmatrix} \alpha & \beta \\ \beta & \alpha \end{bmatrix}$, then 
\begin{enumerate}
 \item $\alpha=2ab,\beta=a^2+b^2$
 \item $\alpha=a^2+b^2,\beta=ab$
 \item $\alpha=a^2+b^2,\beta=2ab$
 \item $\alpha=a^2+b^2,\beta=a^2-b^2$
\end{enumerate}
\item Let A=$\begin{bmatrix} 0 & 0 & -1 \\ 0 & -1 & 0 \\ -1 & 0 & 0 \end{bmatrix}$. The only correct statement about the same matrix A is
\begin{enumerate}
 \item $A^2=I$
 \item A=(-1)I,where I is a unit matrix
 \item $A^{-1}$ does not exist
 \item A is zero matrix
\end{enumerate}
\item Let A=$\begin{bmatrix} 1 & -1 & 1 \\ 2 & 1 & -3 \\ 1 & 1 & 1 \end{bmatrix}$ and 10B=$\begin{bmatrix} 4 & 2 & 2 \\ -5 & 0 & \alpha \\ 1 & -2 & 3 \end{bmatrix}$. If B is inverse matrix of A, then $\alpha$ is
\begin{enumerate}
 \item 5
 \item -1
 \item 2
 \item -2
\end{enumerate}
\item If $a_1,a_2,a_3........,a_n....$are in G.P., then the value of determinant  $\begin{bmatrix} log a_n & log a_{n+1} & log a_{n+2} \\ log a_{n+3} & log a_{n+4} & log a_{n+5} \\ log a_{n+6} & log a_{n+7} & log a_{n+8} \end{bmatrix}$ is
\begin{enumerate}
 \item -2
 \item  1
 \item  2
 \item  0
\end{enumerate}
\item If $A^2-A+I=0$, then the inverse of A is
\begin{enumerate}
 \item A+I
 \item A
 \item A-I
 \item I-A
\end{enumerate}
\item The system of equations \begin{align} \alpha x + y + z=\alpha-1\end{align}  \begin{align} x +\alpha y  +z=\alpha-1 \end{align}  \begin{align} x +y +\alpha z=\alpha-1 \end{align} has infinite solutions, if $\alpha$ is 
\begin{enumerate}
 \item -2
 \item either -2 or 1
 \item not -2
 \item 1
\end{enumerate}
\item If $a^2+b^2+c^2=-2$ and \\
f(x)=$\begin{bmatrix} 1+a^2x & (1+b^2)x & (1+c^2)x \\ (1+a^2)x & 1+b^2x & (1+c^2)x \\ (1+a^2)x & (1+b^2)x & 1+c^2x \end{bmatrix}$, then f(x) is a polynomial of degree 
\begin{enumerate}
 \item 1
 \item 0
 \item 3
 \item 2
\end{enumerate}
\item  If $a_1,a_2,a_3,.....a_n$,....are in G.P., then the determinant $\Delta$=$\begin{bmatrix} log a_n & log a_{n+1} & log a_{n+2} \\ log a_{n+3} & log a_{n+4} & log a_{n+5} \\ log a_{n+6} & log a_{n+7} & log a_{n+8} \end{bmatrix}$ is equal to
\begin{enumerate}
 \item 1
 \item 0
 \item 4
 \item 2
\end{enumerate}
\item If A and B are two square matrices of size n$\times$n such that $A^2-B^2=(A-B)(A+B)$, then which of the following will be always true?
\begin{enumerate}
 \item A=B
 \item AB=BA
 \item either A or B is zero matrix
 \item either A or B is identity matrix
\end{enumerate}
\item Let A=$\begin{bmatrix} 1 & 2 \\ 3 & 4 \end{bmatrix}$ and B=$\begin{bmatrix} a & 0 \\ 0 & b \end{bmatrix}$ a, b$\in$N. Then
\begin{enumerate}
 \item there cannot exist any B such that AB=BA
 \item there exists more then one but finite number of B's such that AB=BA
 \item there exists exactly one B such that AB=BA
 \item there exists infinitely many B's such that AB=BA 
\end{enumerate}
\item If D = $\begin{bmatrix} 1 & 1 & 1 \\ 1 & 1+x & 1 \\ 1 & 1 & 1+y \end{bmatrix}$
 for $x \neq 0$, $y \neq 0$ then D is 
\begin{enumerate}
 \item divisible by x but not y
 \item divisible by y but not x
 \item divisible by neither x nor y
 \item divisible by both x and y
\end{enumerate}
\item Let A=$\begin{bmatrix} 5 & 5\alpha & \alpha \\ 0 & \alpha & 5\alpha \\ 0 & 0 & 5 \end{bmatrix}$. If $\begin{vmatrix} A^2 \end{vmatrix}$=25, then $\begin{vmatrix} \alpha \end{vmatrix}$  equals 
\begin{enumerate}
 \item $\frac{1}{5}$
 \item 5
 \item $5^2$
 \item 1
\end{enumerate}
\item Let A be a $2\times 2$ matrix with real entries. Let I be the $2\times2$ identity matrix. Denote by tr(A), the sum of diagonal entries of a. Assume that $A^2$=I.\\
\textbf {STATEMENT-1:} If A$\neq$I and A$\neq$-I, then det(A)=-1\\ 
\textbf {STATEMENT-2:} If A$\neq$I and A$\neq$-I, then tr(A)$\neq$-1\\
\begin{enumerate}
 \item Statement-1 is false, Statement-2 is true
 \item Statement-1 is true, Statement-2 is true; Statement-2 is a correct explanation for statement-1
 \item Statement-1 is true, Statement-2 is true; Statement-2 is not a correct explanation for statement-1
 \item Statement-1 is true, Statement-2 is false
\end{enumerate}
\item Let a,b,c be any real numbers. Suppose that there are real numbers x,y,z mot all zero such that \begin{align}
 x = cy + bz
 \end{align} 
 \begin{align}
  y = az + cx 
  \end{align}  
  \begin{align}
   z = bx + ay \end{align} Then $a^2+b^2+c^2+2abc$ is equal to
\begin{enumerate}
 \item 2
 \item -1
 \item 0
 \item 1
\end{enumerate}
\item Let A be a square matrix all of whose entries are integers. Then which one of the following is true?
\begin{enumerate}
 \item If detA = $\pm$1, then $A^{-1}$ exists but all its entries are not neccessarily integers
 \item If detA$ \neq \pm$1, then $A^{-1}$ exists and  all its entries are non  integers
 \item If detA = $\pm$1, then $A^{-1}$ exists but all its entries are integers
 \item If detA = $\pm$1, then $A^{-1}$ need not exists
\end{enumerate}
\item Let A be a $2\times2$ matrix\\
\textbf{STATEMENT-1:} adj(adjA) = A\\
\textbf{STATEMENT-2:} $\begin{vmatrix} adjA \end{vmatrix}$ = $\begin{vmatrix} A \end{vmatrix}$
\begin{enumerate}
 \item Statement-1 is true, Statement-2 is false. Statement-2 is not a correct explanation for Statement-1.
 \item Statement-1 is true, Statement-2 is false
 \item Statement-1 is false, Statement-2 is true
 \item Statement-1 is true, Statement-2 is true. Statement-2 is a correct explanation for Statement-1.
\end{enumerate}
\item Let a,b,c be such that b(a+c)$\neq$ 0 if $\begin{vmatrix} a & a+1 & a-1 \\ -b & b+1 & b-1 \\ c & c-1 & c+1 \end{vmatrix}$ + $\begin{vmatrix} a+1 & b+1 & c-1 \\ a-1 & b-1 & c+1 \\ (-1)^{n+2}a & (-1)^{n+1}b & (-1)^{n}c \end{vmatrix}=0$, then the value of n is
\begin{enumerate}
 \item any even integer
 \item any odd integer
 \item any integer
 \item zero
\end{enumerate}
\item The number of $3\times3$ non singular matrices with four entries as 1 and all other are as 0, is
\begin{enumerate}
 \item 5
 \item 6 
 \item at least 7
 \item less than 4
\end{enumerate}
\item Let A be a 2$\times$2 matrix with non-zero entries and let $A^2$ = I, where I is a 2$\times$2 identity matrix. Define Tr(A)=sum of diagonal elements of A and $\begin{vmatrix} A \end{vmatrix}$ = determinant of matrix A.\\
\textbf{STATEMENT-1:} Tr(A)=0\\
\textbf{STATEMENT-1:} $\begin{vmatrix} A \end{vmatrix}$=1\\
\begin{enumerate}
 \item Statement-1 is true, Statement-2 is true. Statement-2 is not a correct explanation for Statement-1.
 \item Statement-1 is true, Statement-2 is false
 \item Statement-1 is false, Statement-2 is true
 \item Statement-1 is true, Statement-2 is true. Statement-2 is a correct explanation for Statement-1.
\end{enumerate}
\item Consider the system of linear equations 
\begin{align} x_1 + 2x_2 + x_3 = 3 \end{align} 
\begin{align} 2x_1 + 3x_2 + x_3 = 3 \end{align}
\begin{align} 3x_1 + 5x_2 + 2x_3 = 1 \end{align}. The system has 
\begin{enumerate}
 \item exactly 3 solutions
 \item a unique solution
 \item no solution
 \item infinite number of solutions
\end{enumerate}
\item The number of values of k for which the linear equations 
\begin{align} 4x+ky+2z=0  \end{align}  
\begin{align} kx+4y+z=0 \end{align} 
\begin{align} 2x+2y+z=0  \end{align} posses a non-zero solution is
\begin{enumerate}
 \item 2
 \item 1
 \item 0
 \item 3
\end{enumerate}
\item Let A and B be two symmetric matrices of order 3.\\
\textbf{STATEMENT-1 :} A(BA) and (AB)A are symmetric matrices\\
\textbf{STATEMENT-2 :} AB is a symmetric matrix if matrix multiplication of a A and B is cumulative.
\begin{enumerate}
 \item Statement-1 is true, Statement-2 is true. Statement-2 is not a correct explanation for Statement-1.
 \item Statement-1 is true, Statement-2 is false
 \item Statement-1 is false, Statement-2 is true
 \item Statement-1 is true, Statement-2 is true. Statement-2 is a correct explanation for Statement-1.
\end{enumerate}
\item Let A = $\begin{pmatrix} 1 & 0 & 0 \\ 2 & 1 & 0 \\ 3 & 2 & 1 \end{pmatrix}$ If $u_1$ and $u_2$ are column matrices such that $Au_1$ = $\begin{pmatrix} 1 \\ 0 \\ 0\end{pmatrix}$ and $Au_2$ = $\begin{pmatrix} 0 \\ 1 \\ 0\end{pmatrix}$ then $u_1 + u_2$ is equal to
\begin{enumerate}
 \item $\begin{pmatrix} -1 \\ 1 \\ 0\end{pmatrix}$
 \item $\begin{pmatrix} -1 \\ 1 \\ -1\end{pmatrix}$
 \item $\begin{pmatrix} -1 \\ -1 \\ 0\end{pmatrix}$
 \item $\begin{pmatrix}  1 \\ -1 \\ 0\end{pmatrix}$
\end{enumerate}
\item Let P and Q be 3$\times$3 matrices p$\neq$Q. If $P^3$ = $Q^3$ and $P^2$Q = $Q^2$P then the determinant of $(P^2+Q^2)$ is equal to
\begin{enumerate}
 \item -2
 \item 1
 \item 0
 \item -1
\end{enumerate}
\item P = $\begin{pmatrix} 1 & \alpha & 3 \\ 1 & 3 & 3 \\ 2 & 4 & 4 \end{pmatrix}$ is the adjoint of a 3
$\times$3 matrix A and $\begin{vmatrix} A \end{vmatrix}$=4, then $\alpha$ is equal to 
\begin{enumerate}
 \item 4
 \item 11
 \item 5
 \item 0
\end{enumerate}
\item If $\alpha$, $\beta$ $\neq$0, and f(n) = $(\alpha)^n +(\beta)^n$ and $\begin{vmatrix} 3 & 1+f(1) & 1+f(2) \\ 1+f(1) & 1+f(2) & 1+f(3) \\ 1+f(2) & 1+f(3) & 1+f(4) \end{vmatrix}$ = K$(1-\alpha)^2(1-\beta)^2(\alpha-\beta)^2$, then K is equal to
\begin{enumerate}
 \item 1
 \item -1
 \item $\alpha\beta$
 \item $\frac{1}{\alpha\beta}$
\end{enumerate}
\item If A is a 3$\times$3 non-singular matrix such that $AA^1 = A^1A$ and B = $A^{-1}A^1$, then $BB^1$ is equal to 
\begin{enumerate}
 \item $B^{-1}$
 \item $(B^{-1})^1$
 \item I+B
 \item I
\end{enumerate}
\item The set of all values of $\lambda$ for which the system of linear equations:
\begin{align} 
2x_1 - 2x_2 + x_3 = \lambda x_1 
\end{align}  
\begin{align}
 2x_1 - 3x_2 + 2x_3 = \lambda x_2 
 \end{align}
 \begin{align}
  -x_1 + 2x_2 = \lambda x_3 
 \end{align} has a non-trivial solution. 
\begin{enumerate}
 \item contains two elements
 \item contains more than two elements
 \item is an empty set
 \item is a singleton
\end{enumerate}
\item A = $\begin{pmatrix} 1 & 2 & 2 \\ 2 & 1 & -2 \\ a & 2 & b\end{pmatrix}$ is matrix satisfying the equation $AA^T$ = 9I where I is a 3$\times$3 identity matrix, then the ordered pair(a,b) is equal to 
\begin{enumerate}
 \item (2,1)
 \item (-2,-1)
 \item (2,-1)
 \item (-2,1)
\end{enumerate}
\item The system of linear equations 
\begin{align} 
x + \lambda y - z = 0 
\end{align} 
\begin{align} 
\lambda x - y - z = 0 
\end{align} 
\begin{align}
 x +  y - \lambda z = 0 
 \end{align} has a non-trivial solution for:
\begin{enumerate}
 \item exactly two values of $\lambda$
 \item exactly three values of $\lambda$
 \item infinitely many values of $\lambda$
 \item exactly one value of $\lambda$
\end{enumerate}
\item If A = $\begin{bmatrix} 5a & -b \\ 3 & 2 \end{bmatrix}$ and AadjA = $AA^T$, then 5a+b is equal to
\begin{enumerate}
 \item 4
 \item 13
 \item -1
 \item 5
\end{enumerate}
\item Let k be an integer such that triangle with vertices (k,-3k),(5,k) and (-k,2) has area 28sq.units. Then the orthocentre of this triangle is at the point
\begin{enumerate}
 \item (2, $\frac{1}{2}$)
 \item (2, $\frac{-1}{2}$)
 \item (1, $\frac{3}{4}$)
 \item (1, $\frac{-3}{4}$)
\end{enumerate}
\item Let $\omega$ be a comlex number such that 2$\omega$+1 = z where z = $\sqrt{-3}$. 
If $\begin{vmatrix} 1 & 1 & 1 \\ 1 & -\omega^2-1 & \omega^2 \\ 1 & \omega^2 & \omega^7 \end{vmatrix}$ = 3k, then k is equal to
\begin{enumerate}
 \item 1
 \item -z
 \item z
 \item -1
\end{enumerate}
\item If A = $\begin{bmatrix} 2 & -3 \\ -4 & 1  \end{bmatrix}$, then $adj(3A^2+12A)$ is equal to
\begin{enumerate}
 \item $\begin{bmatrix} 72 & -63 \\ -84 & 51 \end{bmatrix}$
 \item $\begin{bmatrix} 72 & -84 \\ -63 & 51 \end{bmatrix}$
 \item $\begin{bmatrix} 51 &  63 \\  84 & 72 \end{bmatrix}$
 \item $\begin{bmatrix} 51 &  84 \\  63 & 72 \end{bmatrix}$
\end{enumerate}
\item If $\begin{vmatrix} x-4 & 2x & 2x \\ 2x & x-4 & 2x \\ 2x & 2x & x-4\end{vmatrix}$ = $(A+Bx)(x-A)^2$ then the ordered pair (A,B) is equal to
\begin{enumerate}
 \item (-4,3)
 \item (-4,5)
 \item (4,5)
 \item (-4,-5)
\end{enumerate}
\item If the system of linear equations 
\begin{align} x+ky+3z=0 \end{align} 
\begin{align} 3x+ky-2z=0 \end{align} 
\begin{align} 2x+4y-3z=0 \end{align} has a non-zero solution (x,y,z), then $\frac{xz}{y^2}$ is equal to
\begin{enumerate}
 \item  10
 \item -30
 \item  30
 \item -10
\end{enumerate}
\item The system of linear equations 
\begin{align} x+y+z=2 \end{align} 
\begin{align} 2x+3y+2z=5 \end{align} 
\begin{align} 2x+3y+(a^2-1)z=a+1 \end{align}
\begin{enumerate}
 \item  is inconsistent when a=4
 \item has a unique solution for $\begin{vmatrix} a \end{vmatrix}$=$\sqrt{3}$
 \item  has a infinitely many solutions for a=4
 \item is inconsistent $\begin{vmatrix} a \end{vmatrix}$=$\sqrt{3}$
\end{enumerate}
\item If A=$\begin{bmatrix} \cos\theta & -\sin\theta \\ \sin\theta & \cos\theta  \end{bmatrix}$, then the matrix $A^{-50}$ when $\theta$=$\frac{\pi}{12}$ is equal to
\begin{enumerate}
 \item $\begin{bmatrix} \frac{1}{2} & -\frac{\sqrt{3}}{2} \\ \frac{\sqrt{3}}{2} & \frac{1}{2}  \end{bmatrix}$
 \item $\begin{bmatrix} \frac{\sqrt{3}}{2} & -\frac{1}{2} \\ \frac{1}{2} & \frac{\sqrt{3}}{2}  \end{bmatrix}$
 \item  $\begin{bmatrix} \frac{\sqrt{3}}{2} & \frac{1}{2} \\ -\frac{1}{2} & \frac{\sqrt{3}}{2}  \end{bmatrix}$
 \item  $\begin{bmatrix} \frac{1}{2} & \frac{\sqrt{3}}{2} \\ -\frac{\sqrt{3}}{2} & \frac{1}{2}  \end{bmatrix}$
\end{enumerate}
\item If \\
$\begin{bmatrix} 1 & 1 \\ 0 & 1 \end{bmatrix}$.$\begin{bmatrix} 1 & 2 \\ 0 & 1 \end{bmatrix}$.$\begin{bmatrix} 1 & 3 \\ 0 & 1 \end{bmatrix}$........$\begin{bmatrix} 1 & n-1 \\ 0 & 1 \end{bmatrix}$
= $\begin{bmatrix} 1 & 78 \\ 0 & 1 \end{bmatrix}$ then the inverse of $\begin{bmatrix} 1 & n \\ 0 & 1 \end{bmatrix}$ is 
\begin{enumerate}
 \item $\begin{bmatrix} 1 & 0 \\ 12 & 1 \end{bmatrix}$
 \item $\begin{bmatrix} 1 & -13 \\ 0 & 1 \end{bmatrix}$
 \item $\begin{bmatrix} 1 & -12 \\ 0 & 1 \end{bmatrix}$
 \item $\begin{bmatrix} 1 & 0 \\ 13 & 1 \end{bmatrix}$
\end{enumerate}
\item Let $\alpha$ and $\beta$ be the roots of the equation \\$x^2+x+1=0$. Then for y $\neq$ 0 in R $\begin{bmatrix} y+1 & \alpha & \beta \\ \alpha &  y+\beta & 1 \\ \beta & 1 & y+\alpha \end{bmatrix}$ is equal to
\begin{enumerate}
 \item $y(y^2-1)$
 \item $y(y^2-3)$
 \item $y^3$
 \item $y^3$-1
\end{enumerate}
\end{enumerate}
%\end{document}
    
