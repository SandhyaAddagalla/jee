\renewcommand{\theequation}{\theenumi}
\begin{enumerate}[label=\arabic*.,ref=\thesubsection.\theenumi]
\numberwithin{equation}{enumi}
\item Two parabolas with a common vertex and with axes along $x$-axis and $y$-axis, respectively, intersect 
each other in the first quadrant.  If the length of the latus rectum of each parabola is 3, find the equation 
of the common tangent to the two parabolas.
\\
\solution The equation of a conic is given by 
%
\begin{align}
\label{eq:conic_gen}
\vec{x}^T\vec{V}\vec{x}+2\vec{u}^T\vec{x}+F=0
\end{align}
%
For the standard parabola, 
\begin{align}
\label{eq:conic_parab_V}
\vec{V}&=\myvec{0 & 0 \\ 0& 1}
\\
\vec{u}&=-2a\myvec{1 \\ 0}
\label{eq:conic_parab_u}
\\
F&=0
\label{eq:conic_parab_const}
\end{align}
%
The focus
\begin{align}
\label{eq:conic_parab_focus}
\vec{F}&=a\myvec{1 \\ 0}
\end{align}
%
The Latus rectum is the line passing through $\vec{F}$ with direction vector 
\begin{align}
\label{eq:conic_parab_latus_m}
\vec{m}&=\myvec{0 \\ 1}
\end{align}
%
Thus, the equation of the Latus rectum is 
\begin{align}
\label{eq:conic_parab_latus}
\vec{x}&= \vec{F}+\lambda \vec{m}
\end{align}
%
The intersection of the latus rectum and the parabola is obtained from \eqref{eq:conic_parab_const}
, \eqref{eq:conic_parab_latus}
and \eqref{eq:conic_gen} as
\begin{align}
\brak{\vec{F}+\lambda \vec{m}}^T\vec{V}\brak{\vec{F}+\lambda \vec{m}}+2\vec{u}^T\brak{\vec{F}+\lambda \vec{m}}=0
\end{align}
\begin{multline}
\label{eq:conic_parab_lam_quad}
\implies \brak{\vec{m}^T\vec{V}\vec{m}}\lambda^2+2\brak{\vec{V}\vec{F}+\vec{u}}^T\vec{m}\lambda 
\\
+\brak{\vec{V}\vec{F}+2\vec{u}}^T\vec{F} = 0
\end{multline}
From \eqref{eq:conic_parab_V}, \eqref{eq:conic_parab_u}, \eqref{eq:conic_parab_focus} and \eqref{eq:conic_parab_latus_m}, 
\begin{align}
\label{eq:conic_parab_lam_a}
\vec{m}^T\vec{V}\vec{m} &= 1
\\
\label{eq:conic_parab_lam_b}
\brak{\vec{V}\vec{F}+\vec{u}}^T\vec{m} &= 0
\\
\brak{\vec{V}\vec{F}+2\vec{u}}^T\vec{F} &= -4a^2
\label{eq:conic_parab_lam_c}
\end{align}
Substituting from \eqref{eq:conic_parab_lam_a}, \eqref{eq:conic_parab_lam_b}  and \eqref{eq:conic_parab_lam_c} in \eqref{eq:conic_parab_lam_quad}, 
\begin{align}
\lambda^2  - 4a^2 &=0
\\
\implies \lambda_1 &=2a,\lambda_2 =-2a
\label{eq:conic_parab_latus_lam}
\end{align}
%
Thus, from \eqref{eq:conic_parab_latus_m}, \eqref{eq:conic_parab_latus}
 and \eqref{eq:conic_parab_latus_lam}, the length of the latus rectum is
\begin{align}
\brak{\lambda_1-\lambda_2} \norm{\vec{m}} = 4a
\end{align}
%
From the given information, the two parabolas $P_1, P_2$ have parameters
\begin{align}
\label{eq:conic_parab_param1}
\vec{V}_1&=\myvec{0 & 0 \\ 0& 1},\vec{u}_1=-2a\myvec{1 \\ 0},F_1=0
\\
\label{eq:conic_parab_param2}
\vec{V}_2&=\myvec{1 & 0 \\ 0& 0},
\vec{u}_2=-2a\myvec{0 \\ 1},
F_2=0
\\
4a&=3
\label{eq:conic_parab_a}
\end{align}
Let $L$ be the common tangent for $P_1,P_2$ with  $\vec{c},\vec{d}$ being the respective points of contact. The respectivel normal vectors are
\begin{align}
\vec{n}_1 &= \vec{V}_1\vec{c}+\vec{u}_1 = -2a\myvec{1\\-\frac{c_2}{2a}}
\\
\vec{n}_2 &= \vec{V}_2\vec{d}+\vec{u}_2  = d_1\myvec{1\\-\frac{2a}{d_1}}
\end{align}
%
From the above equations, since both normals have the same direction vector, 
\begin{align}
\myvec{1\\-\frac{c_2}{2a}}&=\myvec{1\\-\frac{2a}{d_1}}
\implies c_2d_1 = 4a^2
\end{align}
%Also, 
%\begin{align}
%d_2 = 4ad_1^2, c_1 = 4ac_2^2
%\end{align}
%Thus,
%\begin{align}
%\vec{c} = \myvec{\frac{64a^5}{d_1^2}\\ \frac{4a^2}{d_1}}, \vec{d} = \myvec{d_1\\ 4ad_1^2}
%\end{align}

%\item A hyperbola passes through the point 
%\begin{equation}
%\vec{P}=\myvec{\sqrt{2}\\ \sqrt{3}}
%\end{equation}
%and has foci at $\myvec{\pm 2\\ 0}$.  Find the equation of the tangent to this hyperbola at 
%$\vec{P}$.
%\\
%\solution Let 
%\begin{align}
%\vec{F}_1 = 2\myvec{1 \\ 0},
%\vec{F}_2 = -2\myvec{1 \\ 0}
%\end{align}
%%
%Comparing with \eqref{eq:conic_gen}
%%
%\begin{align}
%\vec{V} &= \myvec{\frac{1}{a^2} & 0 \\ 0 & -\frac{1}{b^2}}, 
%\vec{u} = 0,
%F=-1,
%\\
%  a^2+b^2 &= 4
%\end{align}
%%
%The equation to the tangent is then given by 
%\begin{align}
%\brak{\vec{\vec{V}\vec{P}}}^T\vec{x} &= 1
%\implies \myvec{\frac{\sqrt{2}}{a^2} & -\frac{\sqrt{3}}{b^2}}\vec{x} &=1
%\end{align}

\item Find the product of the perpendiculars drawn from the foci of the ellipse
\begin{equation}
\label{eq:ellipse_prod}
\vec{x}^T\myvec{25 & 0 \\ 0 & 9}\vec{x}  = 225
\end{equation}
upon the tangent to it at the point
\begin{equation}
\frac{1}{2}\myvec{3\\ 5 \sqrt{3}}
\end{equation}
\\
\solution For the ellipse in \eqref{eq:ellipse_prod},
\begin{align}
\label{eq:ellipse_prod_param}
V = \myvec{\frac{1}{9} & 0 \\ 0 & \frac{1}{25}}, \vec{u} = 0, F=-1
\end{align}
%
The equation of the desired tangent is
\begin{align}
\label{eq:ellipse_prod_tangent}
\brak{\vec{V}\vec{P}}^T\vec{x}&=1
\\
\implies \myvec{\frac{1}{3} & \frac{\sqrt{3}}{5}}\vec{x}&=2
\end{align}
%
The foci of the ellipse are located at
\begin{align}
\vec{F}_1 = \myvec{0\\4},
\vec{F}_2 = \myvec{0\\-4}
\end{align}
The product of the perpendiculars is
\begin{align}
\frac{\abs{\myvec{\frac{1}{3} & \frac{\sqrt{3}}{5}}\myvec{0\\4}-2}\abs{\myvec{\frac{1}{3} & \frac{\sqrt{3}}{5}}\myvec{0\\-4}-2}}{\norm{\myvec{\frac{1}{3} & \frac{\sqrt{3}}{5}}}^2}
=9
\end{align}

\item  Consider an ellipse, whose centre is at the origin and its major axis is along the $x$-axis.  If its 
eccentricity is $\frac{3}{5}$ and the distance between its foci is 6, then find the area of the quadrilateral 
inscribed in the ellipse,  with the vertices as the vertices of the ellipse.
\\
\solution If $a$ and $b$ be the semi-major and minor-axis respectively, 
the foci of the ellipse are 
\begin{align}
\vec{F}_1=ae\myvec{1\\0},
\vec{F}_2=-ae\myvec{1\\0}
\end{align}
%
From the given information,
\begin{align}
e &= \frac{3}{5}, 2ae = 6 
\\
\implies a &= 5, b = a\sqrt{1-e^2} = 4
\end{align}
%
Thus, the vertices of the ellipse are
\begin{align}
\myvec{a\\0},
\myvec{-a\\0},
\myvec{0\\b},
\myvec{0\\-b}
\end{align}
and the area of the quadrilateral is
\begin{align}
\frac{1}{2}
\begin{vmatrix}
1 & 1 & 1
\\
a & -a & 0
\\
0 & 0 & b
\end{vmatrix}
+
\frac{1}{2}
\begin{vmatrix}
1 & 1 & 1
\\
a & -a & 0
\\
0 & 0 & -b
\end{vmatrix}
=2ab = 40
\end{align}

\item Let $a$ and $b$ respectively be the semi-transverse and semi-conjugate axes of a hyperbola whose 
eccentricity satisfies the equation
\begin{equation}
\label{eq:hyper_ecc}
9e^2-18e+5 = 0
\end{equation}
If 
\begin{equation} 
\label{eq:hyper_ecc_focus}
\vec{S}=\myvec{5\\ 0}
\end{equation}
is a focus and 
\begin{equation} 
\label{eq:hyper_ecc_direc}
\myvec{5 & 0}\vec{x} = 9
\end{equation} 
%
is the corresponding directrix of this hyperbola, then find $a^2-b^2$.
\\
\solution From \eqref{eq:hyper_ecc},
\begin{align}
\brak{3e-1}\brak{3e-5} = 0
\\
\implies e = \frac{5}{3}, \because e > 1
\label{eq:hyper_ecc_e}
\end{align}
%
for a hyperbola.  
Let $\vec{x}$ be a point on the hyperbola.  
From \eqref{eq:hyper_ecc_direc}, its distance from the directrix is
%
\begin{align}
\label{eq:hyper_ecc_direc_dist}
\frac{\abs{\myvec{5 & 0}\vec{x} - 9}}{5}
\end{align}
%
and from the focus is 
\begin{align}
\label{eq:hyper_ecc_focus_dist}
\norm{\vec{x}-\myvec{5\\0}}
\end{align}
From the definition of a hyperbola, the eccentricity is the ratio of these distances and \eqref{eq:hyper_ecc_e}
, \eqref{eq:hyper_ecc_direc_dist}
 and \eqref{eq:hyper_ecc_focus_dist},
\begin{align}
\frac{5\norm{\vec{x}-\myvec{5\\0}}}{\abs{\myvec{5 & 0}\vec{x} - 9}} &= \frac{5}{3}
\\
\implies 9\cbrak{\brak{x_1-5}^2+x_2^2}&=  \brak{5x_1-9}^2
\\
\text{or, } \vec{x}^T\myvec{16 & 0 \\ 0 & -9}\vec{x} = 225
\end{align}
%
which is the equation of the hyperbola.  Thus, 
\begin{align}
a^2 = \frac{225}{16}, b^2 = \frac{225}{9}
\\
\implies a^2-b^2 = -\frac{175}{16}
\end{align}


%From \eqref{eq:hyper_ecc_focus},
%\begin{align}
%ae = 5 \implies a= 3
%\end{align}

\item A variable line drawn through the 
intersection of the lines 
\begin{align} 
\label{lines_3}
\myvec{4 & 3}\vec{x} &=12 
\\ 
\myvec{3 & 4}\vec{x} &=12 
\end{align} 
meets the coordinate axes at $\vec{A}$ and $\vec{B}$, then find the locus of the midpoint of $AB$. 
\\
\solution The intersection of the lines in \eqref{lines_3} is obtained through the matrix equation 
\begin{align}
\myvec{4 & 3 \\ 3 & 4}\vec{x}  &=\myvec{12\\12}
\end{align}
by forming the augmented matrix and row reduction as  
\begin{align}
\myvec{4 & 3 &12 \\ 3 & 4 &12} &\leftrightarrow \myvec{4 & 3 &12 \\ 0 & 7 &12} \leftrightarrow \myvec{28 & 0 & 48 \\ 0 & 7 &12}
\nonumber \\
\leftrightarrow \myvec{7 & 0 & 12 \\ 0 & 7 &12}&
\end{align}
resulting in 
\begin{align}
\vec{C}=\frac{1}{7}\myvec{12\\12}
\end{align}
%
Let the $\vec{R}$ be the mid point of $AB$. Then,
\begin{align}
\label{eq:lines_3_abr}
\vec{A} =2 \myvec{1 & 0\\0 & 0}\vec{R} 
\\
\vec{B} =2 \myvec{0 & 0\\0 & 1}\vec{R} 
\end{align}
%
Let the equation of $AB$ be 
%equation of a line passing through $\vec{C}$ be 
\begin{align}
\vec{n}^T\brak{\vec{x} - \vec{C}} = 0
\label{eq:lines_3_def}
\end{align}
Since $\vec{R}$ lies on $AB$, 
\begin{align}
\vec{n}^T\brak{\vec{R} - \vec{C}} = 0
\label{eq:lines_3_rc}
\end{align}
Also, 
\begin{align}
\vec{n}^T\brak{\vec{A} - \vec{B}} = 0
\label{eq:lines_3_ab}
\end{align}
%
Substituting from \eqref{eq:lines_3_abr} in \eqref{eq:lines_3_ab},
\begin{align}
\vec{n}^T\myvec{1 & 0 \\ 0 & -1}\vec{R} = 0
\label{eq:lines_3_r_sub}
\end{align}
%
From \eqref{eq:lines_3_rc} and \eqref{eq:lines_3_r_sub},
\begin{align}
\brak{\vec{R} - \vec{C}} = k\myvec{1 & 0 \\ 0 & -1}\vec{R}
\label{eq:lines_3_k}
\end{align}
%
for some constant $k$.
Multiplying both sides of \eqref{eq:lines_3_k} by 
\begin{align}
\vec{R}^T\myvec{0 & 1 \\ 1 & 0},
\end{align}
\begin{align}
\vec{R}^T\myvec{0 & 1 \\ 1 & 0}\brak{\vec{R} - \vec{C}} &= k\vec{R}^T\myvec{0 & 1 \\ 1 & 0}\myvec{1 & 0 \\ 0 & -1}\vec{R}
\nonumber \\
&= k\vec{R}^T\myvec{0 & -1 \\ 1 & 0}\vec{R} = 0
\label{eq:lines_3_mul}
\end{align}
\begin{align}
\because \vec{R}^T\myvec{0 & -1 \\ 1 & 0}\vec{R} = 0
\end{align}
which can be easily verified for any $\vec{R}$.
%
from \eqref{eq:lines_3_mul},
\begin{align}
\vec{R}^T\myvec{0 & 1 \\ 1 & 0}\brak{\vec{R} - \vec{C}} = 0
\nonumber \\
\implies \vec{R}^T\myvec{0 & 1 \\ 1 & 0}\vec{R} - \vec{R}^T\myvec{0 & 1 \\ 1 & 0}\vec{C} = 0
\nonumber \\
\implies \vec{R}^T\myvec{0 & 1 \\ 1 & 0}\vec{R} - \vec{C}^T\myvec{0 & 1 \\ 1 & 0}\vec{R} = 0
\end{align}
%
which is the desired locus.
\end{enumerate}
