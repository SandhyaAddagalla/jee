\documentclass[journal,12pt,onecolumn]{IEEEtran}
%
\usepackage{setspace}
\usepackage{gensymb}
%\doublespacing
\singlespacing

%\usepackage{graphicx}
%\usepackage{amssymb}
%\usepackage{relsize}
\usepackage[cmex10]{amsmath}
%\usepackage{amsthm}
%\interdisplaylinepenalty=2500
%\savesymbol{iint}
%\usepackage{txfonts}
%\restoresymbol{TXF}{iint}
%\usepackage{wasysym}
\usepackage{amsthm}
%\usepackage{iithtlc}
\usepackage{mathrsfs}
\usepackage{txfonts}
\usepackage{stfloats}
\usepackage{bm}
\usepackage{cite}
\usepackage{cases}
\usepackage{subfig}
%\usepackage{xtab}
\usepackage{longtable}
\usepackage{multirow}
%\usepackage{algorithm}
%\usepackage{algpseudocode}
\usepackage{enumitem}
\usepackage{mathtools}
\usepackage{tikz}
\usepackage{circuitikz}
\usepackage{verbatim}
%\usepackage{tfrupee}
\usepackage[breaklinks=true]{hyperref}
%\usepackage{stmaryrd}
\usepackage{tkz-euclide} % loads  TikZ and tkz-base
\usetkzobj{all}
\usepackage{listings}
    \usepackage{color}                                            %%
    \usepackage{array}                                            %%
    \usepackage{longtable}                                        %%
    \usepackage{calc}                                             %%
    \usepackage{multirow}                                         %%
    \usepackage{hhline}                                           %%
    \usepackage{ifthen}                                           %%
  %optionally (for landscape tables embedded in another document): %%
    \usepackage{lscape}     
\usepackage{multicol}
\usepackage{chngcntr}
%\usepackage{enumerate}

%\usepackage{wasysym}
%\newcounter{MYtempeqncnt}
\DeclareMathOperator*{\Res}{Res}
%\renewcommand{\baselinestretch}{2}
\renewcommand\thesection{\arabic{section}}
\renewcommand\thesubsection{\thesection.\arabic{subsection}}
\renewcommand\thesubsubsection{\thesubsection.\arabic{subsubsection}}

\renewcommand\thesectiondis{\arabic{section}}
\renewcommand\thesubsectiondis{\thesectiondis.\arabic{subsection}}
\renewcommand\thesubsubsectiondis{\thesubsectiondis.\arabic{subsubsection}}

% correct bad hyphenation here
\hyphenation{op-tical net-works semi-conduc-tor}
\def\inputGnumericTable{}                                 %%

\lstset{
%language=C,
frame=single, 
breaklines=true,
columns=fullflexible
}
%\lstset{
%language=tex,
%frame=single, 
%breaklines=true
%}

\begin{document}
%


\newtheorem{theorem}{Theorem}[section]
\newtheorem{problem}{Problem}
\newtheorem{proposition}{Proposition}[section]
\newtheorem{lemma}{Lemma}[section]
\newtheorem{corollary}[theorem]{Corollary}
\newtheorem{example}{Example}[section]
\newtheorem{definition}[problem]{Definition}
%\newtheorem{thm}{Theorem}[section] 
%\newtheorem{defn}[thm]{Definition}
%\newtheorem{algorithm}{Algorithm}[section]
%\newtheorem{cor}{Corollary}
\newcommand{\BEQA}{\begin{eqnarray}}
\newcommand{\EEQA}{\end{eqnarray}}
\newcommand{\define}{\stackrel{\triangle}{=}}

\bibliographystyle{IEEEtran}
%\bibliographystyle{ieeetr}


\providecommand{\mbf}{\mathbf}
\providecommand{\pr}[1]{\ensuremath{\Pr\left(#1\right)}}
\providecommand{\qfunc}[1]{\ensuremath{Q\left(#1\right)}}
\providecommand{\sbrak}[1]{\ensuremath{{}\left[#1\right]}}
\providecommand{\lsbrak}[1]{\ensuremath{{}\left[#1\right.}}
\providecommand{\rsbrak}[1]{\ensuremath{{}\left.#1\right]}}
\providecommand{\brak}[1]{\ensuremath{\left(#1\right)}}
\providecommand{\lbrak}[1]{\ensuremath{\left(#1\right.}}
\providecommand{\rbrak}[1]{\ensuremath{\left.#1\right)}}
\providecommand{\cbrak}[1]{\ensuremath{\left\{#1\right\}}}
\providecommand{\lcbrak}[1]{\ensuremath{\left\{#1\right.}}
\providecommand{\rcbrak}[1]{\ensuremath{\left.#1\right\}}}
\theoremstyle{remark}
\newtheorem{rem}{Remark}
\newcommand{\sgn}{\mathop{\mathrm{sgn}}}
\providecommand{\abs}[1]{\left\vert#1\right\vert}
\providecommand{\res}[1]{\Res\displaylimits_{#1}} 
\providecommand{\norm}[1]{\left\lVert#1\right\rVert}
%\providecommand{\norm}[1]{\lVert#1\rVert}
\providecommand{\mtx}[1]{\mathbf{#1}}
\providecommand{\mean}[1]{E\left[ #1 \right]}
\providecommand{\fourier}{\overset{\mathcal{F}}{ \rightleftharpoons}}
%\providecommand{\hilbert}{\overset{\mathcal{H}}{ \rightleftharpoons}}
\providecommand{\system}{\overset{\mathcal{H}}{ \longleftrightarrow}}
	%\newcommand{\solution}[2]{\textbf{Solution:}{#1}}
\newcommand{\solution}{\noindent \textbf{Solution: }}
\newcommand{\cosec}{\,\text{cosec}\,}
\providecommand{\dec}[2]{\ensuremath{\overset{#1}{\underset{#2}{\gtrless}}}}
\newcommand{\myvec}[1]{\ensuremath{\begin{pmatrix}#1\end{pmatrix}}}
\newcommand{\mydet}[1]{\ensuremath{\begin{vmatrix}#1\end{vmatrix}}}
%\numberwithin{equation}{section}
\numberwithin{equation}{subsection}
%\numberwithin{problem}{section}
%\numberwithin{definition}{section}
\makeatletter
\@addtoreset{figure}{problem}
\makeatother

\let\StandardTheFigure\thefigure
\let\vec\mathbf
%\renewcommand{\thefigure}{\theproblem.\arabic{figure}}
\renewcommand{\thefigure}{\theproblem}
%\setlist[enumerate,1]{before=\renewcommand\theequation{\theenumi.\arabic{equation}}
%\counterwithin{equation}{enumi}


%\renewcommand{\theequation}{\arabic{subsection}.\arabic{equation}}

\def\putbox#1#2#3{\makebox[0in][l]{\makebox[#1][l]{}\raisebox{\baselineskip}[0in][0in]{\raisebox{#2}[0in][0in]{#3}}}}
     \def\rightbox#1{\makebox[0in][r]{#1}}
     \def\centbox#1{\makebox[0in]{#1}}
     \def\topbox#1{\raisebox{-\baselineskip}[0in][0in]{#1}}
     \def\midbox#1{\raisebox{-0.5\baselineskip}[0in][0in]{#1}}

\vspace{3cm}

\title{
%	\logo{
Vector Algebra and Three Dimensional Geometry: JEE Maths
%	}
}
\author{ G V V Sharma$^{*}$% <-this % stops a space
	\thanks{*The author is with the Department
		of Electrical Engineering, Indian Institute of Technology, Hyderabad
		502285 India e-mail:  gadepall@iith.ac.in. All content in this manual is released under GNU GPL.  Free and open source.}
	
}	
%\title{
%	\logo{Matrix Analysis through Octave}{\begin{center}\includegraphics[scale=.24]{tlc}\end{center}}{}{HAMDSP}
%}


% paper titles
% can use linebreaks \\ within to get better formatting as desired
%\title{Matrix Analysis through Octave}
%
%
% author names and IEEE memberships
% note positions of commas and nonbreaking spaces ( ~ ) LaTeX will not break
% a structure at a ~ so this keeps an author's name from being broken across
% two lines.
% use \thanks{} to gain access to the first footnote area
% a separate \thanks must be used for each paragraph as LaTeX2e's \thanks
% was not built to handle multiple paragraphs
%

%\author{<-this % stops a space
%\thanks{}}
%}
% note the % following the last \IEEEmembership and also \thanks - 
% these prevent an unwanted space from occurring between the last author name
% and the end of the author line. i.e., if you had this:
% 
% \author{....lastname \thanks{...} \thanks{...} }s
%                     ^------------^------------^----Do not want these spaces!
%
% a space would be appended to the last name and could cause every name on that
% line to be shifted left slightly. This is one of those "LaTeX things". For
% instance, "\textbf{A} \textbf{B}" will typeset as "A B" not "AB". To get
% "AB" then you have to do: "\textbf{A}\textbf{B}"
% \thanks is no different in this regard, so shield the last } of each \thanks
% that ends a line with a % and do not let a space in before the next \thanks.
% Spaces after \IEEEmembership other than the last one are OK (and needed) as
% you are supposed to have spaces between the names. For what it is worth,
% this is a minor point as most people would not even notice if the said evil
% space somehow managed to creep in.



% The paper headers
%\markboth{Journal of \LaTeX\ Class Files,~Vol.~6, No.~1, January~2007}%
%{Shell \MakeLowercase{\textit{et al.}}: Bare Demo of IEEEtran.cls for Journals}
% The only time the second header will appear i/year/1963s for the odd numbered pages
% after the title page when using the twoside option.
% s
% *** Note that you probably will NOT want to include the author's ***
% *** name in the headers of peer review papers.                   ***
% You can use \ifCLASSOPTIONpeerreview for conditional compilation here if
% you desire.




% If you want to put a publisher's ID mark on the page you can do it like
% this:
%\IEEEpubid{0000--0000/00\$00.00~\copyright~2007 IEEE}
% Remember, if you use this you must call \IEEEpubidadjcol in the second
% column for its text to clear the IEEEpubid ma/year/1963rk.



% make the title area
\maketitle



%\tableofcontents

\bigskip

\renewcommand{\thefigure}{\theenumi}
\renewcommand{\thetable}{\theenumi}
%\renewcommand{\theequation}{\theenumi}

%\begin{abstract}
%%\boldmath
%In this letter, an algorithm for evaluating the exact analytical bit error rate  (BER)  for the piecewise linear (PL) combiner for  multiple relays is presented. Previous results were available only for upto three relays. The algorithm is unique in the sense that  the actual mathematical expressions, that are prohibitively large, need not be explicitly obtained. The diversity gain due to multiple relays is shown through plots of the analytical BER, well supported by simulations. 
%
%\end{abstract}
% IEEEtran.cls defaults to using nonbold math in the Abstract.
% This preserves the distinction between vectors and scalars. However,
% if the journal you are submitting to favors bold math in the abstract,
% then you can use LaTeX's standard command \boldmath ast the very start
% of the abstract to achieve this. Many IEEE journals frown on math
% in the abstract anyway.

% Note that keywords are not normally used for peerreview papers.
%\begin{IEEEkeywords}
%Cooperative diversity, decode and forward, piecewise linear
%\end{IEEEkeywords}



% For peer review papers, you can put extra information on the cover
% page as needed:
% \ifCLASSOPTIONpeerreview
% \begin{center} \bfseries EDICS Category: 3-BBND \end{center}
% \fi
%
% For peerreview papers, this IEEEtran command inserts a page break and
% creates the second title. It will be ignored for othesr modes.
%\IEEEpeerreviewmaketitle


%Download python codes using 
%\begin{lstlisting}
%svn co https://github.com/gadepall/school/trunk/ncert/computation/codes
%\end{lstlisting}

\renewcommand{\theequation}{\theenumi}
\begin{enumerate}[label=\arabic*.,ref=\theenumi]
%\begin{enumerate}[label=\arabic*.,ref=\thesubsection.\theenumi]
\numberwithin{equation}{enumi}
\item Let $\overrightarrow{A}$, $\overrightarrow{B}$, $\overrightarrow{C}$ be vectors of length 3, 4, 5 respectively. Let $\overrightarrow{A}$ be perpendicular to $\overrightarrow{B}$ + $\overrightarrow{C}$, $\overrightarrow{B}$ to 
$\overrightarrow{C}$ + $\overrightarrow{A}$ and $\overrightarrow{C}$ to $\overrightarrow{A}$ + $\overrightarrow{B}$. Then the length of vector $\overrightarrow{A}$ + $\overrightarrow{B}$ + $\overrightarrow{C}$ is..............

\item The unit vector perpendicular to the plane determined by P(1, -1, 2), Q(2, 0, -1) and R(0, 2, 1) is.........

\item The area of the triangle whose vertices are  A(1, -1, 2), B(2, 1, -1) and C(3, -1, 2) is........

\item A, B, C, and D are four points in a plane with position vectors a, b, c, and d respectively such that
\begin{align*}
(\overrightarrow{a} - \overrightarrow{d}) (\overrightarrow{b} - \overrightarrow{c}) 
= (\overrightarrow{b} - \overrightarrow{d}) (\overrightarrow{c} - \overrightarrow{a}) = 0
\end{align*}
The Point D, then is the.................of the triangle ABC.

\item If 
$\begin{vmatrix}
a & a^{2} & 1 + a^{3} \\ b & b^{2} & 1 + b^{3} \\ c & c^{2} & 1 + c^{3} 
\end{vmatrix} = 0$ 
and the vectors $\overrightarrow{A}$ = $(1, a, a^{2})$, $\overrightarrow{B}$ = $(1, b, b^{2})$, $\overrightarrow{C}$ = $(1, c, c^{2})$ are non-olar, then the product abc = ............

\item If $\overrightarrow{A} \overrightarrow{B} \overrightarrow{C}$ are three non-polar vectors, then 
$\frac{\overrightarrow{A} . \overrightarrow{B} \times \overrightarrow{C}}{\overrightarrow{C} \times \overrightarrow{A} . \overrightarrow{B}}$ + $\frac{\overrightarrow{B} . \overrightarrow{A} \times \overrightarrow{C}}{\overrightarrow{C} . \overrightarrow{A} \times \overrightarrow{B}}$ = ..............

\item If $\overrightarrow{A}$ = (1, 1, 1), $\overrightarrow{C}$ = (0, 1, -1) are given vectors, then a vector B satisfying the equations $\overrightarrow{A} \times \overrightarrow{B} = \overrightarrow{C}$ and 
$\overrightarrow{A} . \overrightarrow{B}$ = 3.............

\item If the vectors a$\hat{i}$ + $\hat{j}$ + $\hat{k}$, $\hat{i}$ + b$\hat{j}$ + $\hat{j}$, $\hat{i}$ + $\hat{j}$ + c$\hat{k}$ $(a \neq b \neq c \neq 1)$ are co-planar, then the value of 
$\frac{1}{1 - a} + \frac{1}{1 - b} + \frac{1}{1 - c}$ = ..........

\item Let b = 4$\hat{i}$ + 3$\hat{j}$ and $\overrightarrow{c}$ be two vectors perpendicular to each other in the xy - plane. All vectors in the same plane having projections 1 and 2 along $\overrightarrow{b}$ and $\overrightarrow{c}$, respectively are given by..........

\item The components of a vector $\overrightarrow{a}$ along and perpendicular to a non-zero vector 
$\overrightarrow{b}$ are............. and .............respectively.

\item Given that $\overrightarrow{a}$ = (1, 1, 1), $\overrightarrow{c}$ = (0, 1, -1), $\overrightarrow{a}\overrightarrow{b}$ = 3 and $\overrightarrow{a} \times \overrightarrow{b} = \overrightarrow{c}$, then $\overrightarrow{b}$ = .................

\item A unit vector co-planr with $\overrightarrow{i} + \overrightarrow{j} + 2\overrightarrow{k}$ and 
$\overrightarrow{i} + 2\overrightarrow{j} + \overrightarrow{k}$ and perpendicular to $\overrightarrow{i} + \overrightarrow{j} + \overrightarrow{k}$ is............

\item A unit vector perpendicular to the plane determined by the points P(1, -1, 2), Q(2, 0, -1) and R(0, 2, 1) is..........

\item A non-zero vector $\overrightarrow{a}$ is parallel to the line of intersection of the plane determined by the vectors $\hat{i}$, $\hat{i}$ + $\hat{j}$ and plane determined by the vectors $\hat{i}$ - $\hat{j}$, 
$\hat{i}$ + $\hat{k}$. The angle between $\overrightarrow{a}$ and the vector $\hat{i}$ - 2$\hat{j}$ + 2$\hat{k}$ is...........

\item If $\overrightarrow{b}$ and $\overrightarrow{c}$ are any two non-collinear unit vectors and $\overrightarrow{a}$ is any vector $(\overrightarrow{a} . \overrightarrow{b})\overrightarrow{b}$ + $(\overrightarrow{a} . \overrightarrow{c})\overrightarrow{c}$ + $\frac{\overrightarrow{a} . (\overrightarrow{b} \times \overrightarrow{c})}{\begin{vmatrix} \overrightarrow{b} \times \overrightarrow{c} \end{vmatrix}}$ 
$(\overrightarrow{b} \times \overrightarrow{c})$ = ..............

\item Let OA = a, OB = 10a + 2b and OC = b where O, A, and C are non-collinear points. Let p denote thea area of the quadrilateral OABC, and let q denote the area of the parallelgram with OA and OC as adjacent sides. If p = kq, then k = ............

\textbf{(B). True/False}

\item Let $\overrightarrow{A}$, $\overrightarrow{B}$ and $\overrightarrow{C}$ be unit vectors suppose that $\overrightarrow{A}.\overrightarrow{B}$ = $\overrightarrow{A}.\overrightarrow{C}$ = 0, and the angle between $\overrightarrow{B}$ and $\overrightarrow{C}$ is $\frac{\pi}{6}$. Then $\overrightarrow{A}$ = $\pm 2(\overrightarrow{B} \times \overrightarrow{C})$.

\item If X.A = 0, X.B = 0, X.C = 0 for some non-zero vector X, then [A B C] = 0.

\item The points with position vectors a + b, a - b, and a + kb are collinear for all real values of k.

\item For any three vectors $\overrightarrow{a}$, $\overrightarrow{b}$ and $\overrightarrow{c}$, $(\overrightarrow{a}  \overrightarrow{b}) . (\overrightarrow{b} - \overrightarrow{c}) \times (\overrightarrow{c} - \overrightarrow{a})$ = 2$\overrightarrow{a} . (\overrightarrow{b} \times \overrightarrow{c}$).

\textbf{(C). MCQs with One Correct Answer}
\item The scalar $\overrightarrow{A} . (\overrightarrow{B} + \overrightarrow{C}) \times (\overrightarrow{A} + \overrightarrow{B} + \overrightarrow{C})$ equals:
\begin{enumerate}
\item 0
\item $[\overrightarrow{A}  \overrightarrow{B}  \overrightarrow{C}] + [\overrightarrow{B} \overrightarrow{C} \overrightarrow{A}]$
\item $[\overrightarrow{A} \overrightarrow{B} \overrightarrow{C}]$
\item None of these
\end{enumerate}

\item For non-zero vectors $\overrightarrow{a}$, $\overrightarrow{b}$, $\overrightarrow{c}$, $| (\overrightarrow{a} \times \overrightarrow{b}) .  \overrightarrow{c} |$ = $|\overrightarrow{a}||\overrightarrow{b}||\overrightarrow{c}|$ holds if and only if 
\begin{enumerate}
\item $\overrightarrow{a}.\overrightarrow{b} = 0$, $\overrightarrow{b}.\overrightarrow{c} = 0$
\item $\overrightarrow{b}.\overrightarrow{c} = 0$, $\overrightarrow{c}.\overrightarrow{a} = 0$
\item $\overrightarrow{c}.\overrightarrow{a} = 0$, $\overrightarrow{a}.\overrightarrow{b} = 0$
\item $\overrightarrow{a}.\overrightarrow{b}$ = $\overrightarrow{b}.\overrightarrow{c}$  = $\overrightarrow{c}.\overrightarrow{a}$ = 0
\end{enumerate}

\item The volume of the parallelopiped whose sides are given by $\overrightarrow{OA}$ = 2i - 2j, $\overrightarrow{OB}$ = i + j - k, $\overrightarrow{OC}$ = 3i - k, is
\begin{enumerate}
\item $\frac{4}{13}$
\item 4
\item $\frac{2}{7}$
\item None of these
\end{enumerate}

\item The points with position vectors 60i + 3j, 40i - 8j, ai - 52j are collinear if
\begin{enumerate}
\item a = -40
\item a = 40
\item a = 20
\item None of these
\end{enumerate}

\item Let $\overrightarrow{a}$, $\overrightarrow{b}$, $\overrightarrow{c}$ be three non-coplanar vectors and $\overrightarrow{p}$, $\overrightarrow{q}$, $\overrightarrow{r}$, are vectors defined by the relations 
\begin{align*}
\overrightarrow{p} = \frac{\overrightarrow{b} \times \overrightarrow{c}}{[\overrightarrow{a}\overrightarrow{b}\overrightarrow{c}]}, \overrightarrow{q} = \frac{\overrightarrow{c} \times \overrightarrow{a}}{[\overrightarrow{a}\overrightarrow{b}\overrightarrow{c}]}, \overrightarrow{r} = \frac{\overrightarrow{a} \times \overrightarrow{b}}{[\overrightarrow{a}\overrightarrow{b}\overrightarrow{c}]}
\end{align*}
then the value of the expression $(\overrightarrow{a} + \overrightarrow{b}).\overrightarrow{p}$ + $(\overrightarrow{b} + \overrightarrow{c}).\overrightarrow{q}$ + $(\overrightarrow{c} + \overrightarrow{a}).\overrightarrow{r}$ is eqaul to
\begin{enumerate}
\item 0
\item 1
\item 2
\item 3
\end{enumerate}

\item Let a, b, c be distinct non-negative numbers. If the vectors a$\hat{i}$ + a$\hat{j}$ + c$\hat{k}$, 
$\hat{i}$ + $\hat{k}$ and c$\hat{i}$ + c$\hat{j}$ + b$\hat{k}$ lie in a plane, then c is
\begin{enumerate}
\item the arithmetic mean of a and b
\item the geomemetic mean of a and b
\item the harmonic mean of a and b
\item equal to zero
\end{enumerate}

\item Let $\overrightarrow{p}$ and $\overrightarrow{q}$ be the position vectors of P and Q respectively, with respect to O $|\overrightarrow{p}|$ = p, $|\overrightarrow{q}|$ = q. The points R and S divide PQ internally and externally in the ratio 2:3 respectively. If OR and OS are perpendicular then
\begin{enumerate}
\item $9q^{2} = 4q^{2}$
\item $4p^{2} = 9q^{2}$
\item 9p = 4q
\item 4p = 9q
\end{enumerate}

\item Let $\alpha$, $\beta$, $\gamma$ be distinct real numbers. The points with postion vectors 
$\alpha \hat{i}$ + $\beta \hat{j}$ + $\gamma \hat{k}$, $\beta \hat{i}$ + $\gamma \hat{j}$ + $\alpha \hat{k}$, $\gamma \hat{i}$ + $\alpha \hat{j}$ + $\beta \hat{k}$
\begin{enumerate}
\item are collinear
\item form an equilateral triangle
\item form an scalene triangle
\item form a right angled triangle
\end{enumerate}

\item Let $\overrightarrow{a}$ = $\hat{i}$ - $\hat{j}$, $\overrightarrow{b}$ = $\hat{j}$ - $\hat{k}$, 
$\overrightarrow{c}$ = $\hat{k}$ - $\hat{i}$. If $\overrightarrow{d}$ is a unit vector such that 
$\overrightarrow{a}$ . $\overrightarrow{d}$ = 0 = $[\overrightarrow{b} \overrightarrow{c} \overrightarrow{d}]$, then $\overrightarrow{d}$ equals
\begin{enumerate}
\item $\pm$ $\frac{\hat{i} + \hat{j} - 2\hat{k}}{\sqrt{6}}$
\item $\pm$ $\frac{\hat{i} + \hat{j} - \hat{k}}{\sqrt{3}}$
\item $\pm$ $\frac{\hat{i} + \hat{j} + 2\hat{k}}{\sqrt{3}}$
\item $\pm \hat{k}$
\end{enumerate}

\item If $\overrightarrow{a}$, $\overrightarrow{b}$, $\overrightarrow{c}$ are non coplanar vectors such that $\overrightarrow{a} \times (\overrightarrow{b} \times \overrightarrow{c})$ = $\frac{(\overrightarrow{b} + \overrightarrow{c})}{\sqrt{2}}$, then the angle between $\overrightarrow{a}$ and $\overrightarrow{b}$ is
\begin{enumerate}
\item $\frac{3\pi}{4}$
\item $\frac{\pi}{4}$
\item $\frac{\pi}{2}$
\item $\pi$
\end{enumerate}

\item Let $\overrightarrow{u}$, $\overrightarrow{v}$ and $\overrightarrow{w}$ be vectors such that $\overrightarrow{u}$ + $\overrightarrow{v}$ + $\overrightarrow{w}$ = 0. If $|\overrightarrow{u}|$ = 3, $|\overrightarrow{v}|$ = 4, $|\overrightarrow{w}|$ = 5, then $\overrightarrow{u}.\overrightarrow{v}$  + $\overrightarrow{v}.\overrightarrow{w}$ + $\overrightarrow{w}.\overrightarrow{u}$ is
\begin{enumerate}
\item 47
\item -25
\item 0
\item 25
\end{enumerate}

\item If $\overrightarrow{a}$, $\overrightarrow{b}$ and $\overrightarrow{c}$ are three non polar vectors, then $\overrightarrow{a}$ + $\overrightarrow{b}$ + $\overrightarrow{c}$ . [($\overrightarrow{a}$ + $\overrightarrow{b})$ $\times$ ($\overrightarrow{a}$ + $\overrightarrow{c})$] equals
\begin{enumerate}
\item 0
\item 1[$\overrightarrow{a}$  $\overrightarrow{b}$  $\overrightarrow{c}$]
\item 2[$\overrightarrow{a}$  $\overrightarrow{b}$  $\overrightarrow{c}$]
\item - [$\overrightarrow{a}$  $\overrightarrow{b}$  $\overrightarrow{c}$]
\end{enumerate}

\item Let a = 2i + j - 2k and b = i + j. If c is a vector such that a.c = $|c|$, $|c - a|$ = 2$\sqrt{2}$ and the angle between $(a \times b)$ and c is $30^{\degree}$, then $| (a \times b) \times c| $ = 
\begin{enumerate}
\item $\frac{2}{3}$
\item $\frac{3}{2}$
\item 2
\item 3
\end{enumerate}

\item a = 2i + j + k, b = i + 2j - k and a unit vector c be coplanar. If c is perpendicular to a, then c = 
\begin{enumerate}
\item $\frac{1}{\sqrt{2}}$(-j + k)
\item $\frac{1}{\sqrt{3}}$(-i - j - k)
\item $\frac{1}{\sqrt{5}}$(i - 2j)
\item $\frac{1}{\sqrt{3}}$(i - j - k)
\end{enumerate}

\item If the vectors $\overrightarrow{a}$, $\overrightarrow{b}$ and $\overrightarrow{c}$ form the sides BC, CA, AB respectively of a triangle ABC, then
\begin{enumerate}
\item $\overrightarrow{a} . \overrightarrow{b}$ + $\overrightarrow{b} . \overrightarrow{c}$ + $\overrightarrow{c} . \overrightarrow{a}$ = 0
\item $\overrightarrow{a} \times \overrightarrow{b}$ = $\overrightarrow{b} \times \overrightarrow{c}$ = $\overrightarrow{c} \times \overrightarrow{a}$
\item $\overrightarrow{a} . \overrightarrow{b}$ = $\overrightarrow{b} . \overrightarrow{c}$ = $\overrightarrow{c} . \overrightarrow{a}$ = 0
\item $\overrightarrow{a} \times \overrightarrow{b}$ + $\overrightarrow{b} \times \overrightarrow{c}$ + $\overrightarrow{c} \times \overrightarrow{a}$ = 0
\end{enumerate}

\item Let the vectors $\overrightarrow{a}$, $\overrightarrow{b}$, $\overrightarrow{c}$ and 
$\overrightarrow{d}$ be such that 
\begin{align*}
(\overrightarrow{a} \times \overrightarrow{b}) \times (\overrightarrow{c} \times \overrightarrow{d}) = 0.
\end{align*}
Let $P_1$ and $P_2$ be planes determined by the pairs of the vectors $\overrightarrow{a}$, $\overrightarrow{b}$ and $\overrightarrow{c}$, $\overrightarrow{d}$ respectively. Then the angle between $P_1$ and $P_2$ is
\begin{enumerate}
\item 0
\item $\frac{\pi}{4}$
\item $\frac{\pi}{3}$
\item $\frac{\pi}{2}$
\end{enumerate}

\item If $\overrightarrow{a}$, $\overrightarrow{b}$ and $\overrightarrow{c}$ are unit co-planar vectors, then the scalar triple product [2$\overrightarrow{a}$ - $\overrightarrow{b}$, 2$\overrightarrow{b}$ - $\overrightarrow{c}$, 2$\overrightarrow{c}$ - $\overrightarrow{a}$] = 
\begin{enumerate}
\item 0
\item 1
\item -$\sqrt{3}$
\item $\sqrt{3}$
\end{enumerate}

\item Let 
\begin{align*}
\overrightarrow{a} = \overrightarrow{i} - \overrightarrow{k}
\end{align*}
\begin{align*}
\overrightarrow{b} = x\overrightarrow{i} + \overrightarrow{j} + (1 - x)\overrightarrow{k}
\end{align*}
\begin{align*}
\overrightarrow{c} = y\overrightarrow{i} + x\overrightarrow{j} + (1 + x -y)\overrightarrow{k}
\end{align*}
Then $[\overrightarrow{a}\overrightarrow{b}\overrightarrow{c}]$ depends on
\begin{enumerate}
\item only x
\item only y
\item Neither x nor y
\item both x and y
\end{enumerate}

\item If $\overrightarrow{a}$, $\overrightarrow{b}$ and $\overrightarrow{c}$ are unit vectors, then 
\begin{align*}
|\overrightarrow{a} - \overrightarrow{b}|^{2} + |\overrightarrow{b} - \overrightarrow{c}|^{2} + |\overrightarrow{c} - \overrightarrow{a}|^{2}
\end{align*}
does not exceed.
\begin{enumerate}
\item 4
\item 9
\item 8
\item 6
\end{enumerate}

\item If $\overrightarrow{a}$ and $\overrightarrow{b}$ are two unit vectors such that $\overrightarrow{a}$ + 2$\overrightarrow{b}$ and 5$\overrightarrow{a}$ - 4$\overrightarrow{b}$ are perpendicular to each other then the angle between $\overrightarrow{a}$ and $\overrightarrow{b}$ is
\begin{enumerate}
\item $45^{\degree}$
\item $60^{\degree}$
\item $\cos^{-1}(\frac{1}{3})$
\item $\cos^{-1}(\frac{2}{7})$
\end{enumerate}

\item Let $\overrightarrow{V}$ = 2$\overrightarrow{i}$ + $\overrightarrow{j}$ - $\overrightarrow{k}$ and $\overrightarrow{W}$ = $\overrightarrow{i}$ + 3$\overrightarrow{k}$. If $\overrightarrow{U}$ is a unit vector, then the maximum value of the scalar triple product $|\overrightarrow{U}\overrightarrow{V}\overrightarrow{W}|$
is
\begin{enumerate}
\item -1
\item $\sqrt{10}$ + $\sqrt{6}$
\item $\sqrt{59}$
\item $\sqrt{60}$ 
\end{enumerate}

\item The value of k such that $\frac{x - 4}{1}$ = $\frac{y - 2}{1}$ = $\frac{z - k}{2}$ lies in the plane 2x - 4y + z = 7, is
\begin{enumerate}
\item 7
\item -7
\item no real value
\item 4
\end{enumerate}

\item The value of $'a'$ so that the volume of parallelopiped formed by $\overrightarrow{i}$ + a$\overrightarrow{j}$ + $\overrightarrow{k}$, $\overrightarrow{j}$ + a$\overrightarrow{k}$ and a$\overrightarrow{i}$ + $\overrightarrow{k}$ becomes minimum is
\begin{enumerate}
\item -3
\item 3
\item $\frac{1}{\sqrt{3}}$
\item $\sqrt{3}$
\end{enumerate}

\item If $\overrightarrow{a}$ = $(\overrightarrow{i} + \overrightarrow{j} + \overrightarrow{k})$, $\overrightarrow{a}$ . $\overrightarrow{b}$ = 1 and $\overrightarrow{a} \times \overrightarrow{b}$ = 
$\overrightarrow{j}$ - $\overrightarrow{k}$, then $\overrightarrow{b}$ is
\begin{enumerate}
\item $\overrightarrow{i} - \overrightarrow{j} + \overrightarrow{k}$
\item 2$\overrightarrow{j} - \overrightarrow{k}$
\item $\overrightarrow{i}$
\item 2$\overrightarrow{i}$
\end{enumerate}

\item If the lines 
\begin{align*}
\frac{x - 1}{2} = \frac{y + 1}{3} = \frac{z - 1}{4} and \frac{x - 3}{1} = \frac{y - k}{2} = \frac{z}{1}
\end{align*}
intersect, then the value of k is
\begin{enumerate}
\item 3/2
\item 9/2
\item -2/9
\item -3/2
\end{enumerate}

\item The unit vector which is orthogonal to the vector 3$\overrightarrow{i}$ + 2$\overrightarrow{j}$ + $\overrightarrow{k}$ and is co-planar with the vectors 2$\overrightarrow{i}$ + $\overrightarrow{j}$ + $\overrightarrow{k}$ and $\overrightarrow{i}$ - $\overrightarrow{j}$ + $\overrightarrow{k}$ is
\begin{enumerate}
\item $\frac{2\overrightarrow{i} - 6\overrightarrow{j} + \overrightarrow{k}}{\sqrt{41}}$
\item $\frac{2\overrightarrow{i} - 3\overrightarrow{j}}{\sqrt{13}}$
\item $\frac{3\overrightarrow{i} - \overrightarrow{k}}{10}$
\item $\frac{4\overrightarrow{i} + 3\overrightarrow{j} - 3\overrightarrow{k}}{34}$
\end{enumerate}

\item A variable plane at a distance of the one unit from the origin cuts the coordinates axes at A, B, and C. If the centroid D(x, y, z) of triangle ABC satisfies the relation
\begin{align*}
\frac{1}{x^{2}} + \frac{1}{y^{2}} + \frac{1}{z^{2}} = k
\end{align*}
then the value of k is
\begin{enumerate}
\item 3
\item 1
\item $\frac{1}{3}$
\item 9
\end{enumerate}

\item If $\overrightarrow{a}$, $\overrightarrow{b}$, $\overrightarrow{c}$ are three non-zero, non-polar vectors and 
\begin{align*}
\overrightarrow{b_1} = \overrightarrow{b} - \frac{\overrightarrow{b}.\overrightarrow{a}}{|\overrightarrow{a}|^{2}}\overrightarrow{a},
\end{align*}
\begin{align*}
\overrightarrow{b_2} = \overrightarrow{b} + \frac{\overrightarrow{b}.\overrightarrow{a}}{|\overrightarrow{a}|^{2}}\overrightarrow{a} 
\end{align*}
\begin{align*}
\overrightarrow{c_1} = \overrightarrow{c} - \frac{\overrightarrow{c}.\overrightarrow{a}}{|\overrightarrow{a}|^{2}}\overrightarrow{a} + \frac{\overrightarrow{b}.\overrightarrow{c}}{|\overrightarrow{c}|^{2}}\overrightarrow{b_1}
\end{align*}
\begin{align*}
\overrightarrow{c_2} = \overrightarrow{c} - \frac{\overrightarrow{c}.\overrightarrow{a}}{|\overrightarrow{a}|^{2}}\overrightarrow{a} - \frac{\overrightarrow{b_1}.\overrightarrow{c}}{|\overrightarrow{b_1}|^{2}}\overrightarrow{b_1}
\end{align*}
\begin{align*}
\overrightarrow{c_3} = \overrightarrow{c} - \frac{\overrightarrow{c}.\overrightarrow{a}}{|\overrightarrow{c}|^{2}}\overrightarrow{a} + \frac{\overrightarrow{b}.\overrightarrow{c}}{|\overrightarrow{c}|^{2}}\overrightarrow{b_1}
\end{align*}
\begin{align*}
\overrightarrow{c_4} = \overrightarrow{c} - \frac{\overrightarrow{c}.\overrightarrow{a}}{|\overrightarrow{c}|^{2}}\overrightarrow{a} = \frac{\overrightarrow{b}.\overrightarrow{c}}{|\overrightarrow{b}|^{2}}\overrightarrow{b_1}
\end{align*}
then the set of orthogonal vectors ts
\begin{enumerate}
\item $(\overrightarrow{a}, \overrightarrow{b_1}, \overrightarrow{c_3})$
\item $(\overrightarrow{a}, \overrightarrow{b_1}, \overrightarrow{c_2})$
\item $(\overrightarrow{a}, \overrightarrow{b_1}, \overrightarrow{c_1})$
\item $(\overrightarrow{a}, \overrightarrow{b_2}, \overrightarrow{c_2})$
\end{enumerate}

\item A plane which is perpendicular to two planes 
\begin{align}
2x - 2y + z = 0
\end{align}
\begin{align}
x - y + 2z = 4
\end{align}
passes through (1, -2 ,1). The distance of the plane from the point (1, 2, 2) is
\begin{enumerate}
\item 0
\item 1
\item $\sqrt{2}$
\item 2$\sqrt{2}$
\end{enumerate}

\item Let $\overrightarrow{a}$ = $\hat{i}$ + 2$\hat{j}$ + $\hat{k}$, $\overrightarrow{b}$ = $\hat{i}$ - $\hat{j}$ + $\hat{k}$ and $\overrightarrow{c}$ = $\hat{i}$ + $\hat{j}$ - $\hat{k}$. A vector in the plane of $\overrightarrow{a}$ and $\overrightarrow{b}$ whose projection on $\overrightarrow{c}$ is $\frac{1}{\sqrt{3}}$ is
\begin{enumerate}
\item $4\hat{i}$ - $\hat{j}$ + 4$\hat{k}$
\item $3\hat{i}$ + $\hat{j}$ - 3$\hat{k}$
\item $2\hat{i}$ + $\hat{j}$ - 2$\hat{k}$
\item $4\hat{i}$ + $\hat{j}$ - 4$\hat{k}$
\end{enumerate}

\item The number of distinct real values of $\lambda$, for which the vectors $-\lambda^{2}\hat{i}$ + $\hat{j}$ + $\hat{k}$, $\hat{i}$ - $\lambda^{2}\hat{j}$ + $\hat{k}$ and $\hat{i}$ + $\hat{j}$ - $\lambda^{2}\hat{k}$ are coplanar is
\begin{enumerate}
\item 0
\item 1
\item 2
\item 3
\end{enumerate}

\item Let $\overrightarrow{a}$, $\overrightarrow{b}$, $\overrightarrow{c}$ be unit vectors such that 
$\overrightarrow{a}$ + $\overrightarrow{b}$ + $\overrightarrow{c}$ = $\overrightarrow{0}$. Which one of the following is correct?
\begin{enumerate}
\item $(\overrightarrow{a} \times \overrightarrow{b})$ = $(b \times \overrightarrow{c})$ = $(\overrightarrow{c} \times \overrightarrow{a})$ = $\overrightarrow{0}$
\item $(\overrightarrow{a} \times \overrightarrow{b})$ = $(b \times \overrightarrow{c})$ = $(\overrightarrow{c} \times \overrightarrow{a})$ $\neq$ $\overrightarrow{0}$
\item $(\overrightarrow{a} \times \overrightarrow{b})$ = $(b \times \overrightarrow{c})$ = $(\overrightarrow{a} \times \overrightarrow{c})$ $\neq$ $\overrightarrow{0}$
\item $(\overrightarrow{a} \times \overrightarrow{b})$, $(b \times \overrightarrow{c})$, $(\overrightarrow{c} \times \overrightarrow{a})$ are mutually perpendicular.
\end{enumerate}

\item The edges of a parallelopiped are of unit length and are to parallel to non-coplanar unit vectors $\hat{a}$, $\hat{b}$, $\hat{c}$ such that $\hat{a}$.$\hat{b}$ = $\hat{b}$.$\hat{c}$ = $\hat{c}$.$\hat{a}$ = $\frac{1}{2}$. Then, the volume of the parallelopiped is
\begin{enumerate}
\item $\frac{1}{\sqrt{2}}$
\item $\frac{1}{2\sqrt{2}}$
\item $\frac{\sqrt{3}}{2}$
\item $\frac{1}{\sqrt{3}}$
\end{enumerate}

\item Let two non-collinear vectors $\hat{a}$ and $\hat{b}$ form an acute angle. A point P moves so that at any time t the position vector $\overrightarrow{OP}$(where O is the origin) is given by $\hat{a}\cos t + \hat{b}\sin t$. When P is farthest from origin O, let M be the length of $\overrightarrow{OP}$ and $\hat{u}$ be the unit vector along $\overrightarrow{OP}$. Then,
\begin{enumerate}
\item $\hat{u}$ = $\frac{\hat{a} + \hat{b}}{|\hat{a} + \hat{b}|}$ and M = $(1 + \hat{a}.\hat{b})^{1/2}$
\item $\hat{u}$ = $\frac{\hat{a} - \hat{b}}{|\hat{a} - \hat{b}|}$ and M = $(1 + \hat{a}.\hat{b})^{1/2}$
\item $\hat{u}$ = $\frac{\hat{a} + \hat{b}}{|\hat{a} + \hat{b}|}$ and M = $(1 + 2\hat{a}.\hat{b})^{1/2}$
\item $\hat{u}$ = $\frac{\hat{a} - \hat{b}}{|\hat{a} - \hat{b}|}$ and M = $(1 + 2\hat{a}.\hat{b})^{1/2}$
\end{enumerate}

\item Let P(3, 2, 6) be a point in a space and Q be a point on the line 
\begin{align*}
\overrightarrow{r} = (\hat{i} - \hat{j} + 2\hat{k}) + \mu(-3\hat{i} + \hat{j} + 5\hat{k})
\end{align*}
Then the value of $\mu$ for which the vector $\overrightarrow{PQ}$ is parallel to the plane x - 4y + 3z = 1 is
\begin{enumerate}
\item $\frac{1}{4}$
\item $\frac{-1}{4}$
\item $\frac{1}{8}$
\item $\frac{-1}{8}$
\end{enumerate}

\item If $\overrightarrow{a}$, $\overrightarrow{b}$, $\overrightarrow{c}$ and $\overrightarrow{d}$ are unit vectors such that $(\overrightarrow{a} \times \overrightarrow{b}$) . $(\overrightarrow{c} \times\overrightarrow{d}$) = 1 and $\overrightarrow{a}$.$\overrightarrow{c}$ = $\frac{1}{2}$, then
\begin{enumerate}
\item $\overrightarrow{a}$, $\overrightarrow{b}$, $\overrightarrow{c}$ are non-polar
\item $\overrightarrow{b}$, $\overrightarrow{c}$, $\overrightarrow{d}$ are non-polar
\item $\overrightarrow{b}$, $\overrightarrow{d}$ are non-parallel
\item $\overrightarrow{a}$, $\overrightarrow{d}$ are parallel and $\overrightarrow{b}$, $\overrightarrow{c}$ are parallel 
\end{enumerate}

\item A line with positive direction cosines passes through the point P(2, -1, 2) and make equal angles with the coordinate axes. The line meets the plane 2x + y + z = 9 at a point Q. The length of the line segment PQ equals
\begin{enumerate}
\item 1
\item $\sqrt{2}$
\item $\sqrt{3}$
\item 2
\end{enumerate}

\item Let P, Q, R and S be the points on the plane with position vectors -2$\hat{i}$ - $\hat{j}$, 4$\hat{i}$, 3$\hat{i}$ + 3$\hat{j}$ and -3$\hat{i}$ + 2$\hat{j}$ respectively. The qudrilateral PQRS must be a
\begin{enumerate}
\item parallelgram, which is neither a rhombus nor a rectangle
\item square
\item rectangle, but not a square
\item rhombus, but a square
\end{enumerate}

\item Equation of the plane containing the straight line 
\begin{align*}
\frac{x}{2} = \frac{y}{3} = \frac{z}{4}
\end{align*}
and perpendicular to the plane containing the straight lines 
\begin{align*}
\frac{x}{3} = \frac{y}{4} = \frac{z}{2} and \frac{x}{4} = \frac{y}{2} = \frac{z}{3}
\end{align*}
is
\begin{enumerate}
\item x + 2y - 2z = 0
\item 3x + 2y - 2z = 0
\item x - 2y + z = 0
\item 5x + 2y - 4z = 0
\end{enumerate}

\item If the distance of the point P(1, -2, 1) from the plane x + 2y - 2z = $\alpha$, where $\alpha > 0$, is 5, then the foot of the perpendicular from P to the plane is
\begin{enumerate}
\item ($\frac{8}{3}, \frac{4}{3}, \frac{-7}{3}$)
\item ($\frac{4}{3}, \frac{-4}{3}, \frac{1}{3}$)
\item ($\frac{1}{3}, \frac{2}{3}, \frac{10}{3}$)
\item ($\frac{2}{3}, \frac{-1}{3}, \frac{5}{2}$)
\end{enumerate}

\item Two adjacent sides of a parallelgram ABCD are given by $\overrightarrow{AB}$ = 2$\hat{i}$ + 10$\hat{j}$ + 11$\hat{k}$ and $\overrightarrow{AD}$ = $\hat{i}$ + 2$\hat{j}$ + 2$\hat{k}$. The side AD is rotated by an acute angle $\alpha$ in the plane of the parallelgram so that AD becomes AD'. If AD' makes a right angle with the side AB, then the cosine of the angle $\alpha$ os given by
\begin{enumerate}
\item $\frac{8}{9}$
\item $\frac{\sqrt{17}}{9}$
\item $\frac{1}{9}$
\item $\frac{4\sqrt{5}}{9}$
\end{enumerate}

\item Let $\overrightarrow{a}$ = $\hat{i}$ + $\hat{j}$ + $\hat{k}$, $\overrightarrow{b}$ = $\hat{i}$ - $\hat{j}$ + $\hat{k}$ and $\overrightarrow{c}$ = $\hat{i}$ - $\hat{j}$ - $\hat{k}$ be three vectors. A vector 
$\overrightarrow{v}$ in the plane of $\overrightarrow{a}$ and $\overrightarrow{b}$, whose projection on $\overrightarrow{c}$ is $\frac{1}{\sqrt{3}}$, is given by
\begin{enumerate}
\item $\hat{i}$ - 3$\hat{j}$ + 3$\hat{k}$
\item -3$\hat{i}$ - 3$\hat{j}$  -  $\hat{k}$
\item 3$\hat{i}$ - $\hat{j}$ + 3$\hat{k}$
\item $\hat{i}$ + 3$\hat{j}$ - 3$\hat{k}$
\end{enumerate}

\item The point P is the intersection of the straight line joining the points Q(2, 3, 5) and R(1, -1, 4) with the plane 5x - 4y - z = 1. If S is the foot of the perpendicular drawn from the point T(2, 1, 4) to QR, then 
the length of the line segment PS is
\begin{enumerate}
\item $\frac{1}{\sqrt{2}}$
\item $\sqrt{2}$
\item 2
\item $2\sqrt{2}$
\end{enumerate}

\item The equation of the plane passing through the line of intersection of the planes
\begin{align*}
x + 2y + 3z = 2
\end{align*}
\begin{align*}
x - y + z = 3
\end{align*}
and at a distance $\frac{2}{\sqrt{3}}$ from the point (3, 1, -1) is
\begin{enumerate}
\item 5x - 11y + z = 17
\item $\sqrt{3}$x + y = 3$\sqrt{2}$ - 1
\item x + y + z = $\sqrt{3}$
\item x - $\sqrt{2}$y = 1 - $\sqrt{2}$
\end{enumerate}

\item If $\overrightarrow{a}$ and $\overrightarrow{b}$ are vectors such that $|\overrightarrow{a} + \overrightarrow{b}|$ = $\sqrt{29}$and $\overrightarrow{a}$ $\times$ (2$\hat{i}$ + 3$\hat{j}$ + 4$\hat{k}$) = (2$\hat{i}$ + 3$\hat{j}$ + 4$\hat{k}$) $\times$ $\overrightarrow{b}$, then a possible value of ($\overrightarrow{a}$ + $\overrightarrow{b}$).(-7$\hat{i}$ + 2$\hat{j}$ + 3$\hat{k}$) is 
\begin{enumerate}
\item 0
\item 3
\item 4
\item 8
\end{enumerate}

\item Let P be the image of the point(3, 1, 7) with respect to the plane x - y + z = 3. Then the equation of the plane passing through P and containing the straight line $\frac{x}{1} = \frac{y}{z} = \frac{z}{1}$ is
\begin{enumerate}
\item x + y - 3z = 0
\item 3x + z = 0
\item x - 4y + 7z = 0
\item 2x - y = 0
\end{enumerate}

\item The equation of the plane passing through the point(1, 1, 1) and perpendicular to the planes 2x + y - 2z = 0 and 3x - 6y - 2z = 7, is
\begin{enumerate}
\item 14x + 2y - 15z = 1
\item 14x - 2y + 15z = 27
\item 14x + 2y + 15z = 31
\item -14x + 2y + 15z = 3
\end{enumerate}

\item Let O be the origin and let PQR be an arbitrary triangle. The points S is such that
\begin{align*}
\overrightarrow{OP}.\overrightarrow{OQ} + \overrightarrow{OR}.\overrightarrow{OS} = \overrightarrow{OR}.\overrightarrow{OP} + \overrightarrow{OQ}.\overrightarrow{OS}\\
 = \overrightarrow{OQ}.\overrightarrow{OR} + \overrightarrow{OP}.\overrightarrow{OS}
\end{align*}
Then the triangle PQR has S as its
\begin{enumerate}
\item Centroid
\item Circumcentre
\item Incentre
\item Orthocentre
\end{enumerate}

\textbf{(D). MCQs with One or More than One Correct}
\item Let $\overrightarrow{a}$ = $a_{1}i + a_{2}j + a_{3}k$, $\overrightarrow{b}$ = $b_{1}i + b_{2}j + b_{3}k$, $\overrightarrow{c}$ = $c_{1}i + c_{2}j + c_{3}k$, be three non-zero vectors such that $\overrightarrow{c}$ is a unit vector perpendicular to both the vectors $\overrightarrow{a}$ and $\overrightarrow{b}$. If the angle between $\overrightarrow{a}$ and $\overrightarrow{b}$ is $\frac{\pi}{6}$, then 
\begin{align*}
\begin{vmatrix} a_1 & a_2 & a_3 \\ b_1 & b_2 & b_3 \\ c_1 & c_2 & c_3  \end{vmatrix}^{2}
\end{align*}
is equal to
\begin{enumerate}
\item 0
\item 1
\item $\frac{1}{4}(a_1^{2} + a_2^{2} + a_2^{3})(b_1^{2} + b_2^{2} + b_3^{2})$
\item $\frac{1}{4}(a_1^{2} + a_2^{2} + a_3^{2})(b_1^{2} + b_2^{2} + b_3^{2})(c_1^{2} + c_2^{2} + c_3^{3})$
\end{enumerate}

\item The number of vectors of unit length perpendicular to vectors $\overrightarrow{a}$ = (1, 1, 0) and $\overrightarrow{b}$ = (0, 1, 1) is
\begin{enumerate}
\item 0
\item 1
\item 2
\item 3
\item $\infty$
\item None of these
\end{enumerate}

\item Let $\overrightarrow{a}$ = 2$\hat{i}$ - $\hat{j}$ + $\hat{k}$, $\overrightarrow{b}$ = $\hat{i}$ + $2\hat{j}$ - $\hat{k}$ and $\overrightarrow{c}$ = $\hat{i}$ + $\hat{j}$ - 2$\hat{k}$ be three vectors. A vector in the plane of $\overrightarrow{b}$ and $\overrightarrow{c}$, whose projection on $\overrightarrow{a}$ is magnitudu of $\sqrt{2/3}$, is
\begin{enumerate}
\item 2$\hat{i}$ + 3$\hat{j}$ - 3$\hat{k}$
\item 2$\hat{i}$ + 3$\hat{j}$ + 3$\hat{k}$
\item -2$\hat{i}$ - $\hat{j}$ + 5$\hat{k}$
\item 2$\hat{i}$ + $\hat{j}$ + 5$\hat{k}$
\end{enumerate}

\item The vector $\frac{1}{3}(2\hat{i} - 2\hat{j} + \hat{k})$ is
\begin{enumerate}
\item a unit vector
\item makes an angle $\frac{\pi}{3}$ with the vector (2$\hat{i}$ - 4$\hat{j}$ + 3$\hat{k}$)
\item parallel to the vector (-$\hat{i}$ + $\hat{j}$ - $\frac{1}{2}\hat{k}$)
\item perpendicular to the vector (3$\hat{i}$ + 2$\hat{j}$ - 2$\hat{k}$)
\end{enumerate}

\item If a = i + j + k, $\overrightarrow{b}$ = 4i + 3j + 4k and c = i + $\alpha$j + $\beta$k are linearly dependent vectors and $|c|=\sqrt{3}$, then
\begin{enumerate}
\item $\alpha$ = 1, $\beta$ = -1
\item $\alpha$ = 1, $\beta$ = $\pm$1
\item $\alpha$ = -1, $\beta$ = $\pm$1
\item $\alpha$ = $\pm$1, $\beta$ = 1
\end{enumerate}

\item For three vectors u, v, w which of the following expression in not equal to any one of the remaining three?
\begin{enumerate}
\item u.(v $\times$ w)
\item (v $\times$ w).u
\item v.(u $\times$ w)
\item (u $\times$ v).w
\end{enumerate}

\item Which of the following expressions are meaningful?
\begin{enumerate}
\item u(v $\times$ w)
\item (u . v).w
\item (u . v)w
\item u $\times$(v . w)
\end{enumerate}

\item Let a and b two non-collinear unit vectors. If u = a - (a.b)b and v = a $\times$ b, then $|v|$ is
\begin{enumerate}
\item $|u|$
\item $|u| + |u.a|$
\item $|u| + |u.b|$
\item $|u| + u.(a+b)$
\end{enumerate}

\item Let $\overrightarrow{A}$ be a parallel to line of intersection of planes $P_1$ and $P_2$. Plane $P_1$ is parallel to the vectors 2$\hat{j}$ + 3$\hat{k}$ and 4$\hat{j}$ - 3$\hat{k}$ and that $P_2$ is parallel to 
$\hat{j}$ - $\hat{k}$ and 3$\hat{i}$ + 3$\hat{j}$, then the angle between vector $\overrightarrow{A}$ and a given vector 2$\hat{i}$ + $\hat{j}$ - 2$\hat{k}$ is
\begin{enumerate}
\item $\frac{\pi}{2}$
\item $\frac{\pi}{4}$
\item $\frac{\pi}{6}$
\item $\frac{3\pi}{4}$
\end{enumerate}

\item The vectors which are coplanr with vectors $\hat{i}$ + $\hat{j}$ + 2$\hat{k}$ and $\hat{i}$ + 2$\hat{j}$ + $\hat{k}$, and perpendicular to the vector $\hat{i}$ + $\hat{j}$ + $\hat{k}$ are
\begin{enumerate}
\item $\hat{j}$ - $\hat{k}$
\item -$\hat{i}$ - $\hat{j}$
\item $\hat{i}$ - $\hat{j}$
\item -$\hat{j}$ + $\hat{k}$
\end{enumerate}

\item If the straight lines 
\begin{align*}
\frac{x-1}{2} = \frac{y+1}{k} = \frac{z}{2} and \frac{x+1}{5} = \frac{y+1}{2} = \frac{z}{k}
\end{align*}
are co-planar, then the plane(s) containing these two lines is(are)
\begin{enumerate}
\item y + 2z = -1
\item y + z = -1
\item y - z = -1
\item y - 2z = -1
\end{enumerate}

\item A line l passing through the origin is perpendicular to the lines
\begin{align*}
l_1: (3 + t)\hat{i} + (-1 + 2t)\hat{j} + (4 + 2t)\hat{k}, -\infty < t < \infty
\end{align*}
\begin{align*}
l_2: (3 + 2s)\hat{i} + (3 + 2s)\hat{j} + (2 + s)\hat{k}, -\infty < s < \infty
\end{align*}
Then, the coordinates of the points on $l_2$ at a distance of $\sqrt{17}$ from the point of intersection of l and $l_1$ is(are)
\begin{enumerate}
\item $(\frac{7}{5}, \frac{7}{3}, \frac{5}{3})$
\item (-1, -1, 0)
\item (1, 1, 1)
\item $(\frac{7}{9}, \frac{7}{9}, \frac{8}{9})$
\end{enumerate}

\item Two lines
\begin{align*}
L_1: x = 5, \frac{y}{3 - \alpha} = \frac{z}{-2}
\end{align*}
\begin{align*}
L_2: x = \alpha, \frac{y}{-1} = \frac{z}{2-\alpha}
\end{align*}
are coplanar. Then $\alpha$ can take value(s)
\begin{enumerate}
\item 1
\item 2
\item 3
\item 4
\end{enumerate}

\item Let $\overrightarrow{x}$, $\overrightarrow{y}$ and $\overrightarrow{z}$ be three vectors each of magnitude 
$\sqrt{2}$ and the angle between each pair of them is $\frac{\pi}{3}$. If $\overrightarrow{a}$ is a non-zero vector perpendicular to $\overrightarrow{x}$ and $\overrightarrow{y}$ $\times$ $\overrightarrow{z}$ and $\overrightarrow{b}$ is a non-zero vector perpendicular to $\overrightarrow{y}$ and $\overrightarrow{z} \times \overrightarrow{x}$, then
\begin{enumerate}
\item $\overrightarrow{b}$=($\overrightarrow{b} . \overrightarrow{z}$)($\overrightarrow{z} - \overrightarrow{x}$)
\item $\overrightarrow{a}$=($\overrightarrow{a} . \overrightarrow{y}$)($\overrightarrow{y} - \overrightarrow{z}$)
\item $\overrightarrow{a}$.$\overrightarrow{b}$ = -($\overrightarrow{a} . \overrightarrow{y}$)($\overrightarrow{b} . \overrightarrow{z}$)
\item $\overrightarrow{a}$ = ($\overrightarrow{a} . \overrightarrow{y}$)($\overrightarrow{z} - \overrightarrow{y}$)
\end{enumerate}

\item From a point P$(\lambda, \lambda, \lambda)$ perpendicular to PQ and PR are drawn respectively on the lines y = x, z = 1. If P is such that $\angle$ QPR is a right angle, then the possible value(s) of $\lambda$ is(are)
\begin{enumerate}
\item $\sqrt{2}$
\item 1
\item -1
\item -$\sqrt{2}$
\end{enumerate} 

\item In $R^{3}$, consider the planes $P_1$: y = 0 and $P_2$: X + Z = 1. Let $P_3$ be the plane, different from $P_1$ and $P_2$ , which passes through the intersection of $P_1$ and $P_2$. If the distance the point (0, 1, 0) from $P_3$ is 1 and the distance of a point $(\alpha, \beta, \gamma)$ from $P_3$ is 2, then which of the following relation is(are) true?
\begin{enumerate}
\item 2$\alpha$ + $\beta$ + 2$\gamma$ + 2 = 0
\item 2$\alpha$ - $\beta$ + 2$\gamma$ + 4 = 0
\item 2$\alpha$ + $\beta$ - 2$\gamma$ - 10 = 0
\item 2$\alpha$ - $\beta$ + 2$\gamma$ - 8 = 0
\end{enumerate} 

\item In $R^{3}$,let L be a straight line passing through the origin. Suppose that all the points on L are at a constant distance from the two planes 
\begin{align*}
P_1: x + 2y - z + 1 = 0
\end{align*}
\begin{align*}
P_2: 2x - y + z - 1 = 0
\end{align*}
Let M be the locus of the feet of the perpendiculars drawn from the points on L to the plane $P_1$. Which of the following points lie(s) on M?
\begin{enumerate}
\item $(0, \frac{-5}{6}, \frac{-2}{3})$
\item $(\frac{-1}{6}, \frac{-1}{3}, \frac{1}{6})$
\item $(\frac{-5}{6}, 0, \frac{1}{3})$
\item $\frac{-1}{3}, 0, \frac{2}{3})$
\end{enumerate}

\item Let $\Delta$ PQR be a triangle. Let $\overrightarrow{a}$ = $\overrightarrow{QR}$, $\overrightarrow{b}$ = 
$\overrightarrow{RP}$ and $\overrightarrow{c}$ = $\overrightarrow{PQ}$. If $|\overrightarrow{a}|$ = 12, 
$|\overrightarrow{b}|$ = 4$\sqrt{3}$, $\overrightarrow{b}.\overrightarrow{c}$ = 24, then the which of the following is(are) true?
\begin{enumerate}
\item $\frac{|c|^{2}}{2} - |\overrightarrow{a}| = 12$
\item $\frac{|c|^{2}}{2} + |\overrightarrow{a}| = 30$
\item $|\overrightarrow{a} \times \overrightarrow{b} + \overrightarrow{c} \times \overrightarrow{a}| = 48\sqrt{3}$
\item $\overrightarrow{a}$.$\overrightarrow{b}$ = -72
\end{enumerate}

\item Consider a pyramid OPQRS located in the first octant(x $\geq$ 0, y $\geq$ 0, z $\geq$ 0) with O as origin, and OP and OR along x-axis and along y-axis respectively. The base OPQR of the pyramid is a square with OP = 3. The point S is directly above the mid-point, T of diagonal OQ such that TS = 3. Then
\begin{enumerate}
\item the acute angle between OQ and OS is $\frac{\pi}{3}$
\item the equation of the plane containing the triangle OQS is x-y = 0
\item the length of the perpendicular from P to the plane containing the triangle OQS is $\frac{3}{\sqrt{2}}$
\item The perpendicular distance from O to the straight line containing RS is $\sqrt{\frac{15}{2}}$.
\end{enumerate}

\item Let $\hat{u}$ = $u_{1}i + u_{2}j + u_{3}k$ be a unit vector in $R^{3}$ and $\hat{w}$ = $\frac{1}{\sqrt{6}}$
($\hat{i}$ + $\hat{j}$ + 2$\hat{k}$). Given that there exists a vector $\overrightarrow{v}$ in following $R^{3}$ such that $|\overrightarrow{u} \times \overrightarrow{v}|$ = 1 and $\overrightarrow{w}(\overrightarrow{u} \times \overrightarrow{v})$ = 1. Which of the following statement(s) is(are) correct?
\begin{enumerate}
\item There is exactly one choice for such $\overrightarrow{v}$
\item There are infinitely many choices for such $\overrightarrow{v}$
\item If $\hat{u}$ lies in the xy-plane then $|u_1|$ = $|u_2|$
\item If $\hat{u}$ lies in the xz-plane then 2$|u_1|$ = $|u_3|$
\end{enumerate}

\item Let 
\begin{align*}
P_1: 2x + y - z = 3
\end{align*}
\begin{align*}
P_2: x + 2y + z = 2
\end{align*}
be two planes. Then, which of the following statement(s) is(are) correct?
\begin{enumerate}
\item The line of intersection of $P_1$ and $P_2$ has direction ratios 1, 2, -1
\item The line $\frac{3x - 4}{9} = \frac{1 - 3y}{9} = \frac{z}{3}$ is perpendicular to the line of intersection of $P_1$ and $P_2$.
\item The acute angle between $P_1$ and $P_2$ is $60^{\degree}$
\item If $P_3$ is the plane passing through the point(4, 2, -2) and perpendicular to the line of intersection of $P_1$ and $P_2$, then the distance of the point (2, 1, 1) from the plane $P_3$ is $\frac{2}{\sqrt{3}}$.
\end{enumerate}

\item Let $L_1$ and $L_2$ denote the lines 
\begin{align*}
\overrightarrow{r} = \hat{i} + \lambda(-\hat{i} + 2\hat{j} + 2\hat{k}), \lambda \in R
\end{align*}
\begin{align*}
\overrightarrow{r} = \mu(2\hat{i} - \hat{j} + 2\hat{k}), \mu \in R
\end{align*}
respectively. If $L_3$ is a line which is perpendicular to both $L_1$ and $L_2$ and cuts both of them, then which of the following options describe(s) $L_3$?
\begin{enumerate}
\item $\overrightarrow{r}$ = $\frac{2}{9}(4\hat{i} + \hat{j} + \hat{k}) + t(2\hat{i} + 2\hat{j} - \hat{k})$, t $\in$ R
\item $\overrightarrow{r}$ = $\frac{2}{9}(2\hat{i} - \hat{j} + 2\hat{k}) + t(2\hat{i} + 2\hat{j} - \hat{k})$, t $\in$ R
\item $\overrightarrow{r}$ = $t(2\hat{i} + 2\hat{j} - \hat{k})$, t $\in$ R
\item $\overrightarrow{r}$ = $\frac{1}{3}(2\hat{i} + \hat{k}) + t(2\hat{i} + 2\hat{j} - \hat{k})$, t $\in$ R
\end{enumerate}

\item Three lines 
\begin{align*}
L_1: \overrightarrow{r} = \lambda\hat{i}, \lambda \in R
\end{align*}
\begin{align*}
L_2: \overrightarrow{r} = \hat{k} + \mu\hat{j}, \mu \in R
\end{align*}
\begin{align*}
L_3: \overrightarrow{r} = \hat{i} + \hat{j} + v\hat{k}, v \in R
\end{align*}
are given. For which point(s) Q on $L_2$ can we find a point P on $L_1$ and a point R on $L_3$ so that P,Q and R are collinear?
\begin{enumerate}
\item $\hat{k}$ - $\frac{1}{2}\hat{j}$
\item $\hat{k}$
\item $\hat{k}$ + $\hat{j}$
\item $\hat{k}$ + $\frac{1}{2}\hat{j}$
\end{enumerate}

\textbf{(E). Subjective Problems}

\item From a point O inside a triangle ABC, perpendiculars OD, OE, OF are drawn to the sides BC, CA, AB respectively. Prove that the perpendiculars from A, B, C to the sides EF, FD, DE are concurrent.

\item $A_1$, $A_2$,............$A_n$ are the vertices of a regular plane polygon with n sides and O is its centre. Show that 
\begin{align*}
\sum_{i=1}^{n-1} (\overrightarrow{OA_{i}} \times \overrightarrow{OA_{i+1}}) = (1-n)(\overrightarrow{OA_{2}} \times \overrightarrow{OA_{1}})
\end{align*}

\item Find all values of $\lambda$ such that x, y, z $\neq$ (0, 0, 0) and 
\begin{align*}
(\overrightarrow{i} + \overrightarrow{j} + 3\overrightarrow{k})x + (3\overrightarrow{i} - 3\overrightarrow{j} + \overrightarrow{k})y + (-4\overrightarrow{i} + 5\overrightarrow{j})z\\
 = \lambda(x\overrightarrow{i} \times \overrightarrow{j}y + \overrightarrow{k}z)
\end{align*}
where $\overrightarrow{i}$, $\overrightarrow{j}$, $\overrightarrow{k}$ are unit vectors along the coordinate axes.

\item A vector $\overrightarrow{A}$ has components $\overrightarrow{A_1}$, $\overrightarrow{A_2}$, $\overrightarrow{A_3}$ ina right-handed rectangular Cartesian coordinate system oxyz. The coordinate system is rotated about the x-axis through an angle of $\frac{\pi}{2}$. Find the components of A in the new coordinate system, interms of $\overrightarrow{A_1}$, 
$\overrightarrow{A_2}$, $\overrightarrow{A_3}$.

\item The position vectors of the points A, B, C and D are (3$\hat{i}$ - 2$\hat{j}$ - $\hat{k}$), (2$\hat{i}$ + 3$\hat{j}$ - 4$\hat{k}$), (-$\hat{i}$ + $\hat{j}$ + 2$\hat{k}$) and (4$\hat{i}$ + 5$\hat{j}$ + $\lambda\hat{k}$) respectively. If the points A, B, C and D lie on a plane, find the value of $\lambda$?

\item If A, B, C, D are any four points in space, Prove that
\begin{align*}
|\overrightarrow{AB} \times \overrightarrow{CD} + \overrightarrow{BC} \times \overrightarrow{AD} + \overrightarrow{CA} \times \overrightarrow{BD}|=4
\end{align*}  
(area of triangle ABC)

\item Let OACB be a parallelgram with O at the origin and OC a diagonal. Let D be the mid-point of OA. Using vector methods prove that BD and CO intersect in the same ratio. Determine this ratio.

\item If vectors $\overrightarrow{a}$, $\overrightarrow{b}$, $\overrightarrow{c}$ are coplanar, show that
\begin{align*}
\begin{vmatrix}
\overrightarrow{a} & \overrightarrow{b} & \overrightarrow{c} \\ 
\overrightarrow{a} . \overrightarrow{a} & \overrightarrow{a} . \overrightarrow{b} & \overrightarrow{a} . \overrightarrow{c} \\ 
\overrightarrow{b} . \overrightarrow{a} & \overrightarrow{b} . \overrightarrow{b} & \overrightarrow{b} . \overrightarrow{c}
\end{vmatrix} = \overrightarrow{0}
\end{align*}

\item In a triangle OAB, E is the midpoint of BO and D is a point on AB such that AD : DB = 2 : 1. If OD and AE intersect at P, Determine the ratio OP : PD using vectors methods?

\item Let $\overrightarrow{A}$ = 2$\hat{i}$ + $\hat{k}$, $\overrightarrow{B}$ = $\hat{i}$ + $\hat{j}$ + $\hat{k}$, and 
$\overrightarrow{C}$ = 4$\hat{i}$ - 3$\hat{j}$ + 7$\hat{k}$. Determine a vector $\overrightarrow{R}$. Satisfying 
$\overrightarrow{R} \times \overrightarrow{B}$ = $\overrightarrow{C} \times \overrightarrow{B}$ and $\overrightarrow{R}.\overrightarrow{A}$ = 0

\item Determine the value of 'c' so that for all real x, the vector cx$\hat{i}$ - 6$\hat{j}$ - 3$\hat{k}$ and 
x$\hat{i}$ + 2$\hat{j}$ + 2cx$\hat{k}$ make an obtuse angle with each other.

\item In a triangle ABC, D and E are points on BC and AC respectively, such that BD = 2DC and AE = 3EC. Let P be the point of intersection of AD and BE. Find BP/PE using vector methods.

\item If the vectors $\overrightarrow{b}$, $\overrightarrow{c}$, $\overrightarrow{d}$ are not coplanar, then prove that the vector
\begin{align*}
(\overrightarrow{a} \times \overrightarrow{b}) \times (\overrightarrow{c} \times \overrightarrow{d}) + (\overrightarrow{a} \times \overrightarrow{c}) \times (\overrightarrow{d} \times \overrightarrow{b})\\ + 
(\overrightarrow{a} \times \overrightarrow{d}) \times (\overrightarrow{b} \times \overrightarrow{c})
\end{align*}
is parallel to $\overrightarrow{a}$.

\item The position vectors of the vertices A, B, C of a tetrahedron ABCD are $\hat{i}$ + $\hat{j}$ + $\hat{k}$, $\hat{i}$ and 3$\hat{i}$ respectively. The altitude from vertex D to the opposite face ABC meets the median line through A of the triangle ABC at a point E. If the length of the side AD is 4 and the volume of tetrahedron is $\frac{2\sqrt{2}}{3}$. find the position vector of the point E for all its possible positions.

\item If A, B and C vectors such that $|B|$ = $|C|$, Prove that
\begin{align*}
[(A+B) \times (A+C)] \times (B \times C)(B + C) = 0.
\end{align*}

\item Prove, by vector methods or otherwise, that the point of intersection of the diagonals of a trapezium lies on the line passing through the mid-points of the parallel sides. (You may assume that the trapezium is not a parallelgram.)

\item For any two vectors u and v, prove that
\begin{enumerate}
\item $(u.v)^{2} + |u \times v|^{2} = |u|^{2}|v|^{2}$
\item $(1 + |u|^{2})(1 + |v|^{2}) = (1 - u.v)^{2} + |u + v + (u \times v)|^{2}$
\end{enumerate}

\item Let u and v be unit vectors. If w is a vector such that w + (w $\times$ u) = v, then prove that $|u \times v|.w \leq 1/2$ and that the equality holds if and onlt if u is perpendicular to v.

\item Show, by vector methods, that the angular bisectors of a triangle are concurrent and find an expression for the position vector of the point concurrency in terms of the position vectors of the vertices.

\item Find 3-dimensional vectors $\overrightarrow{v_1}$, $\overrightarrow{v_2}$, $\overrightarrow{v_3}$ satisfying
\begin{align*}
\overrightarrow{v_1}.\overrightarrow{v_1} = 4, \overrightarrow{v_1}.\overrightarrow{v_2} = -2, \overrightarrow{v_1}.\overrightarrow{v_3} = 6, 
\end{align*}
\begin{align*}
\overrightarrow{v_2}.\overrightarrow{v_2} = 2, \overrightarrow{v_2}.\overrightarrow{v_3} = -5, \overrightarrow{v_3}.\overrightarrow{v_3} = 29
\end{align*}

\item Let 
\begin{align*}
\overrightarrow{A(t)} = f_1(t)\hat{i} + f_2(t)\hat{j}
\end{align*}
\begin{align*}
\overrightarrow{B(t)} = g_1(t)\hat{i} + g_2(t)\hat{j}, t \in [0,1]
\end{align*}
where $f_1$, $f_2$, $g_1$, $g_2$ are continuous functions. If $\overrightarrow{A(t)}$ and $\overrightarrow{B(t)}$ are non-zero vectors for all t and $\overrightarrow{A(0)}$ = 2$\hat{i}$ + 3$\hat{j}$, $\overrightarrow{A(1)}$ = 6$\hat{i}$ + 2$\hat{j}$, $\overrightarrow{B(0)}$ = 3$\hat{i}$ + 2$\hat{j}$, 
$\overrightarrow{B(1)}$ = 2$\hat{i}$ + 6$\hat{j}$. Then show that $\overrightarrow{A(t)}$ and $\overrightarrow{B(t)}$ are parallel for some t.

\item Let V be the volume of the parallelopiped formed by the vectors
\begin{align*}
\overrightarrow{a} = a_1\hat{i} + a_2\hat{j} + a_3\hat{k}
\end{align*}
\begin{align*}
\overrightarrow{b} = b_1\hat{i} + b_2\hat{j} + b_3\hat{k}
\end{align*}
\begin{align*}
\overrightarrow{c} = c_1\hat{i} + c_2\hat{j} + c_3\hat{k}
\end{align*}
If $a_r$, $b_r$, $c_r$ where r = 1, 2, 3, are non-negative real numbers and 
\begin{align*}
\sum_{r=1}^{3}(a_r + b_r + c_r)=3L
\end{align*}
Show that $V \leq L^{3}$.

\item 
\begin{enumerate}
\item Find the equation of the plane passing through the points (2, 1, 0), (5, 0, 1) and (4, 1, 1).
\item If P is the point (2, 1, 6) then find the point Q such that PQ is perpendicular to the plane in (i) and the midpoint of PQ lies on it.
\end{enumerate}

\item If $\overrightarrow{u}$, $\overrightarrow{v}$, $\overrightarrow{w}$ are three non-coplanar unit vectors and $\alpha$, $\beta$, $\gamma$ are the angles between $\overrightarrow{u}$ and $\overrightarrow{v}$ and $\overrightarrow{w}$, $\overrightarrow{w}$ and $\overrightarrow{u}$ respectively and $\overrightarrow{x}$, $\overrightarrow{y}$, $\overrightarrow{z}$ are unit vectors along the bisection of the angles $\alpha$, $\beta$, $\gamma$ respectively. Prove that
\begin{align*}
[(\overrightarrow{x} \times \overrightarrow{y})  (\overrightarrow{y} \times \overrightarrow{z}) (\overrightarrow{z} \times \overrightarrow{x})] =\\
\frac{1}{16}[\overrightarrow{u}\overrightarrow{v}\overrightarrow{w}]^{2} \sec^{2}\frac{\alpha}{2}\sec^{2}\frac{\beta}{2}\sec^{2}\frac{\gamma}{2}.
\end{align*}

\item If $\overrightarrow{a}$, $\overrightarrow{b}$, $\overrightarrow{c}$ and $\overrightarrow{d}$ are distinct vectors such that $\overrightarrow{a} \times \overrightarrow{c}$ = $\overrightarrow{b} \times 
\overrightarrow{d}$ and $\overrightarrow{a} \times \overrightarrow{b}$ = $\overrightarrow{c} \times 
\overrightarrow{d}$. Prove that
\begin{align*}
(\overrightarrow{a} - \overrightarrow{d}).(\overrightarrow{b} - \overrightarrow{c}) \neq 0 \\
(i.e, \overrightarrow{a}.\overrightarrow{b}+\overrightarrow{d}.\overrightarrow{c} \neq \overrightarrow{d}.\overrightarrow{b}+\overrightarrow{a}.\overrightarrow{c})
\end{align*}

\item Find the equation of the plane passing through (1, 1, 1) and parallel to the lines $L_1$, $L_2$ having direction ratios (1, 0, -1), (1, -1, 0). Find the volume of tetrahedron formed by origin and the points where these planes intersect the coordinate axes.

\item A parallelopiped $'S'$ has base points A, B, C and D and upper face points $A'$, $B'$, $C'$, $D'$. This parallelopiped is compressed by upper face $A'B'C'D'$ to form a new parallelopiped $'T'$ having upper face points $A''$,$B''$,$C''$,$D''$. Volume of parallelopiped T is 90 perecnt of the volume of the parallelopiped S. Prove that the locus of a $A''$, is a plane.

\item $P_1$ and $P_2$ are planes passing through origin. $L_1$ and $L_2$ are two lines on $P_1$ and $P_2$ respectively such that their intersection is origin. Show that there exists points A, B, C whose permutation $A'$, $B'$, $C'$ can be chosen such that
\begin{enumerate}
\item A is on $L_1$, B on $P_1$ but not on $L_1$ and C not on $P_1$
\item $A'$ is on $L_2$, $B'$ on $P_2$ but not on $L_2$ and $C'$ not on $P_2$
\end{enumerate} 

\item Find the equation of the plane containing the line
\begin{align*}
2x - y + z - 3 = 0
\end{align*}
\begin{align*}
3x + y + z = 5
\end{align*}
 and at a distance of $\frac{1}{\sqrt{6}}$ from the point (2, 1, -1).
 
\item If the incident ray on a surface is along the unit vector $\hat{v}$ the reflected ray is along the unit vector 
 $\hat{w}$ and the normal is along unit vector $\hat{a}$ outwards. Express $\hat{w}$ in terms of $\hat{a}$ and $\hat{v}$.

\textbf{(F). Match the following}

\item Match the following:
\begin{table}[ht!]
\centering
\begin{tabular}{c c} 
 \textbf{Column I} & \textbf{Column II}\\ [0.5ex] 
 (A) Two rays x+y=$|a|$ and ax-y=1
  intersects each other\\ in the first
  quadrant in the interval a 
  $\in$ ($a_0$, $\infty$), \\
  the value of $a_0$ is                                               &(p) 2\\ 
 (B) Point $(\alpha,\beta,\gamma)$ lies on the plane x+y+z=2. 
 Let\\ $\overrightarrow{a}=\alpha\hat{i}+\beta\hat{j}+\gamma\hat{k}$, 
 $\hat{k} \times (\hat{k} \times \hat{a})$=0, then $\gamma$=          &(q) $\frac{4}{3}$\\
 (C) $|\int_{0}^{1}(1-y^{2})dy|$ 
 + $|\int_{1}^{0}(y^{2}-1)dy|$                                        &(r) $|\int_{0}^{1}\sqrt{1-x}dx|$
                                                                           +$|\int_{-1}^{0}\sqrt{1+x}dx|$\\
 (D) If $\sin A \sin B \sin C$ + $\cos A \cos B$ = 1,\\ 
 then value of $\sin C$=                                              &(s) 1\\[1ex] 
\end{tabular}
\end{table}\\

\item Match the statements/expressions in \textbf{Column-I} with the values given in \textbf{Column-II}.
\begin{table}[ht!]
\centering
\begin{tabular}{c c} 
 \textbf{Column I} & \textbf{Column II}\\ [0.5ex] 
 (A) Roots of the equation
     $2\sin^{2}\theta + \sin^{2}2\theta$                    &(p) $\frac{\pi}{6}$\\ 
 (B) Points of discontinuity of the unction
     f(x)=$[\frac{6x}{\pi}]\cos[\frac{3x}{\pi}]$            &(q) $\frac{\pi}{4}$\\
 (C) Volume of the parallelopiped with its edges\\
     represented by the vectors
     $\hat{i}+\hat{j},\hat{i}+2\hat{j}$ 
     and $\hat{i}+\hat{j}+\pi\hat{k}$                       &(r) $\frac{\pi}{3}$\\
 (D) Angle between vector $\overrightarrow{a}$
     and $\overrightarrow{b}$ where $\overrightarrow{a}$, 
     $\overrightarrow{b}$ and $\overrightarrow{c}$ are\\ 
     unit vectors satisfying $\overrightarrow{a}  
     + \overrightarrow{b} + \sqrt{3}\overrightarrow{c}=0$   &(s) $\frac{\pi}{2}$\\[1ex]
                                                            
\end{tabular}
\end{table}

\clearpage
\item Consider the following linear equations
\begin{align*}
ax+by+cz=0
\end{align*}
\begin{align*}
bx+cy+az=0
\end{align*}
\begin{align*}
cx+ay+bz=0
\end{align*}
Match the conditions/expressions in \textbf{Column-I} with statements in \textbf{Column-II} and indicate your answer by darkening the bubbles in the 4 $\times$ 4 matrix given in the $ORS$.
\begin{table}[ht!]
\centering
\begin{tabular}{c c} 
 \textbf{Column I} & \textbf{Column II}\\ [0.5ex] 
 (A) a+b+c $\neq$ 0 and\\
     $a^2+b^2+c^2$=ab+bc+ca                             &(p) the equation represents planes meeting 
                                                             only at single point\\ 
 (B) a+b+c=0 and\\
  $a^2+b^2+c^2$ $\neq$ ab+bc+ca                         &(q) the equation represents the line x=y=z\\
 (C) a+b+c $\neq$ 0 and\\
  $a^2+b^2+c^2$ $\neq$ ab+bc+ca                         &(r) the equation represents identical planes\\
 (D) a+b+c=0 and\\
  $a^2+b^2+c^2$=ab+bc+ca                                &(s) the equation represents the whole of the 
                                                             3 dimensional space\\[1ex] 
\end{tabular}
\end{table}

\item Mtach the statements/expressions given in \textbf{Column-I} with the values in given in \textbf{Column-II}.
\begin{table}[ht!]
\centering
\begin{tabular}{c c} 
 \textbf{Column I} & \textbf{Column II}\\ [0.5ex] 
 (A) The number of solutions of the equation
     $xe^{\sin x}-\cos x=0$\\
     in the interval $(0, \frac{\pi}{2})$                   &(p) 1\\ 
 (B) Values of k for which the planes
      kx+4y+z=0, 4x+ky+2z=0\\ and 2x+2y+z=0
      intersect in a straight line                          &(q) 2\\
 (C) Values of k for which
     $|x-1|+|x-2|+|x+1||x+2|=4k$\\
     has integer solutions                                  &(r) 3\\
 (D) If $y'=y+1$ and y(0)=1,
     then values of y(ln2)                                  &(s) 4\\[1ex]
                                                            
\end{tabular}
\end{table}

\item Mtach the statement in \textbf{Column-I} with the values in given in \textbf{Column-II}.
\begin{table}[ht!]
\centering
\begin{tabular}{c c} 
 \textbf{Column I} & \textbf{Column II}\\ [0.5ex] 
 (A) A line from the origin meets the lines\\
     $\frac{x-2}{1}=\frac{y-1}{-2}=\frac{z+1}{1}$ and
     $\frac{x-8/3}{2}=\frac{y+3}{-1}=\frac{z-1}{1}$ at
     P and Q\\ respectively. If length PQ=d, then $d^2$                                  &(p) -4\\ 
 (B) The values of x for
      $\tan^{-1}(x+3)-\tan^{-1}(x-3)$\\=$\sin^{-1}(\frac{3}{5})$ are                     &(q) 0\\
 (C) Non-zero vectors $\overrightarrow{a}, \overrightarrow{b}, 
     \overrightarrow{c}$ satisfy $\overrightarrow{a}.\overrightarrow{b}=0$\\
     $(\overrightarrow{b}-\overrightarrow{a}).(\overrightarrow{b}+\overrightarrow{c})$ and 
     $2|\overrightarrow{b}+\overrightarrow{c}|=|\overrightarrow{b}-\overrightarrow{a}|$.\\ 
     If $\overrightarrow{a}=\mu\overrightarrow{b}+4\overrightarrow{c}$,
     then the possible values of $\mu$ are                                               &(r) 4\\
 (D) Let f be the function in $[-\pi, \pi]$ given by f(0)=9\\
     and $f(x)=\sin(\frac{9x}{2})/\sin(\frac{x}{2})$ for $x \neq 0$\\
     The value of $\frac{2}{\pi}\int_{\pi}^{\pi}f(x)dx$ is                               &(s) 5\\[1ex]
                                                            
\end{tabular}
\end{table}

\clearpage
\item Mtach the statement in \textbf{Column-I} with the values in given in \textbf{Column-II}.
\begin{table}[ht!]
\centering
\begin{tabular}{c c} 
 \textbf{Column I} & \textbf{Column II}\\ [0.5ex] 
 (A) If $\overrightarrow{a}=\hat{j}+\sqrt{3}\hat{k}, \overrightarrow{b}
     =-\hat{j}+\sqrt{3}\hat{k}$ and $\overrightarrow{c}=2\sqrt{3}\hat{k}$
     form a triangle,\\ then the internal angle of the triangle between
     $\overrightarrow{a}$ and $\overrightarrow{b}$ is                             &(p) $\frac{\pi}{6}$\\ 
 (B) If $\int_{a}^{b}(f(x)-3x)dx=a^2-b^2$, then the value of $f(\frac{\pi}{6})$   &(q) $\frac{2\pi}{3}$\\
 (C) The value of $\frac{\pi^{2}}{ln3}\int_{7/6}^{5/6}\sec(\pi x)dx$ is           &(r) $\frac{\pi}{3}$\\
 (D) The maximum value of $|Arg(\frac{1}{1-z})|$ for $|z|$=1, 
     $z \neq 1$ is given by                                                       &(s) $\pi$\\[1ex] 

\end{tabular}
\end{table}

\item Mtach the List-I with List-II and select the correct answer using the code given the below lists..
\begin{table}[ht!]
\centering
\begin{tabular}{c c} 
 \textbf{Column I} & \textbf{Column II}\\ [0.5ex] 
 (A) Volume of parallelopiped determined by the vectors $\overrightarrow{a}$
     and $\overrightarrow{b}$ and $\overrightarrow{c}$ is 2.\\Then the volume 
     the parallelopiped determined by the vectors\\
     $2(\overrightarrow{a}+\overrightarrow{b})$, $3(\overrightarrow{b}+\overrightarrow{c})$
     and $2(\overrightarrow{c}+\overrightarrow{a})$ is                               &(p) 100\\ 
 (B) Volume of parallelopiped determined by the vectors $\overrightarrow{a}$
     and $\overrightarrow{b}$ and $\overrightarrow{c}$ is 5.\\Then the volume 
     the parallelopiped determined by the vectors\\
     $3(\overrightarrow{a}+\overrightarrow{b})$, $3(\overrightarrow{b}+\overrightarrow{c})$
     and $2(\overrightarrow{c}+\overrightarrow{a})$ is                               &(p) 30\\ 
 (C) Area of the triangle with adjacent sides determined by the vectors\\ 
     $\overrightarrow{a}$ and $\overrightarrow{b}$ is 20. Then the area of the 
     triangle with adjacent sides determined\\ by vectors $(2\overrightarrow{a}
     +3\overrightarrow{b})$ and $\overrightarrow{a}-\overrightarrow{b}$ is           &(r) 24\\
 (D) Area of the parallelgram with adjacent sides determined by the vectors\\ 
     $\overrightarrow{a}$ and $\overrightarrow{b}$ is 30. Then the area of the 
     parallelgram with adjacent sides determined\\ by vectors $(\overrightarrow{a}
     +\overrightarrow{b})$ and $\overrightarrow{a}$ is                              &(s) 60\\[1ex]
     
\textbf{codes:}
\begin{tabular}{ c c c c c}
      P & Q & R & S\\
  (a) 4 & 2 & 3 & 1\\
  (b) 2 & 3 & 1 & 4\\
  (c) 3 & 4 & 1 & 2\\
  (d) 1 & 4 & 3 & 2\\
\end{tabular}
\end{tabular}
\end{table} 


\item Mtach the statements/expressions given in \textbf{Column-I} with the values in given in \textbf{Column-II}.
\begin{table}[ht!]
\centering
\begin{tabular}{c c} 
 \textbf{Column I} & \textbf{Column II}\\ [0.5ex] 
 (A) In $R^2$, if the magnitude of the projection vector of the vector\\
     $\alpha\hat{i}+\beta\hat{j}$ on $\sqrt{3}\hat{i}+\hat{j}$ is $\sqrt{3}$
     and if $\alpha =2+\sqrt{3}\beta$, then possible\\ value of $|\alpha|$ is                   &(p) 1\\ 
 (B) Let a and b real numbers such that the function\\
     $f(x)=-3ax^2-2, x<1$ and $f(x)=bx+a^2, x \geq 1$ if differentiable for 
     all $x \in R$.\\Then possible value of a is(are)                                           &(q) 2\\
 (C) Let $\omega \neq 1$ be a complex cube root of unity. If\\
     $(3-3\omega+2\omega^2)^{4n+3}+(2+3\omega-3\omega^2)^{4n+3}+
     (-3+2\omega+3\omega^2)^{4n+3}=0$\\ then possible value(s) of n is(are)                     &(r) 3\\
 (D) Let the harmonic mean of two positive real numbers a and b be 4.\\
     If q is a positive real number such that a,5,q,b is an arithmetic\\
     progression, then the value(s) of $|q-a|$ is(are)                                          &(s) 4\\[1ex]
                                                            
\end{tabular}
\end{table}
\clearpage

\item Consider the lines
\begin{align*}
L_1: \frac{x-1}{2}=\frac{y}{-1}=\frac{z+3}{1}
\end{align*}
\begin{align*}
L_2: \frac{x-4}{1}=\frac{y+3}{1}=\frac{z+3}{2}
\end{align*}
and the planes
$P_1: 7x+y+2z=3$, $P_2: 3x+5y-6z=4$. Let ax+by+cz=d be the equation of the plane passing through the point of intersection of lines $L_1$ and $L_2$ and perpendicular to the planes $P_1$ and $P_2$.
\begin{table}[ht!]
\centering
\begin{tabular}{c c} 
 \textbf{Column I} & \textbf{Column II}\\ [0.5ex] 
 (P) a=                                                       &(1) 13\\ 
 (Q) b=                                                       &(2) -3\\
 (R) c=                                                       &(3) 1\\
 (S) d=                                                       &(4) -2\\[1ex] 
 \textbf{codes:}
\begin{tabular}{ c c c c c}
      P & Q & R & S\\
  (a) 3 & 2 & 4 & 1\\
  (b) 1 & 3 & 4 & 2\\
  (c) 3 & 2 & 1 & 4\\
  (d) 2 & 4 & 1 & 3\\
\end{tabular}

\end{tabular}
\end{table}

\item Mtach the statements/expressions given in \textbf{Column-I} with the values in given in \textbf{Column-II}.
\begin{table}[ht!]
\centering
\begin{tabular}{c c} 
 \textbf{Column I} & \textbf{Column II}\\ [0.5ex] 
 (A) In a triangle $\Delta XYZ$, let a,b,c be the lengths of the sides opposite\\
     to the angles X,Y,Z respectively. If $2(a^2-b^2)=c^2$ and $\lambda=\frac{\sin(X-Y)}{\sin Z}$\\
     then possible values of n for which $\cos(n\pi\lambda)=0$ is(are)                              &(p) 1\\ 
 (B) In a triangle $\Delta XYZ$, let a,b,c be the lengths of the sides opposite\\
     to the angles X,Y,Z respectively. If $1+\cos2X-2\cos2Y=2\sin X$\\
     then possible values of $\frac{a}{b}$ is(are)                                                  &(q) 2\\
 (C) In a $R^2$ let $\sqrt{3}\hat{i}+\hat{j}, \hat{i}+\sqrt{3}\hat{j}$ and $\beta\hat{i}
     +(1-\beta)\hat{j}$ be the position vectors of X,Y\\ and Z w.r.t. to the origin O respectively.
     If the distance Z from\\ the bisector of the acute angle of $\overrightarrow{OX}$ with
     $\overrightarrow{OY}$ is $\frac{3}{\sqrt{2}}$, then possible\\ values of $|\beta|$ is(are)     &(r) 3\\
 (D) Suppose that $F(\alpha)$ denotes the area of the region bounded by x=0,\\
     x=2, $y^2=4x$ and $y=|\alpha x-1|+|\alpha x-2|+\alpha x$, where $\alpha \in (0,1)$\\
     Then the value(s) of $F(\alpha)+\frac{8}{3}\sqrt{2}$, when $\alpha =0, 1$ is(are)              &(s) 5\\[1ex]
                                                            
\end{tabular}
\end{table}
\clearpage

\textbf{(G). Comprehension Based Questions:}

Consider the lines
\begin{align*}
L_1: \frac{x + 1}{3} = \frac{y + 2}{1} = \frac{z + 1}{2}
\end{align*}
\begin{align*}
L_2: \frac{x - 2}{1}=\frac{y + 2}{2} = \frac{z - 3}{3}
\end{align*}
\item The unit vector perpendicular to both $L_1$ and $L_2$ is
\begin{enumerate}
\item $\frac{-\hat{i} + 7\hat{j} + 7\hat{k}}{\sqrt{99}}$
\item $\frac{-\hat{i} - 7\hat{j} + 5\hat{k}}{5\sqrt{3}}$
\item $\frac{-\hat{i} + 7\hat{j} + 5\hat{k}}{5\sqrt{3}}$
\item $\frac{7\hat{i} - 7\hat{j} - \hat{k}}{\sqrt{99}}$
\end{enumerate}

\item The shortest distance between $L_1$ and $L_2$ is
\begin{enumerate}
\item 0
\item $\frac{17}{\sqrt{3}}$
\item $\frac{41}{5\sqrt{3}}$
\item $\frac{17}{5\sqrt{3}}$
\end{enumerate}

\item The distance of the point (1, 1, 1) from the plane passing through the point (-1, -2, -1) and whose normal is perpendicular to both the lines $L_1$ and $L_2$ is
\begin{enumerate}
\item $\frac{2}{\sqrt{75}}$
\item $\frac{7}{\sqrt{75}}$
\item $\frac{13}{\sqrt{75}}$
\item $\frac{23}{\sqrt{75}}$
\end{enumerate}

\textbf{(H). Assertion and Reason Type Questions}

\item Consider the planes 3x - 6y - 2z = 15 and 2x + y - 2z = 5.\\
\textbf{STATEMENT-1}: The parametric equations of the line of intersection of the given planes are x = 3 + 14t, y = 1 + 2t, z = 15t.\\
\textbf{STATEMENT-2}: The vector $14\hat{i} + 2\hat{j} + 15\hat{k}$ is parallel to the line of intersection of given planes.
\begin{enumerate}
\item Statement-1 is true, Statement-2 is true; Statement-2 is not a correct explanation for Statement-1
\item Statement-1 is true, Statement-2 is false
\item Statement-1 is false, Statement-2 is true
\item Statement-1 is true, Statement-2 is true; Statement-2 is a correct explanation for Statement-1
\end{enumerate}

\item Let the vectors $\overrightarrow{PQ}$, $\overrightarrow{QR}$, $\overrightarrow{RS}$,$\overrightarrow{ST}$, $\overrightarrow{TU}$ and $\overrightarrow{UP}$ represent the sides of a regular hexagon.\\
\textbf{STATEMENT-1}: $\overrightarrow{PQ} \times (\overrightarrow{RS} + \overrightarrow{ST} \neq \overrightarrow{0}$\\
\textbf{STATEMENT-2}: $\overrightarrow{PQ} \times \overrightarrow{RS} = \overrightarrow{0}$ and $\overrightarrow{PQ} \times \overrightarrow{ST} \neq \overrightarrow{0}$.
\begin{enumerate}
\item Statement-1 is true, Statement-2 is true; Statement-2 is not a correct explanation for Statement-1
\item Statement-1 is true, Statement-2 is false
\item Statement-1 is false, Statement-2 is true
\item Statement-1 is true, Statement-2 is true; Statement-2 is a correct explanation for Statement-1
\end{enumerate}

\item Consider three planes
\begin{align*}
P_1: x - y + z = 1
\end{align*}
\begin{align*}
P_2: x + y - z = 1
\end{align*}
\begin{align*}
P_3: x - 3y + 3z = 2
\end{align*}
Let $L_1$, $L_2$, $L_3$ be the lines of intersection of the planes $P_2$ and $P_3$, $P_3$ and $P_1$, $P_1$ and $P_2$ respectively.\\
\textbf{STATEMENT-1}: At least two of the lines $L_1$, $L_2$, $L_3$ are non-parallel\\
\textbf{STATEMENT-2}: The three planes does not have a common point.
\begin{enumerate}
\item Statement-1 is true, Statement-2 is true; Statement-2 is not a correct explanation for Statement-1
\item Statement-1 is true, Statement-2 is false
\item Statement-1 is false, Statement-2 is true
\item Statement-1 is true, Statement-2 is true; Statement-2 is a correct explanation for Statement-1
\end{enumerate}
 
\textbf{(I). Integer Value Correct Type:}

\item If $\overrightarrow{a}$ and $\overrightarrow{b}$ are vectors in space given by 
\begin{align*}
\overrightarrow{a} = \frac{\hat{i} - 2\hat{j}}{\sqrt{5}}
\end{align*}
\begin{align*}
\overrightarrow{b} = \frac{2\hat{i} + \hat{j} + 3\hat{k}}{\sqrt{14}}
\end{align*}
then find the value of (2$\overrightarrow{a}$ + $\overrightarrow{b}$).[$(\overrightarrow{a} \times \overrightarrow{b}) \times (\overrightarrow{a} - 2\overrightarrow{b})$].

\item If the distance between the plane Ax - 2y + z = d and the plane containing the lines
\begin{align*}
\frac{x-1}{2} = \frac{y-2}{3} = \frac{z-3}{4}
\end{align*}
and
\begin{align*}
\frac{x-2}{3} = \frac{y-3}{4} = \frac{z-4}{5}
\end{align*}
is $\sqrt{6}$, then find $|d|$.

\item Let $\overrightarrow{a}$ = -$\overrightarrow{i}$-$\overrightarrow{k}$, $\overrightarrow{b}$ = -$\overrightarrow{i}$+$\overrightarrow{j}$ and $\overrightarrow{c}$ = $\overrightarrow{i}$+2$\overrightarrow{j}$+3$\overrightarrow{k}$ be three given vectors. If $\overrightarrow{r}$ is a vector such that $\overrightarrow{r} \times \overrightarrow{b} = \overrightarrow{c} \times \overrightarrow{b}$ and $\overrightarrow{r}.\overrightarrow{a}$ = 0, then the value of $\overrightarrow{r}.\overrightarrow{b}$ is

\item If $\overrightarrow{a}$, $\overrightarrow{b}$ and $\overrightarrow{c}$ are unit vectors satisfying
\begin{align*}
|\overrightarrow{a}-\overrightarrow{b}|^{2} + |\overrightarrow{b}-\overrightarrow{c}|^{2} + |\overrightarrow{c}-\overrightarrow{a}|^{2} = 9,
\end{align*}
then $|2\overrightarrow{a} + 5\overrightarrow{b} + 5\overrightarrow{c}|$ is

\item Consider the set of eight vectors
\begin{align*}
V = \{a\hat{i} + b\hat{j} + c\hat{k}:a, b, c \in \{-1, 1\}\}
\end{align*}
Three non-copolanar vectors can be chosen from V in $2^{p}$ ways. Then p is 

\item A pack contains n cards numbered from 1 to n. Two consecutive numbered cards are removed from the pack and the sum of the numbers on the remaining cards is 1224. If the smaller of the numbers on the removed cards is k, then k - 20 = 

\item Let $\overrightarrow{a}$, $\overrightarrow{b}$ and $\overrightarrow{c}$ be three non-copolar unit vectors such that the angle between every pair of them is $\frac{\pi}{3}$. If 
\begin{align*}
(\overrightarrow{a} \times \overrightarrow{b}) + (\overrightarrow{b} \times \overrightarrow{c}) = p\overrightarrow{a} + q\overrightarrow{b} + r\overrightarrow{c},
\end{align*}
where p, q, r are scalars, then the value of $\frac{p^{2} + 2q^{2} + r^{2}}{q^{2}}$ is

\item Suppose that $\overrightarrow{p}$, $\overrightarrow{q}$ and $\overrightarrow{r}$ are three non-coplanar vectors in $R^{3}$. Let the components of a vector $\overrightarrow{s}$ along $\overrightarrow{p}$, $\overrightarrow{q}$ and 
$\overrightarrow{r}$ be 4, 3 and 5, respectively. If the components of this vector $\overrightarrow{s}$ along 
(-$\overrightarrow{p}$ + $\overrightarrow{q}$ + $\overrightarrow{r}$), ($\overrightarrow{p}$ - $\overrightarrow{q}$ + $\overrightarrow{r}$) and (-$\overrightarrow{p}$ - $\overrightarrow{q}$ + $\overrightarrow{r}$) are x, y and z, respectively. then the value of 2x+y+z is

\item Let $\overrightarrow{a}$ and $\overrightarrow{b}$ be two unit vectors such that $\overrightarrow{a}$.$\overrightarrow{b}$=0. For some x, y $\in$ R, let 
\begin{align*}
\overrightarrow{c} = x\overrightarrow{a} + y\overrightarrow{b} + (\overrightarrow{a} \times \overrightarrow{b})
\end{align*}
If $|\overrightarrow{c}|$ = 2 and the vector $\overrightarrow{c}$ is inclined at the same angle $\alpha$ to both $\overrightarrow{a}$ and $\overrightarrow{b}$, then the value of 8$\cos^{2}\alpha$ is

\item Let P be a point in the first octant, whose image Q in the plane x + y = 3(that is the line segment PQ is perpendicular to the plane x + y = 3 and mid-point of PQ lies in the plane x + y = 3) lies on the z-axis. Let the distance of P from the x-axis be 5. If R is the image of P in the xy-plane, then the length of PR is

\item Consider the cube in the first octant with sides OP, OQ, OR of length 1, along the x-axis, y-axis, z-axis respectively, where O(0, 0, 0) is the origin. Let S($\frac{1}{2}$, $\frac{1}{2}$, $\frac{1}{2}$) be the centre of the cube and T be the vertex of the cube opposite to the origin O such that S lies on the diagonal OT. If 
$\overrightarrow{p}$ = $\overrightarrow{SP}$, $\overrightarrow{q}$ = $\overrightarrow{SQ}$, $\overrightarrow{r}$ = $\overrightarrow{SR}$ and $\overrightarrow{t}$ = $\overrightarrow{ST}$, then the value of $|(\overrightarrow{p} \times \overrightarrow{q}) \times (\overrightarrow{r} \times \overrightarrow{t})|$ is

\item Three lines are given by $\overrightarrow{r} = \lambda\hat{i}$, $\lambda$ $\in$ R; $\overrightarrow{r}$ = $\mu$($\hat{i} + \hat{j}$), $\mu \in R$ and $\overrightarrow{r} = v(\hat{i}+\hat{j}+\hat{k})$, v $\in$ R. Let the lines cut the plane x+y+z=1 at the points A, B, C respectively. If the area of the triangle ABC is $\Delta$ then the value of $(6\Delta)^{2}$ equals

\item Let $\overrightarrow{a} = 2\hat{i} + \hat{j} - \hat{k}$ and $\overrightarrow{b} = \hat{i} + 2\hat{j} + \hat{k}$ be two vectors. Consider a vector $\overrightarrow{c} = \alpha\hat{a} + \beta\hat{b}$, $\alpha, \beta \in R$. If the projection of $\overrightarrow{c}$ on the vector $(\overrightarrow{a} + \overrightarrow{b})$ is 3$\sqrt{2}$, then the minimum value of $(\overrightarrow{c}-(\overrightarrow{a} \times \overrightarrow{b})).\overrightarrow{c}$ equals

\textbf{Section-B}

\item A plane which passes through the point (3, 2, 0) and the line
\begin{align*}
\frac{x-4}{1} = \frac{y-7}{5} = \frac{z-4}{4}
\end{align*}
is
\begin{enumerate}
\item x - y + z = 1
\item x + y + z = 5
\item x + 2y -z = 1
\item 2x - y + z = 5
\end{enumerate}

\item If $|\overrightarrow{a}|$ = 4, $|\overrightarrow{b}|$ = 2 and the angle between $\overrightarrow{a}$ and $\overrightarrow{b}$ is $\frac{\pi}{6}$, then $(\overrightarrow{a} \times \overrightarrow{b})^{2}$ is equal to
\begin{enumerate}
\item 48
\item 16
\item $\overrightarrow{a}$
\item none of these
\end{enumerate}

\item If $\overrightarrow{a}$, $\overrightarrow{b}$, $\overrightarrow{c}$ are vectors show that $\overrightarrow{a} + \overrightarrow{b} + \overrightarrow{c}$ = 0 and $|\overrightarrow{a}|$ = 7, $|\overrightarrow{b}|$ = 5, 
$|\overrightarrow{c}|$ = 3 then angle between vector $\overrightarrow{b}$ and $\overrightarrow{c}$ is
\begin{enumerate}
\item $60^{o}$
\item $30^{o}$
\item $45^{o}$
\item $90^{o}$
\end{enumerate}

\item If $|\overrightarrow{a}|$ = 5, $|\overrightarrow{b}|$ = 4, $|\overrightarrow{c}|$ = 3 thus what will be the value of 
$|\overrightarrow{a}.\overrightarrow{b} + \overrightarrow{b}.\overrightarrow{c} + \overrightarrow{c}.\overrightarrow{a}|$, given that $\overrightarrow{a} + \overrightarrow{b} + \overrightarrow{c}$ = 0
\begin{enumerate}
\item 25
\item 50
\item -25
\item -50
\end{enumerate}

\item If the vectors $\overrightarrow{c}$, $\overrightarrow{a} = x\hat{i} + y\hat{j} + z\hat{k}$ and $\overrightarrow{b} = \hat{j}$ are such that $\overrightarrow{a}$, $\overrightarrow{b}$ and $\overrightarrow{c}$ form a right handed system then $\overrightarrow{c}$ is
\begin{enumerate}
\item $z\hat{i} - x\hat{k}$
\item 0
\item y$\hat{j}$
\item $-z\hat{i} + x\hat{k}$
\end{enumerate}

\item $\overrightarrow{a}=3\hat{i}-5\hat{j}$ and $\overrightarrow{b}=6\hat{i}+3\hat{j}$ are two vectors and $\overrightarrow{c}$ is a vector such that $\overrightarrow{c}=\overrightarrow{a} \times \overrightarrow{b}$ then 
$|\overrightarrow{a}|:|\overrightarrow{b}|:|\overrightarrow{c}|$
\begin{enumerate}
\item $\sqrt{34}:\sqrt{45}:\sqrt{39}$
\item $\sqrt{34}:\sqrt{45}:39$
\item $34:39:45$
\item $39:35:34$
\end{enumerate}

\item If $\overrightarrow{a} \times \overrightarrow{b} = \overrightarrow{b} \times \overrightarrow{c} = \overrightarrow{c} \times \overrightarrow{a}$ then $\overrightarrow{a}+\overrightarrow{b}+\overrightarrow{c}$=
\begin{enumerate}
\item abc
\item -1
\item 0
\item 2
\end{enumerate}

\item The d.r. of a normal to the plane through (1,0,0), (0,1,0) which makes an angle $\pi/4$ with plane x+y=3 are
\begin{enumerate}
\item 1, $\sqrt{2}$, 1
\item 1, 1, $\sqrt{2}$
\item 1, 1, 2
\item $\sqrt{2}$, 1, 1
\end{enumerate}

\item Let $\overrightarrow{u}=\hat{i}+\hat{j}$, $\overrightarrow{v}=\hat{i}-\hat{j}$ and $\overrightarrow{w}=\hat{i}+2\hat{j}+3\hat{k}$. If $\hat{n}$ is a unit vector such that $\overrightarrow{u}.\hat{n}=0$ and $\overrightarrow{v},\hat{n}=0$, then $|\overrightarrow{w}.\hat{n}|$ is equal to
\begin{enumerate}
\item 3
\item 0
\item 1
\item 2
\end{enumerate}

\item A particle acted on by constant forces $4\hat{i}+\hat{j}-3\hat{k}$ and $3\hat{i}+\hat{j}-\hat{k}$ is displaced from the point $\hat{i}+2\hat{j}-3\hat{k}$ to the point $5\hat{i}+4\hat{j}+\hat{k}$. The total work done the by the forces is
\begin{enumerate}
\item 50 units
\item 20 units
\item 30 units
\item 40 units
\end{enumerate}

\item The vectors $\overrightarrow{AB}=3\hat{i}+4\hat{k}$ and $\overrightarrow{AC}=5\hat{i}-2\hat{j}+4\hat{k}$ are the sides of a triangle ABC. The length of the median through A is
\begin{enumerate}
\item $\sqrt{288}$
\item $\sqrt{18}$
\item $\sqrt{72}$
\item $\sqrt{33}$
\end{enumerate}

\item The shortest distance from the plane 
\begin{align}
12x+4y+3z=327
\end{align}
to the sphere
\begin{align}
x^{2}+y^{2}+z^{2}+4x-2y-6z=155
\end{align}
is
\begin{enumerate}
\item 39
\item 26
\item 11$\frac{4}{13}$
\item 13
\end{enumerate}

\item The two lines 
\begin{align*}
\frac{x-2}{1} = \frac{y-3}{1} = \frac{z-4}{-k}
\end{align*}
and 
\begin{align*}
\frac{x-1}{k} = \frac{y-4}{1} = \frac{z-5}{1}
\end{align*}
are coplanar if
\begin{enumerate}
\item k = 3 or -2
\item k = 0 or -1
\item k = 1 or -1
\item k = 0 or -3
\end{enumerate}

\item $\overrightarrow{a}$, $\overrightarrow{b}$, $\overrightarrow{c}$ are 3 vectors such that 
$\overrightarrow{a}+\overrightarrow{b}+\overrightarrow{c}=0$, $|\overrightarrow{a}|=1$, $|\overrightarrow{b}|=2$, $\overrightarrow{c}=3$, then $\overrightarrow{a}.\overrightarrow{b}+\overrightarrow{b}.\overrightarrow{c}+\overrightarrow{c}.\overrightarrow{a}=0$ is equal to
\begin{enumerate}
\item 1
\item 0
\item -7
\item 7
\end{enumerate}

\item The radius of the circle in which the sphere
\begin{align}
x^2+y^2+z^2+2x-2y-4z-19=0 
\end{align}
is cut by the plane
\begin{align}
x+2y+2z+7=0
\end{align}
is
\begin{enumerate}
\item 4
\item 1
\item 2
\item 3
\end{enumerate}

\item A tetrahedron has vertices O(0,0,0), A(1,2,1), B(2,1,3) and C(-1,1,2). Then the angle between the faces OAB and ABC will be
\begin{enumerate}
\item $90^{o}$
\item $\cos^{-1}(\frac{19}{35})$
\item $\cos^{-1}(\frac{17}{31})$
\item $30^{o}$
\end{enumerate}

\item If 
$\begin{vmatrix}
a & a^{2} & 1 + a^{3} \\
b & b^{2} & 1 + b^{3} \\
c & c^{2} & 1 + c^{3} \\
\end{vmatrix}$=0
 and vectors $(1, a, a^{2})$, $(1, b, b^{2})$  and  $(1, c, c^{2})$ are non-coplanar, then the product abc equals
\begin{enumerate}
\item 0
\item 2
\item -1
\item 1
\end{enumerate}

\item Consider a points A, B, C, D with position vectors $7\hat{i}-4\hat{j}+7\hat{k}$, $\hat{i}-6\hat{j}+10\hat{k}$, $-\hat{i}-3\hat{j}+4\hat{k}$ and $5\hat{i}-\hat{j}+5\hat{k}$ respectively. Then ABCD is a 
\begin{enumerate}
\item parallelogram but not a rhombus
\item square
\item rhombus
\item rectangle
\end{enumerate}

\item If $\overrightarrow{u}$, $\overrightarrow{v}$ and $\overrightarrow{w}$ are three non-coplanar vectors, then 
\begin{align*}
(\overrightarrow{u} + \overrightarrow{v} - \overrightarrow{w}).(\overrightarrow{u}-\overrightarrow{v}) \times (\overrightarrow{v}-\overrightarrow{w})
\end{align*}
equals
\begin{enumerate}
\item 3$\overrightarrow{u}.\overrightarrow{v} \times \overrightarrow{w}$
\item 0
\item $\overrightarrow{u}.\overrightarrow{v} \times \overrightarrow{w}$
\item $\overrightarrow{u}.\overrightarrow{w} \times \overrightarrow{v}$
\end{enumerate}

\item Two system of rectangle axes have the same origin. If a plane cuts them at distances a,b,c and $a',b',c'$ from the origin then
\begin{enumerate}
\item $\frac{1}{a^2}+\frac{1}{b^2}+\frac{1}{c^2}-\frac{1}{a^2}-\frac{1}{b^2}-\frac{1}{c^2}=0$
\item $\frac{1}{a^2}+\frac{1}{b^2}+\frac{1}{c^2}+\frac{1}{a^2}+\frac{1}{b^2}+\frac{1}{c^2}=0$
\item $\frac{1}{a^2}+\frac{1}{b^2}-\frac{1}{c^2}+\frac{1}{a^2}+\frac{1}{b^2}-\frac{1}{c^2}=0$
\item $\frac{1}{a^2}-\frac{1}{b^2}-\frac{1}{c^2}+\frac{1}{a^2}-\frac{1}{b^2}-\frac{1}{c^2}=0$
\end{enumerate}

\item Distance between two parallel planes 
\begin{align}
2x+y+2z=8
\end{align}
\begin{align}
4x+2y+4z+5=0
\end{align}
is
\begin{enumerate}
\item $\frac{9}{2}$
\item $\frac{5}{2}$
\item $\frac{7}{2}$
\item $\frac{3}{2}$
\end{enumerate}

\item A line with direction cosines proportional to 2, 1, 2 meets each of the lines 
\begin{align}
x=y+a=z
\end{align}
\begin{align}
x+a=2y=2z
\end{align}
The coordinates os each of the points of intersection are given by
\begin{enumerate}
\item (2a,3a,3a), (2a,a,a)
\item (3a,2a,3a), (a,a,a)
\item (3a,2a,3a), (a,a,2a)
\item (3a,3a,3a), (a,a,a)
\end{enumerate}

\item If the straight lines
\begin{align}
x=1+s, y=-3-\lambda s, z=1+\lambda s
\end{align}
\begin{align}
x=\frac{t}{2}, y=1+t, z=2-t
\end{align}
\begin{enumerate}
\item 0
\item -1
\item $\frac{-1}{2}$
\item -2
\end{enumerate}

\item The intersection of the spheres
\begin{align}
x^2+y^2+z^2+7x-2y-z=13
\end{align}
\begin{align}
x^2+y^2+z^2-3x+3y+4z=8
\end{align}
is the same as the intersection of one of the sphere and the plane
\begin{enumerate}
\item 2x-y-z=1
\item x-2y-z=1
\item x-y-2z=1
\item x-y-z=1
\end{enumerate}

\item Let $\overrightarrow{a}$, $\overrightarrow{b}$ and $\overrightarrow{c}$ be three non zero vectors such that no two of these are collinear. If the vector $\overrightarrow{a}+2\overrightarrow{b}$ is collinear with $\overrightarrow{c}$ and $\overrightarrow{b}+3\overrightarrow{c}$ is collinear with $\overrightarrow{a}$($\lambda$ being some non-zero scalar) then $\overrightarrow{a}+2\overrightarrow{b}+6\overrightarrow{c}$ equals
\begin{enumerate}
\item 0
\item $\lambda\overrightarrow{b}$
\item $\lambda\overrightarrow{c}$
\item $\lambda\overrightarrow{a}$
\end{enumerate}

\item A particle is acted upon by constant forces $4\hat{i}+\hat{j}-3\hat{k}$ and $3\hat{i}+\hat{j}-\hat{k}$ which displace it from a point $\hat{i}+2\hat{j}+3\hat{k}$ to the point $5\hat{i}+4\hat{j}+\hat{k}$. The work done in standard units by the forces is given by
\begin{enumerate}
\item 15
\item 30
\item 25
\item 40
\end{enumerate}

\item If $\overrightarrow{a}$, $\overrightarrow{b}$, $\overrightarrow{c}$ are non-coplanar vectors and $\lambda$ is a real number, then the vctors $\overrightarrow{a}+2\overrightarrow{b}+3\overrightarrow{c}$, $\lambda\overrightarrow{b}+4\overrightarrow{c}$ and $(2\lambda-1)\overrightarrow{c}$ are non coplanar for
\begin{enumerate}
\item no values of $\lambda$
\item all except one value of $\lambda$
\item all except two values of $\lambda$
\item all value of $\lambda$
\end{enumerate}

\item Let $\overrightarrow{u}$, $\overrightarrow{v}$, $\overrightarrow{w}$ be such that $|\overrightarrow{u}|$=1, $|\overrightarrow{v}|$=2, $|\overrightarrow{w}|$=3. If the projection $\overrightarrow{v}$ along $|\overrightarrow{u}|$ is equal to that of $\overrightarrow{w}$ along $\overrightarrow{u}$ and $\overrightarrow{v}$, $\overrightarrow{w}$ are perpendicular to each other then $|\overrightarrow{u}-\overrightarrow{v}+\overrightarrow{w}|$ equals
\begin{enumerate}
\item 14
\item $\sqrt{7}$
\item $\sqrt{14}$
\item 2
\end{enumerate}

\item Let $\overrightarrow{a}$, $\overrightarrow{b}$ and $\overrightarrow{c}$ be non-zero vectors such that $(\overrightarrow{a} \times \overrightarrow{b}) \times \overrightarrow{c}$ = $\frac{1}{3}|\overrightarrow{b}||\overrightarrow{c}|\overrightarrow{a}$. If $\theta$ is the acute angle between the vectors $\overrightarrow{b}$ and $\overrightarrow{c}$, then $\sin\theta$ equals
\begin{enumerate}
\item $\frac{2\sqrt{2}}{3}$
\item $\frac{\sqrt{2}}{3}$
\item $\frac{2}{3}$
\item $\frac{1}{3}$
\end{enumerate}

\item If C is the mid-point of AB and P is any point outside AB, then 
\begin{enumerate}
\item $\overrightarrow{AB}+\overrightarrow{PB}=2\overrightarrow{PC}$
\item $\overrightarrow{PA}+\overrightarrow{PB}=\overrightarrow{PC}$
\item $\overrightarrow{PA}+\overrightarrow{PB}+2\overrightarrow{PC}=\overrightarrow{0}$
\item $\overrightarrow{PA}+\overrightarrow{PB}+\overrightarrow{PC}=\overrightarrow{0}$
\end{enumerate}

\item If the angle $\theta$ between the line 
\begin{align*}
\frac{x+1}{1}=\frac{y-1}{2}=\frac{z-2}{2}
\end{align*}
and the plane $2x-y+\sqrt{\lambda}z+4=0$ is such that $\sin\theta=\frac{1}{3}$ then the value of $\lambda$ is
\begin{enumerate}
\item $\frac{5}{3}$
\item $\frac{-3}{5}$
\item $\frac{3}{4}$
\item $\frac{-4}{3}$
\end{enumerate}

\item The angle between the lines
\begin{align*}
2x=3y=-z
\end{align*}
\begin{align*}
6x=-y=-4z
\end{align*}
is
\begin{enumerate}
\item $0^{o}$
\item $90^{o}$
\item $45^{o}$
\item $30^{o}$
\end{enumerate}

\item If the plane
\begin{align*}
2ax-3ay+4az+6=0
\end{align*}
passes through the midpoint of the line joining the centres of the spheres
\begin{align}
x^2+y^2+z^2+6x-8y-2z=13
\end{align}
\begin{align}
x^2+y^2+z^2-10x+4y-2z=8
\end{align}
the $a$ equals
\begin{enumerate}
\item -1
\item 1
\item -2
\item 2
\end{enumerate}

\item The distance between the lines
\begin{align*}
\overrightarrow{r}=2\hat{i}-2\hat{j}+3\hat{k}+\lambda(i-j+4k)
\end{align*}
and the plane
\begin{align*}
\overrightarrow{r}.(\hat{i}+5\hat{j}+\hat{k})=5
\end{align*}
is
\begin{enumerate}
\item $\frac{10}{9}$
\item $\frac{10}{3\sqrt{3}}$
\item $\frac{3}{10}$
\item $\frac{10}{3}$
\end{enumerate}

\item For any vector $\overrightarrow{a}$, the value of 
\begin{align*}
(\overrightarrow{a} \times \hat{i})^{2} + (\overrightarrow{a} \times \hat{j})^{2} + (\overrightarrow{a} \times \hat{k})^{2}
\end{align*}
is equal to
\begin{enumerate}
\item $3\overrightarrow{a}^{2}$
\item $\overrightarrow{a}^{2}$
\item $2\overrightarrow{a}^{2}$
\item $4\overrightarrow{a}^{2}$
\end{enumerate}

\item If non-zero numbers a,b,c are in H.P., then the straight line $\frac{x}{a}+\frac{y}{b}+\frac{1}{c}=0$ always passes through a fixed point. The point is
\begin{enumerate}
\item (-1, 2)
\item (-1 -2)
\item (1, -2)
\item $(1, \frac{-1}{2}$
\end{enumerate}

\item Let a, b and c be distinct non-negative numbers. If the vectors $a\hat{i}+a\hat{j}+c\hat{k}$, $\hat{i}+\hat{k}$ and $c\hat{i}+c\hat{j}+b\hat{k}$ lie in a plane, then c is
\begin{enumerate}
\item the Geometric Mean of a and b
\item the Arithmetic Mean of a and b
\item equal to zero
\item the Harmonic Mean of a and b
\end{enumerate}

\item If $\overrightarrow{a}$, $\overrightarrow{b}$, $\overrightarrow{c}$ are non-coplanar vectors and $\lambda$ is a real number the 
\begin{align*}
\lambda(\overrightarrow{a}+\overrightarrow{b})\lambda^{2}\overrightarrow{b}\lambda\overrightarrow{c}=[\overrightarrow{a}  (\overrightarrow{b}+\overrightarrow{c}) \overrightarrow{b}]
\end{align*}
for
\begin{enumerate}
\item exactly one value of $\lambda$
\item no value of $\lambda$
\item exactly three values of $\lambda$
\item exactly two values of $\lambda$
\end{enumerate}

\item Let $\overrightarrow{a}=\hat{i}-\hat{k}$, $\overrightarrow{b}=x\hat{i}+\hat{j}+(1-x)\hat{k}$ and $\overrightarrow{c}=y\hat{i}+x\hat{j}+(1+x-y)\hat{k}$. Then $[\overrightarrow{a},\overrightarrow{b},\overrightarrow{c}]$ depends on
\begin{enumerate}
\item only y
\item only x
\item both x and y
\item neither x nor y
\end{enumerate}

\item The plane 
\begin{align*}
x+2y-z=4
\end{align*}
cuts the sphere
\begin{align}
x^2+y^2+z^2-x+z-2=0
\end{align}
in a circle of radius
\begin{enumerate}
\item 3
\item 1
\item 2
\item $\sqrt{2}$
\end{enumerate}

\item If $(\overrightarrow{a} \times \overrightarrow{b}) \times \overrightarrow{c}=\overrightarrow{a} \times (\overrightarrow{b} \times \overrightarrow{c})$ where $\overrightarrow{a}$, $\overrightarrow{b}$ and $\overrightarrow{c}$ are any three vectors such that $\overrightarrow{a}.\overrightarrow{b} \neq 0$, $\overrightarrow{b}.\overrightarrow{c} \neq 0$ then $\overrightarrow{a}$ and $\overrightarrow{c}$ are
\begin{enumerate}
\item inclined at an angle of $\frac{\pi}{3}$ between them
\item inclined at an angle of $\frac{\pi}{6}$ between them
\item perpendicular
\item parallel
\end{enumerate}

\item The values of a , for which points A, B, C with position vectors $2\hat{i}-\hat{j}+\hat{k}$, $\hat{i}-3\hat{j}-5\hat{k}$ and $a\hat{i}-3\hat{j}+\hat{k}$ respectively are the vertices of a right angled triangle with C=$\frac{\pi}{2}$ are
\begin{enumerate}
\item 2 and 1
\item -2 and -1
\item -2 and 1
\item 2 and -1
\end{enumerate}

\item The two lines x=ay+b, z=cy+d; and x=$a'y+b'$, z=$c'y+d'$ are perpendicular to each other if
\begin{enumerate}
\item $aa'+cc'=-1$
\item $aa'+cc'=1$
\item $\frac{a}{a'}+\frac{c}{c'}$=-1
\item $\frac{a}{a'}+\frac{c}{c'}$=1
\end{enumerate}

\item The image of the point (-1,3,4) in the plane x-2y=0 is
\begin{enumerate}
\item $(\frac{-17}{3}, \frac{-19}{3}), 4$
\item (15,11,4)
\item $(\frac{-17}{3}, \frac{-19}{3}), 1$
\item None of these
\end{enumerate}

\item If a line makes an angle of $\pi/4$ with the positive directions of each of x-axis and y-axis, then the angle that the line makes with the positive direction of the z-axis is
\begin{enumerate}
\item $\frac{\pi}{4}$
\item $\frac{\pi}{2}$
\item $\frac{\pi}{6}$
\item $\frac{\pi}{3}$
\end{enumerate}

\item If $\hat{u}$ and $\hat{v}$ are unit vectors and $\theta$ is the acute angle between them, then $2\hat{u} \times 3\hat{v}$ is a unit vector for
\begin{enumerate}
\item no value of $\theta$
\item exactly one value of $\theta$
\item exactly two values of $\theta$
\item more than two values of $\theta$
\end{enumerate}

\item If (2,3,5) is one end of a diameter of the sphere
\begin{align}
x^2+y^2+z^2-6x-12y-2z+20=0
\end{align}
then the coordinates of the other end of the diameter are
\begin{enumerate}
\item (4,3,5)
\item (4,3,-3)
\item (4,9,-3)
\item (4,-3,3)
\end{enumerate}

\item Let $\overrightarrow{a}=\hat{i}+\hat{j}+\hat{k}$, $\overrightarrow{b}=\hat{i}-\hat{j}+2\hat{k}$ and $\overrightarrow{c}=x\hat{i}+(x-2)\hat{j}-\hat{k}$. If the vectors $\overrightarrow{c}$ lies in the plane of $\overrightarrow{a}$ and $\overrightarrow{b}$, then x equals
\begin{enumerate}
\item -4
\item -2
\item 0
\item 1
\end{enumerate}

\item If L be the line of intersection of the planes 
\begin{align*}
2x+3y+z=1
\end{align*}
\begin{align*}
x+3y+2z=2
\end{align*}
If L makes an angle $\alpha$ with the positive x-axis, then $\cos\alpha$ equals
\begin{enumerate}
\item 1
\item $\frac{1}{\sqrt{2}}$
\item $\frac{1}{\sqrt{3}}$
\item $\frac{1}{2}$
\end{enumerate}

\item The vector $\overrightarrow{a}=\alpha\hat{i}+2\hat{j}+\beta\hat{k}$ lies in the plane of the vectors $\overrightarrow{b}=\hat{i}+\hat{j}$ and $\overrightarrow{c}=\hat{j}+\hat{k}$ and bisects the angle between $\overrightarrow{b}$ and $\overrightarrow{c}$. Then whoch one of the following fives possible values of $\alpha$ and $\beta$?
\begin{enumerate}
\item $\alpha=2, \beta=2$
\item $\alpha=1, \beta=2$
\item $\alpha=2, \beta=1$
\item $\alpha=1, \beta=1$
\end{enumerate}

\item The non-zero vectors are $\overrightarrow{a}$, $\overrightarrow{b}$ and $\overrightarrow{c}$ are related by $\overrightarrow{a}=8\overrightarrow{b}$ and $\overrightarrow{c}=7\overrightarrow{b}$. Then the angle between $\overrightarrow{a}$ and $\overrightarrow{c}$ is
\begin{enumerate}
\item 0
\item $\frac{\pi}{4}$
\item $\frac{\pi}{2}$
\item $\pi$
\end{enumerate}

\item The line passing through the points (5,1,a) and (3,b,1) crosses the yz-plane at the point $(0,\frac{17}{2},\frac{-13}{2})$. Then
\begin{enumerate}
\item a=2, b=8
\item a=4, b=6
\item a=6, b=4
\item a=8, b=2
\end{enumerate}

\item If the straight lines 
\begin{align*}
\frac{x-1}{k}=\frac{y-2}{2}=\frac{z-3}{3}
\end{align*}
and 
\begin{align*}
\frac{x-2}{3}=\frac{y-3}{k}=\frac{z-1}{2}
\end{align*}
intersect at a point, then the integer k is equal to
\begin{enumerate}
\item -5
\item 5
\item 2
\item -2
\end{enumerate}

\item Let the line
\begin{align*}
\frac{x-2}{3}=\frac{y-1}{-5}=\frac{z+2}{2}
\end{align*}
lie in the plane x+3y-$\alpha$z+$\beta$=0. Then $(\alpha, \beta)$ equals
\begin{enumerate}
\item (-6,7)
\item (5,-15)
\item (-5,5)
\item (6,-17)
\end{enumerate}

\item The projections of a vector om the three coordinates axis are 6, -3, 2 respectively. The direction cosnies of the vector are:
\begin{enumerate}
\item $\frac{6}{5},\frac{-3}{5},\frac{2}{5}$
\item $\frac{6}{7},\frac{-3}{7},\frac{2}{7}$
\item $\frac{-6}{7},\frac{-3}{7},\frac{2}{7}$
\item 6, -3, 2
\end{enumerate}

\item If $\overrightarrow{u}$, $\overrightarrow{v}$, $\overrightarrow{w}$ are non=coplanar vectors and p,q are real numbers then the equality
\begin{align*}
[3\overrightarrow{u}p\overrightarrow{v}p\overrightarrow{w}]-[p\overrightarrow{v}\overrightarrow{w}q\overrightarrow{u}]-[2\overrightarrow{w}q\overrightarrow{v}q\overrightarrow{u}]=0
\end{align*}
holds for:
\begin{enumerate}
\item exactly two values of (p,q)
\item more than two but not all values of (p,q)
\item all values of (p,q)
\item exactly one value of (p,q)
\end{enumerate}

\item Let $\overrightarrow{a}=\hat{j}-\hat{k}$ and $\overrightarrow{c}=\hat{i}-\hat{j}-\hat{k}$. Then the vector $\overrightarrow{b}$ satisfying $\overrightarrow{a} \times \overrightarrow{b} + \overrightarrow{c}=\overrightarrow{0}$ and $\overrightarrow{a}.\overrightarrow{b}=3$
\begin{enumerate}
\item $2\hat{i}-\hat{j}+2\hat{k}$
\item $\hat{i}-\hat{j}-2\hat{k}$
\item $\hat{i}+\hat{j}-2\hat{k}$
\item $-\hat{i}+\hat{j}-2\hat{k}$
\end{enumerate}

\item If the vectors $\overrightarrow{a}=\hat{i}-\hat{j}+2\hat{2k}$, $\overrightarrow{b}=2\hat{i}+4\hat{j}+\hat{k}$ and $\overrightarrow{c}=\lambda\hat{i}+\hat{j}+\mu\hat{k}$ are mutually orthogonal, then $(\lambda, \mu)$=
\begin{enumerate}
\item 2,-3
\item -2,3
\item 3,-2
\item -3,2
\end{enumerate}

\item \textbf{Statement-1}: The point A(3,1,6) is the mirror image of the point B(1,3,4) in the plane x-y+z=5\\
\textbf{Statement-2}: The plane x-y+z=5 bisects the line segment joining A(3,1,6) and B(1,3,4).
\begin{enumerate}
\item Statement-1 is true, Statement-2 is true; Statement-2 is not a correct explanation for Statement-1
\item Statement-1 is true, Statement-2 is false
\item Statement-1 is false, Statement-2 is true
\item Statement-1 is true, Statement-2 is true; Statement-2 is a correct explanation for Statement-1
\end{enumerate}

\item A line AB in three-dimensional space makes angle $45^{o}$ and $120^{o}$ with the positive x-axis and the positive y-axis respectively. If AB makes an acute angle $\theta$ with the positive z-axis, then $\theta$ equals
\begin{enumerate}
\item $45^{o}$
\item $60^{o}$
\item $75^{o}$
\item $30^{o}$
\end{enumerate}

\item If the angle between the line $x=\frac{y-1}{2}=\frac{z-3}{\lambda}$ and the plane x+2y+3z=4 is $\cos^{-1}(\sqrt{\frac{5}{14}})$, then $\lambda$ equals
\begin{enumerate}
\item $\frac{3}{2}$
\item $\frac{2}{5}$
\item $\frac{5}{3}$
\item $\frac{2}{3}$
\end{enumerate}

\item If $\overrightarrow{a}=\frac{1}{\sqrt{10}}(3\hat{i}+\hat{k})$ and $\overrightarrow{b}=\frac{1}{7}(2\hat{i}+3\hat{j}-6\hat{k})$, then the value of $(2\overrightarrow{a}-\overrightarrow{b})[(\overrightarrow{a} \times \overrightarrow{b}) \times (\overrightarrow{a}+2\overrightarrow{b})]$ is
\begin{enumerate}
\item -3
\item 5
\item -3
\item -5
\end{enumerate}

\item The vectors $\overrightarrow{a}$ and $\overrightarrow{b}$ are not perpendicular and $\overrightarrow{c}$  and $\overrightarrow{d}$ are two vectors satisfying $\overrightarrow{b} \times \overrightarrow{c}=\overrightarrow{b} \times \overrightarrow{d}$ and $\overrightarrow{a}.\overrightarrow{d}=0$. Then the vector $\overrightarrow{d}$ is equal to
\begin{enumerate}
\item $\overrightarrow{c}+(\frac{\overrightarrow{a}.\overrightarrow{c}}{\overrightarrow{a}.\overrightarrow{b}})\overrightarrow{b}$
\item $\overrightarrow{b}+(\frac{\overrightarrow{b}.\overrightarrow{c}}{\overrightarrow{a}.\overrightarrow{b}})\overrightarrow{c}$
\item $\overrightarrow{c}-(\frac{\overrightarrow{a}.\overrightarrow{c}}{\overrightarrow{a}.\overrightarrow{b}})\overrightarrow{b}$
\item $\overrightarrow{b}-(\frac{\overrightarrow{b}.\overrightarrow{c}}{\overrightarrow{a}.\overrightarrow{b}})\overrightarrow{c}$
\end{enumerate}

\item \textbf{Statement-1}: The point A(1,0,7) is the mirror image of the point B(1,6,3) in the line $\frac{x}{1}=\frac{y-1}{2}=\frac{z-2}{3}$\\
\textbf{Statement-2}: The plane $\frac{x}{1}=\frac{y-1}{2}=\frac{z-2}{3}$ bisects the line segment joining A(1,0,7) and B(1,6,3).
\begin{enumerate}
\item Statement-1 is true, Statement-2 is true; Statement-2 is not a correct explanation for Statement-1
\item Statement-1 is true, Statement-2 is false
\item Statement-1 is false, Statement-2 is true
\item Statement-1 is true, Statement-2 is true; Statement-2 is a correct explanation for Statement-1
\end{enumerate}

\item Let $\overrightarrow{a}$ and $\overrightarrow{b}$ be two unit vectors. If the vectors $\overrightarrow{c}=\overrightarrow{a}+2\overrightarrow{b}$ and $\overrightarrow{d}=5\overrightarrow{a}-4\overrightarrow{b}$ are perpendicular to each other, then the angle between $\overrightarrow{a}$ and $\overrightarrow{b}$ is:
\begin{enumerate}
\item $\frac{\pi}{6}$
\item $\frac{\pi}{2}$
\item $\frac{\pi}{3}$
\item $\frac{\pi}{4}$
\end{enumerate}

\item A equation of a plane parallel to the plane 
\begin{align}
x-2y+2z-5=0
\end{align}
and at a unit distance from the origin is:
\begin{enumerate}
\item x-2y+2z-3=0
\item x-2y+2z+1=0
\item x-2y+2z-1=0
\item x-2y+2z+5=0
\end{enumerate}

\item If the line 
\begin{align*}
\frac{x-1}{2}=\frac{y+1}{3}=\frac{z-1}{4}
\end{align*}
and 
\begin{align*}
\frac{x-3}{1}=\frac{y-k}{2}=\frac{z}{1}
\end{align*}
intersect, then k is equal to
\begin{enumerate}
\item -1
\item $\frac{2}{9}$
\item $\frac{9}{2}$
\item 0
\end{enumerate}

\item Let ABCD be a parallelogram such that $\overrightarrow{AB}=\overrightarrow{q}$, $\overrightarrow{AD}=\overrightarrow{p}$ and $\angle BAD$ be an acute angle. If $\overrightarrow{r}$ is the vector that coinside with the altitude directed from the vertex B to the side AD, then $\overrightarrow{r}$ is given by:
\begin{enumerate}
\item $\overrightarrow{r}=3\overrightarrow{q}-\frac{3(\overrightarrow{p}.\overrightarrow{q})}{(\overrightarrow{p}.\overrightarrow{p})}\overrightarrow{p}$
\item $\overrightarrow{r}=-\overrightarrow{q}+\frac{(\overrightarrow{p}.\overrightarrow{q})}{(\overrightarrow{p}.\overrightarrow{p})}\overrightarrow{p}$
\item $\overrightarrow{r}=\overrightarrow{q}-\frac{(\overrightarrow{p}.\overrightarrow{q})}{(\overrightarrow{p}.\overrightarrow{p})}\overrightarrow{p}$
\item $\overrightarrow{r}=-3\overrightarrow{q}-\frac{3(\overrightarrow{p}.\overrightarrow{q})}{(\overrightarrow{p}.\overrightarrow{p})}\overrightarrow{p}$
\end{enumerate}

\item Distance between the two parallel planes
\begin{align*}
2x+y+2z=8
\end{align*}
\begin{align*}
4x+2y+4z+5=0
\end{align*}
is
\begin{enumerate}
\item $\frac{3}{2}$
\item $\frac{5}{2}$
\item $\frac{7}{2}$
\item $\frac{9}{2}$
\end{enumerate}

\item If the lines 
\begin{align*}
\frac{x-2}{1}=\frac{y-3}{1}=\frac{z-4}{-k}
\end{align*}
and 
\begin{align*}
\frac{x-1}{k}=\frac{y-4}{2}=\frac{z-5}{1}
\end{align*}
are coplanr, then k can have
\begin{enumerate}
\item any value
\item exactly one value
\item exactly two values
\item exactly three values
\end{enumerate}

\item If the vectors $\overrightarrow{AB}=3\hat{i}+4\hat{k}$ and $\overrightarrow{AC}=5\hat{i}-2\hat{j}+4\hat{k}$ are the sides of a triangle ABC, then the length of the median through A is
\begin{enumerate}
\item $\sqrt{18}$
\item $\sqrt{72}$
\item $\sqrt{33}$
\item $\sqrt{45}$
\end{enumerate}

\item The image of the line $\frac{x-1}{3}=\frac{y-3}{1}=\frac{z-4}{-5}$ in the plane 2x-y+z+3=0 is the line:
\begin{enumerate}
\item $\frac{x-3}{3}=\frac{y+5}{1}=\frac{z-2}{-5}$
\item $\frac{x-3}{-3}=\frac{y+5}{-1}=\frac{z-2}{5}$
\item $\frac{x+3}{3}=\frac{y-5}{1}=\frac{z-2}{-5}$
\item $\frac{x+3}{-3}=\frac{y-5}{-1}=\frac{z+2}{5}$ 
\end{enumerate}

\item The angle between the lines whose direction cosines satisfy the equations
\begin{align*}
l+m+n=0
\end{align*}
\begin{align*}
l^2=m^2+n^2
\end{align*}
is
\begin{enumerate}
\item $\frac{\pi}{6}$
\item $\frac{\pi}{2}$
\item $\frac{\pi}{3}$
\item $\frac{\pi}{4}$
\end{enumerate}

\item Let $\overrightarrow{a}$, $\overrightarrow{b}$ and $\overrightarrow{c}$ be three non-zero vectors such that no two of them are collinear and 
\begin{align*}
(\overrightarrow{a} \times \overrightarrow{b}) \times \overrightarrow{c}=\frac{1}{3}|\overrightarrow{b}||\overrightarrow{c}|\overrightarrow{a}
\end{align*}
If $\theta$ is the angle between vectors $\overrightarrow{b}$ and $\overrightarrow{c}$, then a value of $\sin\theta$ is:
\begin{enumerate}
\item $\frac{2}{3}$
\item $\frac{-2\sqrt{3}}{3}$
\item $\frac{2\sqrt{2}}{3}$
\item $\frac{-\sqrt{2}}{3}$
\end{enumerate}

\item The equation of the plane containing the line
\begin{align*}
2x-5y+z=3
\end{align*}
\begin{align*}
x+y+4z=5
\end{align*}
and parallel to the plane x+3y+6z=1 is:
\begin{enumerate}
\item x+3y+6z=7
\item 2x+6y+12z=-13
\item 2x+6y+12z=13
\item x+3y+6z=-7
\end{enumerate}

\item The distance of the point (1,0,2) from the point of intersection of the line 
\begin{align*}
\frac{x-2}{3}=\frac{y+1}{4}=\frac{z-2}{12}
\end{align*}
and the plane x-y+z=16, is
\begin{enumerate}
\item $3\sqrt{21}$
\item 13
\item $2\sqrt{14}$
\item 8
\end{enumerate}

\item If the line 
\begin{align*}
\frac{x-3}{2}=\frac{y+2}{-1}=\frac{z+4}{3}
\end{align*}
lies in the plane, lx+my-z=9, then $l^2+m^2$ is equal to:
\begin{enumerate}
\item 5
\item 2
\item 26
\item 18
\end{enumerate}

\item Let $\overrightarrow{a}$, $\overrightarrow{b}$ and $\overrightarrow{c}$ be three unit vectors such that
\begin{align*}
\overrightarrow{a} \times (\overrightarrow{b} \times \overrightarrow{c})=\frac{\sqrt{3}}{2}(\overrightarrow{b}+\overrightarrow{c})
\end{align*}
If $\overrightarrow{b}$ is not parallel to $\overrightarrow{c}$, then the angle between $\overrightarrow{a}$ and $\overrightarrow{b}$ is:
\begin{enumerate}
\item $\frac{2\pi}{3}$
\item $\frac{5\pi}{6}$
\item $\frac{3\pi}{4}$
\item $\frac{\pi}{2}$
\end{enumerate}

\item The distance of the point (1,-5,9) from the plane x-y+z=5 measured along the line x=y=z is:
\begin{enumerate}
\item $\frac{10}{\sqrt{3}}$
\item $\frac{20}{3}$
\item $3\sqrt{10}$
\item $10\sqrt{3}$
\end{enumerate}

\item Let $\overrightarrow{a}=2\hat{i}+\hat{j}-2\hat{k}$ and $\overrightarrow{b}=\hat{i}+\hat{j}$. Let $\overrightarrow{c}$ be a vector such that
\begin{align*}
|\overrightarrow{c}-\overrightarrow{a}|=3, |(\overrightarrow{a} \times \overrightarrow{b}) \times \overrightarrow{c}|=3
\end{align*}
and the angle between $\overrightarrow{c}$ and $\overrightarrow{a} \times \overrightarrow{b}$ be $30^{o}$. Then $\overrightarrow{a}.\overrightarrow{c}$ is equal to:
\begin{enumerate}
\item $\frac{1}{8}$
\item $\frac{25}{8}$
\item 2
\item 5
\end{enumerate}

\item If the image of the point P(1,-2,3) in the plane 2x+3y-4z+22=0 measured parallel to line $\frac{x}{1}=\frac{y}{4}=\frac{z}{5}$ is Q, then PQ is equal to
\begin{enumerate}
\item $6\sqrt{5}$
\item $3\sqrt{5}$
\item $2\sqrt{42}$
\item $\sqrt{42}$
\end{enumerate}

\item The distance of the point (1,3,-7) from the plane passing through the point (1,-1,-1) having normal perpendicular to both the lines
\begin{align*}
\frac{x-1}{1}=\frac{y+2}{-2}=\frac{z-4}{3}
\end{align*}
and
\begin{align*}
\frac{x-2}{2}=\frac{y+1}{-1}=\frac{z+7}{-1}
\end{align*}
is
\begin{enumerate}
\item $\frac{10}{\sqrt{74}}$
\item $\frac{20}{\sqrt{74}}$
\item $\frac{10}{\sqrt{83}}$
\item $\frac{5}{\sqrt{83}}$
\end{enumerate}

\item Let $\overrightarrow{u}$ be a vector coplanr with the vectors $\overrightarrow{a}=2\hat{i}+3\hat{j}-\hat{k}$ and 
$\overrightarrow{b}=\hat{j}+\hat{k}$. If $\overrightarrow{u}$ is perpendicular to $\overrightarrow{a}$ and $\overrightarrow{u}.\overrightarrow{b}-24$, then $|\overrightarrow{u}|^2$ is equal to:
\begin{enumerate}
\item 315
\item 256
\item 84
\item 336
\end{enumerate} 

\item The length of the projection of the line segment joining the points (5,-1,4) and (4,-1,3) on the plane x+y+z=7 is:
\begin{enumerate}
\item $\frac{2}{3}$
\item $\frac{1}{3}$
\item $\sqrt{\frac{2}{3}}$
\item $\frac{2}{\sqrt{3}}$
\end{enumerate}

\item If $L_1$ is the line of intersection of the planes
\begin{align*}
2x-2y+3z-2=0
\end{align*}
\begin{align*}
x-y+z+1=0
\end{align*}
and $L_2$ is the line of intersection of the planes
\begin{align*}
x+2y-z-3=0
\end{align*}
\begin{align*}
3x-y+2z-1=0
\end{align*}
then the distance of the origin from the plane, containing the lines $L_1$ and $L_2$ is
\begin{enumerate}
\item $\frac{1}{3\sqrt{2}}$
\item $\frac{1}{2\sqrt{2}}$
\item $\frac{1}{\sqrt{2}}$
\item $\frac{1}{4\sqrt{2}}$
\end{enumerate}

\item Let $\overrightarrow{a}=\hat{i}-\hat{j}$, $\overrightarrow{b}=\hat{i}+\hat{j}+\hat{k}$  and $\overrightarrow{c}$ be a vector such that $\overrightarrow{a} \times \overrightarrow{c}+\overrightarrow{b}=\overrightarrow{0}$ and $\overrightarrow{a}.\overrightarrow{c}=4$, then $|\overrightarrow{c}|^2$ is equal to
\begin{enumerate}
\item $\frac{19}{2}$
\item 9
\item 8
\item $\frac{17}{2}$
\end{enumerate}

\item The equation of the line passing through the point (-4,3,1), parallel to the plane x+2y-z-5=0 and intersecting the line $\frac{x+1}{-3}=\frac{y-3}{2}=\frac{z-2}{-1}$ is
\begin{enumerate}
\item $\frac{x-4}{-3}=\frac{y+3}{1}=\frac{z+1}{4}$
\item $\frac{x+4}{1}=\frac{y-3}{1}=\frac{z-1}{3}$
\item $\frac{x+4}{3}=\frac{y-3}{-1}=\frac{z-1}{1}$
\item $\frac{x+4}{-1}=\frac{y-3}{1}=\frac{z-1}{1}$
\end{enumerate}

\item The plane through the inersection of the planes x+y+z=1 and 2x+3y-z+4=0 and parallel to y-axis also passes through the point:
\begin{enumerate}
\item (-3,0,-1)
\item (-3,1,1)
\item (3,3,-1)
\item (3,2,1)
\end{enumerate}

\item If the line $\frac{x-1}{2}=\frac{y+1}{3}=\frac{z-2}{4}$ meets the plane, x+2y+3z=15 at a point P, then the distance of P from the origin is
\begin{enumerate}
\item $\frac{\sqrt{5}}{2}$
\item $2\sqrt{5}$
\item $\frac{9}{2}$
\item $\frac{7}{2}$
\end{enumerate}

\item A plane passing through the points (0, -1, 0) and (0, 0, 1) and making an angle $\frac{\pi}{4}$ with the plane y - z + 5 = 0, also passes through the point:
\begin{enumerate}
\item $(-\sqrt{2}, 1, -4)$
\item $(\sqrt{2}, -1, 4)$
\item $(-\sqrt{2}, -1, -4)$
\item $(\sqrt{2}, 1, 4)$
\end{enumerate}

\item Let $\overrightarrow{\alpha} = 3\hat{i + \hat{j}}$ and $\overrightarrow{\beta} = 2\hat{i} - \hat{j} + 3\hat{k}$. If $\overrightarrow{\beta} = \overrightarrow{\beta_1} - \overrightarrow{\beta_2}$, where $\overrightarrow{\beta_1}$ is parallel to $\overrightarrow{\alpha}$ and $\overrightarrow{\beta_2}$ is perpendicular to $\overrightarrow{\alpha}$, then $\overrightarrow{\beta_1} \times \overrightarrow{\beta_2}$ is equal to:
\begin{enumerate}
\item $-3\hat{i} + 9\hat{j} + 5\hat{k}$
\item $3\hat{i} - 9\hat{j} - 5\hat{k}$
\item $\frac{1}{2}(-3\hat{i} + 9\hat{j} + 5\hat{k})$
\item $\frac{1}{2}(3\hat{i} - 9\hat{j} + 5\hat{k})$
\end{enumerate}





































\end{enumerate}

\end{document}


